% -*- root: 00-main.tex -*-
\section*{Conclusion}
\label{sec:conclusion}

\Regseg{} is a variational framework for the simultaneous segmentation and
  registration of 3D \gls*{dmri} data obtained from the human brain, where within-subject
  anatomical information is used as a reference.
The registration method segments the target multivariate image into several competing regions, which are
  defined explicitly by their limiting surfaces.
The surfaces are active and they evolve on a free-form deformation field supported by the B-spline basis.
A descent optimization strategy is guided by shape gradients computed on the current partition
  of the target image.
\Regseg{} uses active contours without edges and it searches for
  homogeneous regions within the image.
We tested \regseg{} using digital phantoms by simulating \gls*{t1} and \gls*{t2} \gls*{mri}
  warped with smooth random deformations.
The resulting misregistration of the contours was significantly lower than the image resolution
  of the phantoms.

We proposed \regseg{} for simultaneously segmenting \gls*{dmri} data and registering them according to
  their corresponding \gls*{t1} image from the same subject.
We demonstrated 
  the accuracy of the proposed method based on visual assessments of the results obtained by \regseg{} and cross-comparisons with a widely used technique.
Moreover, \regseg{} does not require any images in addition to the minimal acquisition protocol,
  which only utilizes \gls*{t1} and \gls*{dmri}.
As well as the proposed application to \gls*{dmri} data, other potential uses of \regseg{} are
  atlas-based segmentation and tracking objects in time-series.


\section*{Availability and reproducibility statement}
\label{sec:availability}
We considered the reproducibility of our results as a design requirement.
Therefore, we used real data obtained from a publicly available repository
  (the Human Connectome Project \citep{essen_human_2012}) and all of the software
  utilized in this study is also publicly available.
\Regseg{} was implemented on top of ITK-4.6 (Insight Registration and 
  Segmentation Toolkit, \url{http://www.itk.org}).
The evaluation instruments (\autoref{fig:evworkflows}) were implemented using
  \emph{nipype} \citep{gorgolewski_nipype_2011} to assess their reproducibility.
All of the research elements (data, source code, figures, manuscript sources, etc.) involved in this study
  are publicly available under a unique package \citep{esteban_acweregistration_2015}. 