% -*- root: 00-main.tex -*-
\section*{Conclusion}
\label{sec:conclusion}

\emph{Regseg} is a variational framework to simultaneously segment and
  register 3D \gls*{dmri} data of the human brain, using within-subject
  anatomical information as reference.
The registration method segments the target multivariate image in several competing regions
  defined explicitly by their limiting surfaces.
The surfaces are active, and evolve on a B-Spline displacements field, guided by shape
  gradients computed on the current partition of the target image.
\emph{Regseg} uses active contours without edges, and hence looks for
  homogeneous regions within the image.
We tested \emph{regseg} on digital phantoms simulating \gls*{t1} and \gls*{t2} \gls*{mri},
	warped with smooth random deformations.
The resulting average misplacement of the contours was significantly lower than the
  image resolution of the phantoms.

We propose \emph{regseg} to correct \gls*{dmri} datasets for the typically present distortions
  caused by inhomogeneities of the main magnetic field.
Visual assessment of \emph{regseg} and cross-comparison with a widely-used competing
  technique demonstrated its accuracy.
Moreover, \emph{regseg} does not require additional images to the minimal acquisition protocol
  that includes only \gls*{t1} and \gls*{dmri}.

\section*{Availability and reproducibility statement}
\label{sec:availability}
We considered the reproducibility of our results a design requirement.
Therefore, we used real data fetched from a publicly available repository
  (the Human Connectome Project \citep{essen_human_2012}) and all the software
  involved in this paper is also publicly released.
The \emph{regseg} algorithm is implemented in C++, on top of ITK-4.6
  (Insight Registration and Segmentation Toolkit, \url{http://www.itk.org}).
The evaluation instruments (\autoref{fig:evworkflows}), are implemented in Python using
  \emph{nipype} \citep{gorgolewski_nipype_2011} to assess their reproducibility.
Phantom data are automatically generated by the evaluation framework.
All research elements (data, source code, figures, manuscript sources, etc.) in this paper
  are made publicly available under a unique digital object identifier in a
  package \citep{esteban_acweregistration_2015}.