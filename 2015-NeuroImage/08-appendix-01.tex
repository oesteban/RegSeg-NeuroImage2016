% if have a single appendix:
%\appendix[Proof of the Zonklar Equations]
% or
%\appendix  % for no appendix heading
% do not use \section anymore after \appendix, only \section*
% is possibly needed

% use appendices with more than one appendix
% then use \section to start each appendix
% you must declare a \section before using any
% \subsection or using \label (\appendices by itself
% starts a section numbered zero.)
%

\appendix

\section{Proof of duality between the probabilistic and the presented frameworks}
Here the demonstration that we are preparing.

% you can choose not to have a title for an appendix
% if you want by leaving the argument blank
%\section{}
%Appendix two text goes here.

In the probabilistic approach to image segmentation, the smoothness constraints
can be introduced into the \gls{map} criterium as a \gls{mrf}. Accordingly to the
Hammersley-Clifford theorem (\textbf{FIXME}: citations needed), an \gls{mrf} can
be equivalently characterized by a Gibb's distribution:
\begin{equation}
P(\mathbf{K})=\underset{i}{\prod} Z(U)^{-1}\,e^{-U(\mathbf{x}_i \mid \beta,K)} = \mathit{const.}\,e^{- \sum\limits_i \left( U(\mathbf{x}_i \mid \beta,K) \right) } \sim e^{-\nu_B \left| K \right| }.
\end{equation}
This way, we include the assumption that the total length of the edge set $K$ is small
and we draw the equivalence with the \gls{mrf} modeling, for $U(\mathbf{x}_i \mid \beta,K)$
being the Potts model.

In the probabilistic approach, the underlying concept to segmenting an image is the
Bayes' rule. For a discrete image of $N$ pixels indexed by $i$, this rule reads as
follows:
\begin{equation}
P(K|I)=P(I \mid K)\, P(K)=\underset{i}{\prod} p (\mathbf{x}_i | K ) \, p_i(K).
\end{equation}
The normalizer probability $P(I)$ has been ommited for simplicity.

In order to express the likelihood as an absolute energy in our variation framework,
the \textit{log}-likelihood is computed:
\begin{equation}
E(K)= -\log( P(K|I) ) = \sum\limits_k \int_{\Omega_k} -\log p_k(I(\mathbf{x}),\mathbf{x}) \,d\mathbf{x}+\nu_B \left|K\right|,
\end{equation}
where $\mathbf{s} = I(\mathbf{x})$, $\mathbf{s} \in \mathbb{R}^C$ (a vector of $C$ scalar features).
Additionally, the discrete grid of $N$ pixels has been converted to a continuous space by assuming
$d\mathbf{x}$ as infinitesimal bin size.

To explicitly define the likelihood, we firstly define the squared \textit{Mahalanobis distance} that
is the exponential of a multivariate normal distribution:
\begin{equation}
\Delta^2_k (\mathbf{s}) = (\mathbf{s} - \boldsymbol{\mu}_k)^T \, \Sigma^{-1}_k \, (\mathbf{s} - \boldsymbol{\mu}_k).
\end{equation}

Therefore, the likelihood is defined as follows:
\begin{equation}
p_k(I(\mathbf{x}),\mathbf{x}) = p_k(\mathbf{s},\mathbf{x})= p_k(\mathbf{s}) = \frac{1}{ \sqrt{(2\pi)^{C}\,\left|\boldsymbol{\Sigma}_{k}\right|}}\,{e^{\left(-\frac{1}{2}  \Delta^2_k (\mathbf{s}) \right)}}.
\end{equation}

Introducing this definition on (14), we have:
\begin{equation}
E(K)= \sum\limits_k \int_{\Omega_k} -\log{\left[ \frac{1}{ \sqrt{(2\pi)^{C}\,\left|\boldsymbol{\Sigma}_{k}\right|}}\,e^{\left(-\frac{1}{2}  \Delta^2_k (\mathbf{s}) \right)} \right] } \,d\mathbf{x}+\nu_B \left|K\right|.
\end{equation}

\begin{equation}
E(K) = \sum\limits_k \int_{\Omega_k} \left( \frac{1}{2} \log{ \left( (2\pi)^{C}\,\left|\boldsymbol{\Sigma}_{k}\right| \right)} + \frac{1}{2}  \Delta^2_k (\mathbf{s}) \right) \,d\mathbf{x}+\nu_B \left|K\right|
\end{equation}

\begin{equation}
E(K) = \sum\limits_k \left( \frac{ V_k }{2} \log{ \left( (2\pi)^{C}\,\left|\boldsymbol{\Sigma}_{k}\right| \right)}+ \frac{1}{2} \int_{\Omega_k} \Delta^2_k (\mathbf{s}) \,d\mathbf{x}+\nu_B \left|K\right| \right)
\end{equation}

{\color{red} {Using the region descriptors derived in \autoref{sec:regseg-methods_map}, we propose
an \gls{adf}-like, \gls{acwe}-based, piece-wise constant, image segmentation
model (where the unknown is the deformation field)
\cite{chan_active_2001} with the energy functional obtained in
\eqref{eq:regseg-final_map_energy}. This inverse problem is ill-posed
\cite{bertero_illposed_1988,hadamard_sur_1902}.
In order to account for deformation field regularity and to render the
problem well-posed, we include limiting and regularization terms into
the energy functional \cite{morozov_linear_1975,tichonov_solution_1963}:
%
\begin{multline}
E(u) = \sum\limits_l \int_{\Omega'_l} \mdist{f}{l} \,d\vec{x} \\
+ \int \vec{u}^T \, A \, \vec{u} \, d\vec{x} + \int \tr\{(\nabla \vec{u}^T)^T B (\nabla \vec{u}^T)\} d\vec{x}
\label{eq:regseg-complete_energy}
\end{multline}
%
These regularity terms ensure that the segmenting contours in
\gls{dwi} space are still close to their native shape. The model
easily allows to incorporate inhomogeneous and anisotropic
regularization \cite{nagel_investigation_1986} to better regularize
the \gls{epi} distortion.}}

\todo[inline]{this last paragraphs in red color were located a bit
further, but I think we can keep some of the refs and still reproduce
the continuous version of the energy functional}

\section{Extensions}

\subsubsection{Semi-implicit Euler-forward optimization}
It is necessary to discretize $t$ in order to obtain a numerical
implementation of the equation \eqref{eq:regseg-gradient_final}:
\begin{align}
\frac{\vec{u}_k^{t+1}-\vec{u}_k^t}{\tau} =
&- \underset{l,m}{\sum} \underset{i}{\sum}
\left[ \mdist{f_i'}{l} - \mdist{f_i'}{m}\right]
\psi_k(\vec{c}_i)\, \hat{\vec{n}}_i \notag\\
&+2\, \boldsymbol{\alpha} \vec{u}_k^{t+1}
-2\, \boldsymbol{\beta} \Delta \vec{u}_k^{t+1}.
\end{align}

The associated Euler-Lagrange equation is found as:
\begin{equation}
(\tau^{-1} +2\, \boldsymbol{\alpha} - 2\, \boldsymbol{\beta} \Delta )\, \vec{u}_k^{t+1} =
\tau^{-1} \vec{u}_k^t - \frac{\partial}{\partial \vec{u}_k} E_{data}(\vec{u}_k^t),
\end{equation}
that is a linear system that we translate into Fourier domain,
to obtain the next deformation field $\vec{u}_k^{t+1}$:
\begin{multline}
\vec{u}_k^{t+1} = \\
 \mathcal{F}^{-1} \left\lbrace
\frac{\mathcal{F}\lbrace \tau^{-1}\vec{u}_{k}^{t} - \underset{l,m}{\sum} \underset{i}{\sum}
\left[ \mdist{f_i'}{l} - \mdist{f_i'}{m}\right] \rbrace}
     {\mathcal{F}\lbrace (\tau^{-1} +2\,\boldsymbol{\alpha})\mathcal{I} - 2\,\boldsymbol{\beta} \Delta ) \rbrace}
     \right\rbrace
\end{multline}

Here, we rewrite the Laplacian as a linear combination of the identity and shift operators:
\begin{equation}
\Delta = \sum\limits_d \mathcal{S}_d^- + \mathcal{S}_d^+ - 2 \mathcal{I}
\end{equation}
where $\mathcal{S}_{d}^{\pm}$ stands for the forward ($+$) and backward ($-$) shift
operator along coordinates axis $d$, of which the Fourier transform is found easily as
\begin{equation}
\mathcal{F}\{\mathcal{S}_{d}^{\pm}\} = e^{\pm \vec{i}\omega_{d}},
\end{equation}
where $\omega_{d}$ is the normalized pulsation along $d$-direction. Accordingly, the
Fourier transform of the discrete Laplacian is found as
\begin{equation}
\mathcal{F}\{\Delta\} = \sum\limits_d e^{-\vec{i}\omega_d } + e^{\vec{i}\omega_d } - 2 = \sum\limits_d \left( 2\cos(\omega_d) - 2 \right)
\end{equation}

The remaining transforms are trivial or can be computed using the \gls{fft}
as in \citep{estellers_efficient_2011}.