% -*- root: 00-main.tex -*-
\revcomment[RV\#1(C.1)\par RV\#2(C.10)]{%
Current methods for processing \gls*{dmri} to map the connectivity of the human brain
  require precise delineations of anatomical structures.
The requirement has been approached either segmenting the data in native \gls*{dmri} space
  or projecting the structural information from \gls*{t1} images.
The characteristic features of diffusion data in terms of \acrlong*{snr}, resolution, etc.
  and the geometrical distortions caused by the inhomogeneity of magnetic susceptibility
  across tissues render the problem challenging.
To solve the problem in a unified approach, we propose \regseg{}, a surface-to-volume nonlinear
  registration method that segments homogeneous regions within multivariate images by mapping
  a set of nested reference-surfaces.}
\revcomment[RV\#2(C.11)]{%
The surfaces can be extracted from the \gls*{t1} image of the subject, and the target image
  is the bivariate volume comprehending the \gls*{fa} and the \gls*{adc} maps derived from the
  \gls*{dmri} dataset.}
We first verify the subpixel accuracy of \regseg{} on digital phantoms, to then propose an
  evaluation framework that uses \gls*{dmri} datasets from the \gls*{hcp}.
In this framework, the undistorted \gls*{dmri} undergo a known distortion derived from real
  fieldmaps.
We analyze the misregistration error of the surfaces estimated by \regseg{} on 16 datasets.
The distribution of errors shows a 95\% CI of 0.56--0.66 mm, below the \gls*{dmri} resolution
  (1.25 mm, isotropic).
Finally, we cross-compare the proposed tool against a nonlinear \lowb{}-to-T2w registration
  method, thereby obtaining a significantly lower misregistration error with \regseg{}.