% -*- root: 00-main.tex -*-
\revcomment[RV\#1(C.1)\par RV\#2(C.10)]{%
Current methods for processing \gls*{dmri} to map the connectivity of the human brain
  require precise delineations of anatomical structures.
Image segmentation on the native \gls*{dmri} with sub-pixel accuracy has been unsuccessful,
  and the projection of the structural information from \gls*{t1} images is challenging
  since \gls*{dmri} data present geometrical distortions.
To solve this problem, we propose \regseg{}, a surfaces-to-volume nonlinear registration method
  that segments homogeneous regions within multivariate images by projecting a set of nested
  reference-surfaces.}
We first verify the subpixel accuracy of \regseg{} on digital phantoms,
\revcomment[RV\#2(C.11)]{%
and then we evaluate the tool in the aforementioned application.
We propose a set of surfaces extracted from the \gls*{t1} image to successfully segment
  the bivariate volume comprehending the \gls*{fa} and the \gls*{adc} maps derived from the
  corresponding \gls*{dmri} dataset of the same subject.}
We present an evaluation framework, in which the target \gls*{dmri} are deformed with
  magnetic susceptibility -derived and realistic distortions.
Using the framework, we analyze the misregistration error of the surfaces estimated by \regseg{}
  on 16 real datasets of the Human Connectome Project, thereby obtaining an error distribution
  with 95\% CI of 0.56--0.66 mm, below the \gls*{dmri} resolution (1.25 mm, isotropic).
Finally, we cross-compare \regseg{} against a widely-used volume-to-volume registration method,
  and we obtain a significantly lower misregistration error with the proposed tool.