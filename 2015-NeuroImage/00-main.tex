\documentclass[5p,authoryear]{elsarticle}
% \documentclass[authoryear,onecolumn]{elsarticle}
\journal{ArXiv archiving for pre-publication review}

% *** MISC UTILITY PACKAGES ***
\usepackage{glossaries}
\usepackage{booktabs} % to use \toprule
\usepackage{multirow}
\usepackage[colorinlistoftodos]{todonotes}
\usepackage{xifthen}
\usepackage{xspace}

%\usepackage[cmex10,intlimits]{amsmath}
\usepackage{amsmath}
\usepackage{amssymb}
%\usepackage{algorithmic}
\DeclareMathOperator{\tr}{tr}
\DeclareMathOperator{\argmax}{argmax}
\DeclareMathOperator{\argmin}{argmin}
\DeclareMathOperator{\diag}{diag}
\DeclareMathOperator{\dist}{dist}
\DeclareMathOperator{\const}{Const.}
\providecommand{\e}[1]{\ensuremath{\times 10^{#1}}}
\providecommand{\mdist}[2]{ \mathcal{D}_{#2}^2(\mathbf{#1}) }
\providecommand{\omegaset}{\ensuremath{\boldsymbol{\Omega}}}
\providecommand{\gammaset}{\ensuremath{\boldsymbol{\Gamma}}}

\let\oldhat\hat
\renewcommand{\vec}[1]{\mathbf{#1}}

\providecommand{\nvec}[1]{\hat{\mathbf{#1}}}
\providecommand{\suppl}[1]{\href{http://dx.doi.org/mydoi}{Supplemental Materials}, #1}
\providecommand{\isores}[2][]{#2\ensuremath{\times}#2\ensuremath{\times}#2%
\ifthenelse{\equal{#1}{}}{}{ [#1]}\xspace}

% *** PDF, URL AND HYPERLINK PACKAGES ***
%
\usepackage{url}
\usepackage[colorlinks=true,linkcolor=black, citecolor=blue, urlcolor=blue]{hyperref}
\usepackage{doi}

% correct bad hyphenation here
\hyphenation{op-tical net-works semi-conduc-tor an-isot-ropy regis-tra-tion iso-tropic Free-Surfer %
align-ed pre-sent work-flow data-set data-sets}

% -*- root: 00-main.tex -*-
% List of acronyms used in the text
\newacronym{mr}{MR}{magnetic resonance}
\newacronym{mri}{MRI}{magnetic resonance imaging}
\newacronym{dwi}{DWI}{diffusion weighted image}
\newacronym{dmri}{dMRI}{diffusion MRI}
\newacronym{dw}{DW}{diffusion weighted}
\newacronym{dti}{DTI}{diffusion tensor image}
\newacronym{t1}{T1w}{T1 weighted}
\newacronym{t2}{T2w}{T2 weighted}
\newacronym{csf}{CSF}{cerebrospinal fluid}
\newacronym{wm}{WM}{white matter}
\newacronym{gm}{GM}{grey matter}
\newacronym{epi}{EPI}{echo-planar imaging}
\newacronym{gre}{GRE}{gradient echo sequence}
\newacronym{fa}{FA}{fractional anisotropy}
\newacronym{md}{MD}{mean diffusivity}
\newacronym{adc}{ADC}{apparent diffusion coefficient}
\newacronym{acwe}{ACWE}{active contours without edges}
\newacronym{adf}{ADF}{active deformation field}
\newacronym{map}{MAP}{maximum a posteriori}
\newacronym{snr}{SNR}{signal-to-noise ratio}
\newacronym{pve}{PVE}{partial volume effect}
\newacronym{roi}{ROI}{region of interest}
\newacronym{mrf}{MRF}{Markov Random Field}
\newacronym{pde}{PDE}{partial differential equation}
\newacronym{swindex}{sWI}{surface warping index}
\newacronym{wi}{WI}{Warping index}
\newacronym{nof}{NoF}{number of fibers}
\newacronym{t2b}{T2B}{T2w-registration based}
\newacronym{fft}{FFT}{fast fourier transform}
\newacronym{hcp}{HCP}{Human Connectome Project}
\newacronym{pe}{PE}{phase-encoding}

\makeglossaries

%\acrodef{mr}[MR]{magnetic resonance}
%\acrodef{mri}[MRI]{magnetic resonance imaging}
%\acrodef{dwi}[DWI]{diffusion weighted imaging}
%\acrodef{dw}[DW]{diffusion weighted}
%\acrodef{dti}[DTI]{diffusion tensor imaging}
%\acrodef{t1}[T1]{T1-weighted}
%\acrodef{t2}[T2]{T2-weighted}
%\acrodef{csf}[CSF]{cerebrospinal fluid}
%\acrodef{wm}[WM]{white matter}
%\acrodef{gm}[GM]{grey matter}
%\acrodef{epi}[EPI]{echo-planar imaging}
%\acrodef{fa}[FA]{fractional anisotropy}
%\acrodef{md}[MD]{mean diffusivity}
%\acrodef{acwe}[ACWE]{active contours without edges}
%\acrodef{map}[MAP]{maximum a posteriori}
%\acrodef{snr}[SNR]{signal-to-noise ratio}
%\acrodef{pve}[PVE]{partial volume effect}
%\acrodef{roi}[ROI]{region of interest}


\usepackage[T1]{fontenc}
\usepackage{charter}
%\usepackage[expert]{mathdesign}
%\usepackage[libertine,cmintegrals,cmbraces,vvarbb]{newtxmath}
%\renewcommand*{\familydefault}{\sfdefault}
\usepackage[switch]{lineno}
\usepackage{array}
\newcolumntype{L}[1]{>{\raggedright\let\newline\\\arraybackslash\hspace{0pt}}m{#1}}


% http://tex.stackexchange.com/questions/56105/missing-references-heading
\def\bibsection{\section*{References}}
\begin{document}
\begin{frontmatter}

%% Use \dochead if there is an article header, e.g. \dochead{Short communication}
% \dochead{Preprint for ArXiv archiving and pre-submission review}

\title{Active-contours driven registration method for structure-informed segmentation of diffusion MR images}

\author[bit,ciber]{Oscar~Esteban\corref{corrauthor}}
\cortext[corrauthor]{Corresponding author}
\ead{phd@oscaresteban.es}
\author[ucla]{Dominique~Zosso}
\author[scil,lts5]{Alessandro~Daducci}
\author[chuv,lts5]{Meritxell~Bach-Cuadra}
\author[bit,ciber]{Mar\'ia-J.~Ledesma-Carbayo}
\author[lts5]{Jean-Philippe~Thiran}
\author[bit,ciber]{Andres~Santos}


\address[bit]{Biomedical Image Technologies (BIT), ETSI Telecomunicaci\'on, %
Universidad Polit\'ecnica de Madrid, Madrid, Spain}
\address[ciber]{Centro de Investigaci\'on Biom\'edica en Red en Bioingenier\'ia, Biomateriales y Nanomedicina (CIBER-BBN), Zaragoza, Spain}
\address[ucla]{Department of Mathematics, University of California,
Los Angeles (UCLA), Los Angeles, CA, US}
\address[scil]{Computer Science Department, Faculty of Science, Universit\'e de Sherbrooke, 2500 Boulevard Universit\'e, Sherbrooke, QC J1K 2R1, Canada}
\address[lts5]{Signal Processing Laboratory (LTS5), \'Ecole Polytechnique
F\'ed\'erale de Lausanne (EPFL), Lausanne, Switzerland}
\address[chuv]{Dept. of Radiology, University
Hospital Center (CHUV) and University of Lausanne (UNIL), Lausanne, Switzerland}

\begin{abstract}
% -*- root: 00-main.tex -*-
Current methods for processing \gls*{dmri} to map the connectivity of the human brain
  require precise delineations of anatomical structures.
\revcomment[C1 (R.1)]{%
The requirement has been approached either segmenting the data in native \gls*{dmri} space
  or projecting the structural information from \gls*{t1} images.
The characteristic features of diffusion data in terms of \acrlong*{snr}, resolution, etc.
  and the geometrical distortions caused by the inhomogeneity of magnetic susceptibility
  across tissues render the problem challenging.}
\revcomment[C10 (R.2)]{%
To solve the problem in a unified approach, we propose \regseg{}, a surface-to-volume nonlinear
  registration method that segments homogeneous regions within multivariate images by mapping
  a set of nested reference-surfaces.}
\revcomment[C11 (R.2)]{%
The surfaces can be extracted from the \gls*{t1} image of the subject, and the target image
  is the bivariate volume comprehending the \gls*{fa} and the \gls*{adc} maps derived from the
  \gls*{dmri} dataset.}
We first verify the subpixel accuracy of \regseg{} on digital phantoms, to then propose an
  evaluation framework that uses \gls*{dmri} datasets from the \gls*{hcp}.
In this framework, the undistorted \gls*{dmri} undergo a known distortion derived from real
  fieldmaps.
We analyze the misregistration error of the surfaces estimated by \regseg{} on 16 datasets.
The distribution of errors shows a 95\% CI of 0.56--0.66 mm, below the \gls*{dmri} resolution
  (1.25 mm, isotropic).
Finally, we cross-compare the proposed tool against a nonlinear \lowb{}-to-T2w registration
  method, thereby obtaining a significantly lower misregistration error with \regseg{}.
\end{abstract}

% Note that keywords are not normally used for peerreview papers.
\begin{keyword}
diffusion MRI \sep cortical~parcellation \sep segmentation %
\sep nonlinear~registration \sep active~contours \sep susceptibility~distortion.
\end{keyword}

\end{frontmatter}

\linenumbers
% -*- root: 00-main.tex -*-
\section{Introduction}\label{sec:regseg-intro}
\Acrlong*{dmri} enables the mapping of microstructure \citep{basser_microstructural_1996}
  and connectivity \citep{craddock_imaging_2013} of the human brain \emph{in-vivo}.
It is generally acquired using \gls*{epi} schemes, since they are very fast at
  scanning a large sequence of images called \glspl*{dwi}.
Each \gls*{dwi} is sensitized with a gradient to probe proton diffusion in a certain
  orientation.
Subsequent processing involves describing the local microstructure with one of the available
  models, which range from the early \gls*{dti} proposed by \cite{basser_microstructural_1996}
  to current models such as \gls*{amico}.
The microstructural map is then used to draw the preferential orientations of diffusion
  across the brain using tractography \citep{mori_threedimensional_1999}.
Finally, a graph representing the corresponding structural network is built using
  the regions of a cortical parcellation as nodes and the fiber paths found by
  tractography as edges \citep{hagmann_mapping_2008}.
The methodologies to solve reconstruction, tractography and network building
  require the delineation of the anatomy in the \gls*{dmri} space.
Moreover, current trends on reconstruction \citep{jeurissen_multitissue_2014} and
  tractography \citep{smith_anatomicallyconstrained_2012} are increasingly using
  structural information to improve the microstructural mapping and fiber-tracking.

Possibly, the earliest structural information incorporated to aid \gls*{dmri} processing
  is the \gls*{wm} mask used as a termination criteria for tractography.
The standardized procedure to obtain this mask was thresholding the \gls*{fa} map.
However, the mask and subsequent analyses are highly dependent on the 
  threshold that is chosen \citep{taoka_fractional_2009}.
To overcome the unreliability of \gls*{fa} thresholding, and to broaden
  \gls*{wm} segmentation to brain tissue segmentation, a large number of
  methods have been proposed using \glspl*{dwi}, the \lowb{}, and \gls*{dti}-derived
  scalar maps such as \gls*{fa}, \gls*{adc} and others \citep{zhukov_level_2003,
  rousson_level_2004,jonasson_segmentation_2005,hadjiprocopis_unbiased_2005,liu_brain_2007,
  lu_segmentation_2008,han_experimental_2009}.
However, the precise segmentation of \gls*{dmri} is difficult for several reasons.
First, \gls{dmri} images have a resolution that is much lower
  than that of the imaged microstructural features.
Therefore, voxels located in structural discontinuities are affected by partial
  voluming of the signal sources.
Second, the extremely low \gls*{snr} and the high dimensionality of the \glspl*{dwi} prevent
  their direct use in segmentation.
Third, the low contrast between \gls*{gm} and \gls*{wm} in the \lowb{} volume also makes
  it unsuitable for brain tissue segmentation.

An alternative route to segmentation in \gls*{dmri} space is the mapping of the
  structural information extracted from anatomical MR images, such as \gls*{t1}, using
  image registration techniques.
Generally, intra-subject registration of MR images of the brain involves only a linear
  mapping to compensate for head motion between scans.
However, \gls*{epi} introduces a geometrical distortion \citep{jezzard_correction_1995}
  that impedes the linear mapping from the structural space.
Numerous methods have been proposed to overcome this problem by incorporating information
  from extra MR acquisitions such as fieldmaps \citep{jezzard_correction_1995},
  \glspl*{dwi} with a different \gls*{pe} scheme \citep{cordes_geometric_2000,chiou_simple_2000},
  or \gls*{t2} images \citep{kybic_unwarping_2000,studholme_accurate_2000}.
These methods estimate the deformation field associated to \emph{\gls*{epi} distortions} and
  resample the \glspl*{dwi} onto a corrected \gls*{dmri} space.
The retrospective \gls*{epi} correction is an active field of research yielding frequent refinements
  and combinations of the original methods, such as
  \citep{holland_efficient_2010,andersson_comprehensive_2012,
  irfanoglu_drbuddi_2015}.
A standardized method to solve the remaining linear mapping between the corrected-\gls*{dmri}
  and the structural spaces is \emph{bbregister} \citep{greve_accurate_2009}.

Here, we present a segmentation and surface-to-volume registration method
  called \regseg{}, and show its usefulness in mapping anatomical information from structural
  space into native \gls*{dmri} space to aid subsequent processing steps
  (reconstruction, tractography and network building using a cortical parcellation).
The underlying hypothesis is that the registration and segmentation problems
  in \gls*{dmri} can be solved simultaneously.
To implement \regseg{} we first establish an active-contours without edges
  \citep{chan_active_2001} segmentation framework.
A specific set of reference surfaces extracted from the same subject initialize
  the 3D active contours, which evolve searching for homogeneous regions in the multivariate
  target-image.
We apply \regseg{} to segment \gls*{dmri} data by mapping
  a set of nested surfaces extracted from a structural image (e.g. \gls*{t1}) to
  a bivariate target-volume comprehending the \gls*{fa} and \gls*{adc} maps.
The evolution of the surfaces is supported by a B-spline basis, optimized
  iteratively using a descent approach driven by shape-gradients
  \citep{besson_dream2s_2003,herbulot_segmentation_2006}.
Therefore, \regseg{} establishes a registration framework that actually
  deals with the nonlinear warping induced by \gls*{epi} distortions.
\Regseg{} integrates the benefits of segmentation and registration methods together and
  exploits the multivariate nature of \gls*{dmri} data to contribute in the proposed
  application on three key aspects:
  1) the surfaces are typically extracted from the \gls*{t1} image of the same subject,
    therefore \regseg{} does not require additional MR acquisitions to the minimal
    \gls*{dmri} protocol in order to estimate the deformation field;
  2) \revcomment[R\#1-C2]{%
    alternatively to the typical design of the processing flow, the information from the
    reference \gls*{t1} can precisely be mapped onto the distorted \gls*{dmri} space,
    avoiding the interpolation of the \glspl*{dwi} required by unwarping the diffusion data; and} 
%    depending on the application, the resampling of the \glspl*{dwi}
%    introduced by the correction method can be avoided, either by performing the posterior
%    processing on the native \gls*{dmri} space or applying the unwarping on the tractography
%    itself; and
  3) \regseg{} increases the geometrical accuracy of the overall process.
In this paper, we first verify the functionality of the method and the \regseg{}
  implementation using a set of digital phantoms, demonstrating the
  \revcomment[R\#3-C10]{subvoxel} accuracy in registration.
Then, we evaluate \regseg{} on real \gls*{dmri} datasets, using a derivation of our
  instrumentation framework \citep{esteban_simulationbased_2014} which simulates
  known and realistic \gls*{epi} distortions.
We also compare \regseg{} and a nonlinear registration method to map the \lowb{} to
  the corresponding \gls*{t2} image of the same subject.
This approach is the first step of the \gls*{t2b} correction
  methods \citep{kybic_unwarping_2000}.
We reproduce the settings and implementation of a widely used diffusion
  processing software \citep[\emph{ExploreDTI},][]{leemans_exploredti_2009}.
With this comparison, we demonstrate how \regseg{} achieves higher accuracy with the
  simultaneous registration and segmentation process.
\section{Methods}
\label{sec:methods}
%
\subsection{Related work}
\label{sec:methods_background}

\Gls{acwe} and level-set based formulations have been widely applied in image
processing, as reviewed in \citep{suri_shape_2002}. We suggest clustering the 
current methodologies of template-based segmentation methods into three groups. 
The first group typically adds a shape prior term to the energy functional of 
an evolving active contour \citep{bresson_variational_2006,
chan_level_2005,chen_using_2002,cremers_kernel_2006,gastaud_combining_2004}.
These methods generally have an explicit description of the expected relative boundary 
locations of the object to be delineated, and some even model the statistical deviations
from this average shape. By including a coordinates mapping, it is possible to perform
active contours based registration between timesteps in a time-series or between different
images \citep{bertalmio_morphing_2000,wyatt_map_2003,paragios_level_2003,vemuri_joint_2003,
yezzi_variational_2003}.
A summary of these second set of techniques is performed in \citep{droske_mumfordshah_2009},
proposing two different approaches to applying the Mumford-Shah \citep{mumford_optimal_1989}
functional in joint registration and segmentation. Finally, a derivation of the 
latter groups is composed by atlas-based segmentation methods 
\citep{gorthi_segmentation_2009,gorthi_active_2011,pohl_unifying_2005,
pohl_bayesian_2006,wang_joint_2006}, where the prior 
imposes consistent voxel-based classification of contiguous regions.
A comprehensive summary of the \gls{acwe}-derived methodologies with special attention
to joint registration-segmentation methods is found in \citep{gorthi_active_2011}.

Our joint registration and segmentation method is mainly inspired in the active
deformation fields presented by \citep{gorthi_active_2011}, that is a generalization
of the methodologies under second and third groups before. As it is described in
their work, the first joint ``morphing''-segmentation approach (it was not a proper
registration method) was proposed in \citep{bertalmio_morphing_2000}. The first
registration framework is presented by \citep{yezzi_variational_2001} where the
energy functional is defined simultaneously in the moving and target images with
an affine transformation supporting the coordinates mapping (``2 PDEs approach'').
\citep{vemuri_joint_2003} proposed the first atlas-based registration based on the 
level set framework and using only one PDE. \citep{unal_coupled_2005} extended the 
idea of \citep{bertalmio_morphing_2000,yezzi_variational_2001} for non-linear 
registration with a dense deformation field as mapping function. \citep{droske_mumfordshah_2009}
combining ideas from both branches and proposed the propagation of the 
deformation field from the contours to the whole image definition. Finally,
\citep{gorthi_active_2011} present a comprehensive generalization of the methodologies.

All the surveyed frameworks are deduced from the general level set evolution equation
introduced by \citep{osher_fronts_1988}:
\begin{equation}
\frac{\partial \Phi_D(\mathbf{x},t)}{\partial t} = \nu ( \Phi_D(\mathbf{x},t)) \, \left| \nabla \Phi_D(\mathbf{x},t) \right|
\end{equation}
where $\nu$ is the velocity of the flow or speed function that contains the local
segmentation and contour regularization constraints, and $\Phi_D: \Omega \to \mathbb{R}$
is the signed distance function often used to represent implicitly the active contour
by its zero level. An early extension to image registration was proposed by 
\citep{vemuri_joint_2003} but replacing $\Phi_D$ by the intensity function of the
image to register, $\Phi_I: \Omega \to \mathbb{R}$ (the moving image). 
\citep{bertalmio_morphing_2000,vemuri_joint_2003} then define the speed function as
$\nu(\Phi_I(\mathbf{x},t)) = \Phi_I(\mathbf{x},t) - \Phi_T(\mathbf{x},t)$, where $\Phi_T(\mathbf{x})$ is the intensity
function of the target image. Finally, \citep{gorthi_active_2011} propose a different
multiphase label function as level set function ($\Phi_L$) that improves registration
results.

Naming $\Phi_G$ a generic objective function, a dense deformation field of vectors 
$u: \mathbb{R}^n \to \mathbb{R}^n$ (tipically $n = \{ 2, 3 \}$) is introduced. 
Then, the conservation of the morphological description is assumed, such as
$\Phi_G(\mathbf{x},t) = \Phi_G( \mathbf{x} + du, t + dt ) \implies d\Phi_G(\mathbf{x},t) = 0$, where $d\Phi_G$ is
the total derivative of $\Phi_G$. Using the chain rule, it can be rewritten as the 
evolution equation of a vector flow:
\begin{equation}
\frac{\partial u(\mathbf{x},t)}{\partial t}= - \frac{\Phi_G}{\left| \nabla \Phi_G \right|} N_{\Phi_G},
\end{equation}
where $N_{\Phi_G}$ is the normal of the level set. Finally, they introduce the general
level set evolution equation into the vector flow to perform the registration task:
\begin{equation}
\frac{\partial u(\mathbf{x},t)}{\partial t} = - \nu( \Phi_G(\mathbf{x},t) ) N_{\Phi_G}.
\end{equation}
Therefore, the position of the level set function $\Phi_G$ at time $t$ is given by the
deformation field $u(\mathbf{x},t)$ and the initial level set function $\Phi_G(\mathbf{x},0)$, yielding:
\begin{equation}
\frac{\partial u(\mathbf{x},t)}{\partial t} = - \nu( \Phi_G(\mathbf{x} + u(\mathbf{x},t), 0) ) N_{\Phi_G}.
\end{equation}

Here, we propose an active deformation field framework with some differences in
the formulation of the problem. The main diverging points with respect to
\citep{gorthi_active_2011} are: 1) there is no need for an explicit level set function
$\Phi_G$, as we replace the level set gradient computation $N_{\Phi_G}$ with shape
gradients \citep{jehan-besson_dream2s:_2003,herbulot_segmentation_2006}; 2) 
\todo[inline]{regularization is also based on linear diffusion smoothing REF-thirion,
but they use a Gaussian filtering)}; 3) optimization is applied in the spectral
domain, observing anisotropic and inhomogeneous mappings along each direction.


\subsection{Bridging segmentation frameworks}
\label{sec:methods_map}
%
A widely used approach to image segmentation is derived from the
Bayes' rule \eqref{eq:bayes_rule}. This framework seeks for a 
partitioning of a certain observation of features $F: \mathbb{R}^n \to %
\mathbb{R}^C$ (an $n$-dimensional \emph{image} comprising $C$ different 
channels) in piecewise smooth regions $\Omega = \lbrace \Omega_k , 
k\in\left[ 1 \ldots K \right] \rbrace$,  that maximize the \emph{a posteriori}
probability $p(\Omega \mid F)$:
\begin{equation}
p(\Omega \mid F) = \frac{p(F \mid \Omega)\, p(\Omega)}{p(F)},
\label{eq:bayes_rule}
\end{equation}
where $p(F \mid \Omega)$ is the \emph{likelihood} of the realization 
of $F$ given a certain region set $\Omega$. The second term, $p(\Omega)$,
is the a-priori probability of the partitioning. Finally, $p(F)$ is the 
probability of a certain image realization, and thus, it will remain 
constant when computing the \gls{map}. Consequently, the Bayes' rule
can be simplified for the optimization as follows:
\begin{equation}
p(\Omega \mid F) \propto p(F \mid \Omega)\, p(\Omega).
\label{eq:bayes_rule_simplified}
\end{equation}

In the simplest Bayesian segmentation framework, the unknown is the
parameter set $\Theta_k$ of region-wise descriptors that minimize 
\eqref{eq:bayes_rule}. Conversely, we can assume these descriptors as
fixed (fulfilling the concept of conservation of morphological description
mentioned before) and introduce a realization of a dense deformation field $U: %
\mathbb{R}^n \to \mathbb{R}^n$ that maps the initial partition $\Omega_0$
to the estimated one:
\begin{equation}
\Omega' = \Omega_0 + \hat{U}.
\label{eq:omega_tf}
\end{equation}

Thus, \eqref{eq:bayes_rule_simplified} is rewritten as 
$p(U \mid F) \propto p(F \mid U)\, p(U)$ that yields the
following \gls{map} solution:

\begin{equation}
\hat{U} = \underset{U}{\argmax} \left\{ p(U \mid F) \right\} = 
\underset{U}{\argmax} \left\{ p(F \mid U)\, p(U) \right\}.
\label{eq:map_u}
\end{equation}


An extended assumption is that the feature vector realization $F$ is
\emph{i.i.d.}, and thus, it is possible to write the a-posteriori
probability $p(F \mid U)$ as a continuous product with $d\mathbf{x}$ the
infinitesimal voxel size:
\begin{equation}
p(F \mid U) \, p(U) = \underset{k}{\prod} \underset{\mathbf{x}\in \Omega'_k}{\prod}
p_k\left( F(\mathbf{x} + U(\mathbf{x})) \mid U(\mathbf{x}) \right)^{d\mathbf{x}}.
\label{eq:bayes_aposteriori}
\end{equation}
\todo[inline]{I'm not convinced about turning $p(\Omega)$ into $p(U)$.}

A second widely-accepted assumption is the multivariate normal 
distribution of the different tissues in \gls{mri} data. Therefore,
the posterior probability of an infinitesimal voxel can be written as:
\begin{equation}
p_k( F(\mathbf{x}) \mid U(\mathbf{x}) ) = \frac{1}{ \sqrt{(2\pi)^{C}\,\left|\boldsymbol{\Sigma}_{k}\right|}}\,{e^{\left(-\frac{1}{2}  \Delta^2_k (\mathbf{f}') \right)}}.
\label{eq:bayes_mpdf}
\end{equation}
where we can identify the factor in the exponential as the squared \emph{Mahalanobis 
distance} of the feature observed in the displaced position, $\mathbf{f}' = 
F(\mathbf{x} + U(\mathbf{x}))$ with region descriptors
$\Theta_k = \lbrace \boldsymbol{\mu}_k, \boldsymbol{\Sigma}_k \rbrace$:
\begin{equation}
\Delta^2_k (\mathbf{f} \mid \Theta_k ) = (\mathbf{f} - \boldsymbol{\mu}_k)^T \, \boldsymbol{\Sigma}^{-1}_k \, (\mathbf{f} - \boldsymbol{\mu}_k).
\label{eq:bayes_mahalanobis}
\end{equation}

Finally, we can turn the \gls{map} problem into a variational one
applying the following log-transform:
\begin{multline}
E(F \mid U)= -\log \left[ p(F \mid U) \, p(U) \right] = \\
= -\log \left[ \underset{k}{\prod} \underset{\mathbf{x}\in \Omega'_k}{\prod}
p_k( F(\mathbf{x}) \mid U(\mathbf{x}) )^{d\mathbf{x}} \right] = \\
= \sum\limits_k \int_{\Omega'_k} -\log \left[ p_k(F(\mathbf{x}) \mid U(\mathbf{x} ) ) \right] \, d\mathbf{x},
\label{eq:energy_1}
\end{multline}
and introducing the posterior probability term \eqref{eq:bayes_mpdf}, 
we can express the functional in terms of the deformation field $U$ and
the region descriptors $\Theta_k$:
\begin{multline}
E(U) = \\
= \sum\limits_k \int_{\Omega'_k} -\log \left[ \frac{1}{ \sqrt{(2\pi)^{C}\,\left|\boldsymbol{\Sigma}_{k}\right|}}\,{e^{\left(-\frac{1}{2}  \Delta^2_k (\mathbf{f}) \right)}} \right] \, d\mathbf{x} = \\
= \sum\limits_k \int_{\Omega'_k} \left[ \frac{1}{2} \log{ \left( (2\pi)^{C}\,\left|\boldsymbol{\Sigma}_{k}\right| \right)} + \frac{1}{2}  \Delta^2_k (\mathbf{f}) \right] \,d\mathbf{x}.
\end{multline}
Finally, after removing scaling factors and independent constants,
we obtain:
\begin{align}
E(U) = \sum\limits_k \left[ \left|\Omega'_k\right|\,\log \left|\mathbf{\Sigma}_k \right| + \int_{\Omega'_k} \Delta^2_k (\mathbf{f}) \,d\mathbf{x} \right],
\label{eq:map_energy}
\end{align}
where we have a constant term scaled by the total volume of the partition $\Omega_k$ and 
the determinant of the  covariance matrix of the partition $\left|\boldsymbol{\Sigma}_{k}\right|$,
plus an energy term based on the squared \emph{Mahalanobis distance}.
As long as this shape-based registration framework is designed to allow small changes
between boundaries (and therefore, their volumes), and the shape descriptors should not
change significantly due to the conservation principle, the first term can be optionally
dismissed with respect the second one, what yields:
\begin{align}
E(U) = \sum\limits_k \int_{\Omega'_k} \Delta^2_k (\mathbf{f}) \,d\mathbf{x}.
\label{eq:final_map_energy}
\end{align}

This latter expression resembles the Mumford-Shah functional 
\citep{mumford_optimal_1989} which is at the background of all \gls{acwe}-based
methods. In this case, \eqref{eq:final_map_energy} includes covariance as
region descriptor, what modifies the original functional in a way that it 
can deal with more general distributions. This is necessary to avoid the 
assumption that regions $\Omega_k$ have a fixed covariance matrix on their 
complete domain. One immediate advantage of this functional from the original 
one is the possibility to distinguish regions with the same mean vector but 
different covariance matrix \citep{brox_local_2009}. In this work, the 
Mumford-Shah functional is derived for this extension as follows:
\begin{multline}
E(\Theta_k,\Omega_k) = \sum\limits_k \int_{\Omega_k} \left[ \log \left|\mathbf{\Sigma}_k\right| + \Delta^2_k (\mathbf{f}) \right] \,d\mathbf{x} \\
+ \lambda \int_{\Omega_k - K}  ( \left| \nabla \mathbf{\mu} \right| ^2 + \left| \nabla \mathbf{\Sigma}_k \right| ^2 ) \, d\mathbf{x} 
+ \nu |K|,
\end{multline}
that is easily identifiable with \eqref{eq:map_energy} when we apply 
the so-called \emph{cartoon limit}, 
for $\lambda \to \infty$:
\begin{equation}
E(\Theta_k,K) = \sum\limits_k \int_{\Omega_k} \left[ \log \left|\mathbf{\Sigma}_k\right| + \Delta^2_k (\mathbf{f}) \right] \,d\mathbf{x}
+ \nu |K|.
\end{equation}

As long as we do not penalize the edge set $K$ length, $\nu = 0$ and
the result is dual to \eqref{eq:map_energy}:
\begin{equation}
E(\Theta_k,K) = \sum\limits_k \int_{\Omega_k} \left[ \log \left|\mathbf{\Sigma}_k\right| + \Delta^2_k (\mathbf{f}) \right] \,d\mathbf{x}.
\end{equation}


\subsection{Introducing priors}
\label{sec:priors}
%
By introducing $U$ in \eqref{eq:bayes_rule_simplified}, we transformed 
the segmentation problem into a registration one provided with $\Omega_0$ 
(the initial partition) is derived from the shape-priors given in reference 
space. Therefore, the priors can be expressed in terms of the deformation
field:
\begin{equation}
P(U) = \underset{\mathbf{x}}{\prod}\,p(U(\mathbf{x}))\,^{d\mathbf{x}}.
\label{eq:prior_u}
\end{equation}
We define $P(U(\mathbf{x})) = p(\mathbf{u}) = \underset{d}\prod \, p_d(\mathbf{u})$ 
with $d$ the order of the derivative. We also assume that an initial
affine registration let us consider that the deformation field ($d=0$)
and its derivatives up to some order ($d>0$) are all zero mean, and
have some variance $\Sigma_d$. We restrict ourselves to $d=\lbrace 0,1 \rbrace$
(the field and its gradient):
\begin{subequations}
\begin{equation}
p_0(\mathbf{u}) = \mathcal{N}( \mathbf{u} \mid 0, \alpha^{-1}),
\end{equation}
\begin{equation}
p_1(\mathbf{u}) = \mathcal{N}( \nabla \mathbf{u} \mid 0, {\Sigma_\beta}^{-1}), 
\end{equation}
\end{subequations}
And therefore:
\begin{equation}
P(U) = \underset{\mathbf{x}}\prod \left[ p_0(\mathbf{u}) \, p_1(\mathbf{u}) \right]^{d\mathbf{x}}.
\end{equation}




\subsection{Numerical Implementation}
%
Let us denote $\{c_i\}_{i=1 \ldots N_c}$ the nodes of one or several shape-prior
surface(s). In our application, precise tissue interfaces of interest 
extracted from a high-resolution, anatomically correct reference volume. 
On the other hand, we have a number of \gls{dwi}-derived features at each
voxel of the volume. Let us denote by $\mathbf{x}$ the voxel and $f(\mathbf{x}) = 
[ f_1, f_2, \ldots, f_N]^T(\mathbf{x})$ its associated feature vector.

\subsubsection{Deformation field}
\label{sec:deformation_field}
The transformation from reference into \gls{dwi} coordinate space is 
achieved through a dense deformation field $u(\mathbf{x})$, such that:
\begin{equation}
c_i' = T\{c_i\} = c_i + u(c_i),
\end{equation}
what is equivalent to \eqref{eq:omega_tf}. Since the nodes of the anatomical 
surfaces might lay off-grid, it is required to derive $u(\mathbf{x})$ from a discrete 
set of parameters $\{u_k\}_{k=1 \ldots K}$. Densification is achieved through 
a set of associated basis functions $\psi_k$ (e.g. rbf, interpolation splines):
%
\begin{equation}
u(\mathbf{x}) = \sum_k \psi_k(\mathbf{x}) u_k
\end{equation}
%
In our implementation, $\psi_k$ is chosen to be a tensor-product B-Spline kernel.
Then, the transformation writes
%
\begin{equation}
\label{eq:transformation}
c_i' = T\{c_i\} = c_i + u(c_i) = c_i + \sum_k \psi_k(c_i)u_k
\end{equation} 

Based on the current estimate of the distortion $u$, we can compute 
``expected samples'' within the shape prior projected into the \gls{dwi}.
Thus, we now estimate region descriptors of the \gls{dwi} features 
$f(\mathbf{x})$ of the regions defined by the priors in \gls{dwi} space.
%
Using the region descriptors derived in \autoref{sec:methods_map}, we propose
an \gls{acwe}-like, piece-wise constant, variational image segmentation
model (where the unknown is the deformation field)
\cite{chan_active_2001} with the energy functional obtained in 
\eqref{eq:final_map_energy}. This inverse problem is ill-posed
\cite{bertero_ill-posed_1988,hadamard_sur_1902}.
In order to account for deformation field regularity and to render the 
problem well-posed, we include limiting and regularization terms into 
the energy functional \cite{morozov_linear_1975,tichonov_solution_1963}:
%
\begin{multline}
E(u) = \sum\limits_k \int_{\Omega'_k} \Delta^2_k (\mathbf{f}) \,d\mathbf{x} \\
+ \int u^T \, A \, u \, d\mathbf{x} + \int \tr\{(\nabla u^T)^T B (\nabla u^T)\} d\mathbf{x}
\label{eq:complete_energy}
\end{multline}
%
These regularity terms ensure that the segmenting contours in 
\gls{dwi} space are still close to their native shape. The model
easily allows to incorporate inhomogeneous and anisotropic 
regularization \cite{nagel_investigation_1986} to better regularize
the \gls{epi} distortion. \\
%

\subsubsection{Operator splitting}
In order to make the energy minimization computationally more tractable, 
we propose the following operator splitting: Let us optimize the data terms 
and the regularity terms on separate copies of the deformation field, 
now called $u$ and $v$, constrained to be equal:
\begin{multline}
E(u,v) = \sum\limits_k \int_{\Omega'_k(u)} \Delta^2_k (\mathbf{f}) \,d\mathbf{x} \\
+ \int v^T \, A \, v \, d\mathbf{x} + \int \tr\{(\nabla v^T)^T B (\nabla v^T)\} d\mathbf{x}
\label{eq:operator_splitting}
\end{multline}
and now
\begin{equation}
\min_{u,v} \{ E \} \quad s.t. \quad u = v.
\end{equation}

In order to take the equality constraint into account, we may make use of 
augmented Lagrangians (a combination of Lagrangian multipliers and penalty 
terms on the constraint) \cite{bertsekas_multiplier_1976,
glowinski_augmented_1989,nocedal_numerical_2006}:
\begin{equation}
AL(u,v,\lambda,r) = E(u,v) + \langle \lambda, u-v \rangle + \frac{r}{2} \| u - v \|_2^2
\end{equation}

To solve the constraint minimization problem, we may now optimize the 
Augmented Lagrangian in an iterative way:
\begin{equation}
\left\lbrace 
\begin{array}{rcl}
u^{t+1} &=& \argmin_{u} AL(u,v^t,\lambda^t,r)\\
v^{t+1} &=& \argmin_{v} AL(u^{t+1},v,\lambda^t,r)\\
\lambda^{t+1} &=& \lambda^t + \rho(u^{t+1}-v^{t+1})
\end{array}\right.
\end{equation}
where typically $0 < \rho < r$. The two sub-minimization problems will 
now be much easier to handle than the original complete problem.

\subsubsection{Shape gradients}
To compute the gradient-descent of the data-term domain integrals with 
respect to the underlying deformation field, we use shape gradients 
\cite{jehan-besson_dream2s:_2003,herbulot_segmentation_2006}. A little 
bit of theory is therefore in order.

Let $\Omega$ be an image domain and $d\Omega$ its boundary. Further, $r(\mathbf{x})$ 
is an ``arbitrary'' function over the image domain. We now derive the domain 
integral w.r.t. the contour evolution parameter $t$ (time):
\begin{equation}
\frac{\partial}{\partial t} \int_\Omega r(\mathbf{x}) d\mathbf{x} = \int_\Omega \frac{\partial r}{\partial t}(\mathbf{x}) d\mathbf{x} - \int_{d\Omega} r(\mathbf{x}) \left\langle \frac{\partial{d\Omega}}{\partial t}, N_{d\Omega}\right\rangle d\mathbf{x}
\label{eq:shape_gradients}
\end{equation}
where $\left\langle\frac{\partial{d\Omega}}{\partial t}, N_{d\Omega}\right\rangle$ is 
the projection of the boundary movement on the unit inward normal. We recall
here that equation \eqref{eq:shape_gradients} stands the bridge between the 
explicit level set formulations surveyed in \autoref{sec:methods_background} 
with the presented approach in setting up the velocity function.


\subsubsection{Minimization w.r.t. $u$}
The first minimization problem optimizes the data-term. The problem is equivalent 
to minimizing the following energy:
\begin{multline}
E(u,v) = \sum\limits_k \int_{\Omega'_k(u)} \Delta^2_k (\mathbf{f}) \,d\mathbf{x}
+ \langle \lambda, u-v \rangle + \frac{r}{2} \| u - v \|_2^2
\end{multline}
where we identify one instance of domain integrals of the form $\int_\Omega r(\mathbf{x}) 
d\mathbf{x}$ for each region $k$. Optimality requires the derivative of this energy 
with respect to the parameters $u$ to be null. At this point, we decide to ignore 
the influence of the boundary shift on the statistics of the regions (i.e. moving the 
boundary does not significantly impact the $\mu$ and $\Sigma$ descriptors). This means 
that we can drop the derivative of $r(\mathbf{x})$ w.r.t. contour evolution. 
What remains are the surface integrals at the subdomain interfaces, plus the Lagrangian 
and penalty terms:

\begin{multline}
\frac{\partial E}{\partial u_k} = \sum\limits_k \int_{d\Omega'_k} \left[ \Delta^2_{O(k)} (\mathbf{f}) - \Delta^2_k (\mathbf{f}) \right]
\left\langle\frac{\partial{d\Omega'_k}}{\partial t}, N_{d\Omega'_k}\right\rangle \,d\mathbf{x} \\
+ \lambda_k + r(u_k - v_k) = 0
\label{eq:energy_gradient}
\end{multline}
where $O(k)$ is a function that returns the neighboring region at $\mathbf{x}$ (i.e.
the region \emph{outside} the current location in the contour). For the shake of
simple notation, we will refer as $c'$ to the positions of points belonging to
any of the existing contours $d\Omega_k$.
Given the deformation field interpolation stated in \autoref{sec:deformation_field}, 
the boundary moves according to
\begin{equation}
\frac{\partial c'}{\partial u_k} = \psi_k(c')\,e_a
\end{equation}
where $e_a$ is the unit vector along direction $a \in \mathbb{R}^n$ and $n$ the dimension
of the image. Thus
\begin{equation}
\left\langle\frac{\partial c'}{\partial u_k}, N_{c'}\right\rangle = \psi_k(c')\,\hat{\mathbf{n}}_{c'}
\end{equation}
The discrete implementation of \eqref{eq:energy_gradient} is straightforward as
we have a triangularized mesh representation of the interfacing surfaces:
\begin{multline}
\frac{\partial E}{\partial u_k} = \sum\limits_k \sum\limits_{c'} \left[ \Delta^2_{O(k)} (\mathbf{f}) - \Delta^2_k (\mathbf{f}) \right]
\, \psi_k(c')\,\hat{\mathbf{n}}_{c'} \\
+ \lambda_k + r(u_k - v_k) = 0
\label{eq:discrete_energy_gradient}
\end{multline}
The optimal distortion $u_k$ is found at each iteration as:
\begin{multline}
u_k^{t+1} = v_k^t - \frac{1}{r}\lambda_k^{t} \\
- \frac{1}{r} \sum\limits_k \sum\limits_{c'} \left[ \Delta^2_{O(k)} (\mathbf{f}) - \Delta^2_k (\mathbf{f}) \right]
\, \psi_k(c')\,\hat{\mathbf{n}}_{c'}
\label{eq:update_u}
\end{multline}


\subsubsection{Minimization w.r.t. $v$}
For the optimization w.r.t. $v$, the relevant energy writes
\begin{align}
E(v) &= \int v^T A v d\mathbf{x} + \int \tr\{(\nabla v^T)^T B (\nabla v^T)\} d\mathbf{x}\\
&\quad + \langle \lambda, u-v \rangle + \frac{r}{2} \| u - v \|_2^2\nonumber
\end{align}
\todo[inline]{Do not assume this}
Let's assume the simplest, homogeneous isotropic case, $A = \alpha/2$ 
and $B = \beta/2$. The associated Euler-Lagrange equation is found as:
\begin{equation}
\alpha v - \beta\Delta v + rv = ru + \lambda
\end{equation}
which easily translates into Fourier domain:
\begin{equation}
v^{t+1} = \mathcal{F}^{-1}\left\{ \frac{\mathcal{F}\{ru + \lambda\}}{\mathcal{F}\{(\alpha+r)\mathcal{I}-\beta\Delta\}} \right\}
\end{equation}
where $\mathcal{I}$ denotes the identity operator.

Here, we rewrite the Laplacian as a linear combination of the identity and shift operators:
\begin{equation}
\Delta = \sum\limits_d \mathcal{S}_d^- + \mathcal{S}_d^+ - 2 \mathcal{I}
\end{equation}
where $\mathcal{S}_{d}^{\pm}$ stands for the forward ($+$) and backward ($-$) shift 
operator along coordinates axis $d$, of which the Fourier transform is found easily as
\begin{equation}
\mathcal{F}\{\mathcal{S}_{d}^{\pm}\} = e^{\pm i\omega_{d}},
\end{equation}
where $\omega_{d}$ is the normalized pulsation along $d$-direction. \todo{stop!} Accordingly, the 
Fourier transform of the discrete Laplacian is found as
\begin{align}
\mathcal{F}\{\Delta\} &= e^{-i\omega_x } + e^{i\omega_x } + e^{-i\omega_y } + e^{i\omega_y } - 4\nonumber\\
&= 2\left( \cos(\omega_x) + \cos(\omega_y) - 2 \right)
\end{align}

The remaining transforms are trivial or can be computed using FFT (as in \citep{estellers_efficient_2011}).


\subsubsection{Lagrangian multiplier update}
At each iteration, the Lagrangian multipliers are updated as noted before:
\begin{equation}
\lambda^{t+1} = \lambda^t + \rho(u^{t+1}-v^{t+1})
\end{equation}

\subsubsection{Region descriptor reestimation}
In regular intervals, i.e. after $n$ iterations, the parameters $\mu$ and $\Sigma$ of the involved regions need to be reestimated based on the shifted volumetric samples $w_j'$, $g_j'$ and $o_j'$.

\subsubsection{Convergence}
In order to ``fixate'' the evolution when close to convergence, it is advised to slightly increase the penalty weight $r$ at each iteration. Note that as can be seen in the above equations, $r$ governs the step-size or  leash-length at each iteration, i.e. the amount by which the new estimate $u$ may move away from the preceding $v$ and vice-versa. 

% -*- root: 00-main.tex -*-
\section{Results}
\label{sec:results}

\subsection{Proof of concept on digital phantoms}
\label{sec:results_phantom}

\begin{figure*}
  \centering
  \includestandalone[width=\linewidth]{figure05}
  \caption{A. Visual assessment of the results on the low resolution sets:
    ``gyrus'' (top-left), ``L'' (top-right), ``ball'' (bottom-left),
    and ``box'' at (bottom-right).
  In yellow color, the recovered contours after registration are represented.
  Our method showed high accuracy, as it demonstrates the almost exact location of the registered
    contours with respect to their ground truth position depicted in green.
  Partial volume effect turns segmentation of the sulci a challenging problem with voxel-wise
    clustering methods, but it is successfully segmented with our method.
  B. Quantitative evaluation of registration error in terms of average Hausdorff distance of
    surfaces at low (left) and high (right) resolutions, demonstrating that the error is
    consistently below the voxel size.
    }\label{fig:phantom}
\end{figure*}
A total of 1200 experiments (4 phantom types $\times$ 150 random warpings $\times$ 2 resolutions) were
  run using the workflow given in \autoref{fig:evphantoms}.
For each experiment, the misregistration error was measured using the Hausdorff distance
  (see \autoref{sec:experiments_evaluation}) between the theoretical $\gammaset_\text{true}$ and
  the estimation done by \regseg{} ($\hat{\gammaset}_{test}$).
The results showed a consistent and high accuracy, below the image resolution.
\autoref{fig:phantom} (block C) presents the violin plots by model type corresponding
  to the two sets of resolutions of generated phantoms.
In order to relate average misregistration error to the resolution of the moving image,
  we proceeded as follows.
First, we confirmed that the vertex-wise error distributions were skewed using a Shapiro-Wilk test of
  normality.
All the distributions of errors under test (4 phantom types $\times$ 2 resolutions) resulted
  non-normal with $p<0.001$.
Consequently, we used the non-parametric Wilcoxon signed-rank test along with Bonferroni
  correction for multiple comparisons ($N=150$, for each phantom type).
Averaged errors resulted significantly lower than voxel size with $p < (0.001 / 150)$
  for all the tests (4 phantom types $\times$ 2 resolutions).
Since statistical tests may not be conclusive enough, we also computed the confidence intervals,
  reported in \autoref{tab:ci_phantom}.


\begin{table}
		\centering
		\footnotesize
    \tabcolsep=0.1cm
    \begin{tabular}{lccccc}
    Res.   & ``gyrus'' & ``ball''  & ``box''   & ``L''     & Aggreg.    \\\hline
    1.0mm  & 0.18-0.38 & 0.31-0.45 & 0.34-0.42 & 0.34-0.40 & 0.34-0.38  \\
    2.0mm  & 0.59-0.60 & 0.65-0.76 & 0.68-0.71 & 0.67-0.77 & 0.64-0.66  \\
    \hline
    \end{tabular}
    \caption{The vertex-wise Haussdorf distance between the ground-truth surfaces and their
    corresponding estimation with \regseg{} presented distributions with 95\% CI below
    the half-voxel size in all the phantom types.
    CIs were computed using bootstrapping of 10$^4$ samples, and median as location statistic.}\label{tab:ci_phantom}
\end{table}

\subsection{Evaluation on real datasets and cross-comparison}\label{sec:results_hcp}
%
Finally, we compared the performance of \regseg{} and the standard \gls*{t2b}
  method.
Visual assessment of the 16 cases is included in the \suppl{section S5}.
In \autoref{fig:results_real}, box A, the visual report for one subject is supplied.
We computed the \gls*{swindex} \eqref{eq:swindex} of every surface after registration,
  for both \regseg{} and the \gls*{t2b} methods.
Finally, to statistically compare the results, we performed Kruskal-Wallis H-tests
  (a non-parametric alternative to ANOVA) on the warping indices for three specific 
  surfaces of interest selected in \autoref{sec:experiments_evaluation}.
All the statistical tests evidenced that error distributions obtained with \regseg{} and
  the \gls*{t2b} were significantly different, and using the violin plots in box B of
  \autoref{fig:results_real} we observed that errors are always larger for \gls*{t2b}.
We also reported the 95\% CIs of the \gls*{swindex} for those surfaces.
The aggregate CI for \regseg{} was 1.08 - 1.50 [mm], whereas the \gls*{t2b} method
  yielded an aggregate CI of 2.06 - 2.43 [mm].
The results of the statistical tests and the CIs are summarized in \autoref{tab:results_real}.



\begin{table}
		\centering
		\footnotesize
		\tabcolsep=0.08cm
    \begin{tabular}{cccccc}
    & & $\Gamma_{VdGM}$  & $\Gamma_{WM}$ & $\Gamma_{pial}$ & Aggregate \\
    \hline
    \multirow{2}{*}{CI}
       & \regseg{}        & 0.50 - 0.78 & 0.50 - 0.55 & 0.66 - 0.73 & 0.56 - 0.66 \\
       & T2B                  & 1.78 - 2.58 & 1.94 - 2.36 & 2.16 - 2.58 & 2.05 - 2.39 \\
    \hline
    \multirow{2}{*}{H-tests}
       & p-value  & 4.1$\times$10$^{-6}$& 2.3$\times$10$^{-6}$& 2.3$\times$10$^{-6}$ & 1.8$\times$10$^{-16}$ \\
       & H-stat   & 21.20               & 22.31               & 22.31                & 67.85              \\
    \hline
    \end{tabular}
    \caption{The statistical analysis of results on real data supported that \regseg{} overperformed
    the alternative \acrfull{t2b} method.
    The distribution of errors computed for the surfaces of interest ($\Gamma_{VdGM}$, $\Gamma_{WM}$, $\Gamma_{pial}$)
      and the aggregate of all surfaces presented lower 95\% CIs for \regseg{}.
    CIs reported in this table were computed using bootstrapping with mean as location statistic and 10$^4$ samples.
    The Kruskal-Wallis H-tests indicated that there is a significant difference between \regseg{} and
      the \gls*{t2b} method.
    }\label{tab:results_real}
\end{table}

\begin{figure*}
  \centering
  \includestandalone[width=\linewidth]{figure06}
  \caption{A. Example of one report for visual assessment, automatically generated by the evaluation instrument.
    Each view shows one component of the input image (in this case, the \gls*{fa} map), the ground-truth location
    of the surfaces (green contours), and the resulting surfaces with the method under test (yellow contours).
  First two rows show axial slices for \regseg{} and the \acrfull*{t2b} method, and the last two rows
    show corresponding sagittal views.
  Coronal view is omitted since it is the least informative due to the the directional property
    of distortions.
  Certain regions where \regseg{} overperformed the \gls*{t2b} have been zoomed in.
  B. Violin plots of error distributions of each surface, with indication of the voxel size of the \gls*{dmri} images
    (1.25 mm), and supporting the improved results of \regseg{} in the proposed settings.
	}\label{fig:results_real}
\end{figure*}
% -*- root: 00-main.tex -*-
\section{Discussion}
\label{sec:discussion}

\newcomment[RV\#1(C.3)]{%
\paragraph*{Contributions}\label{sec:related_work} %
Previously, joint segmentation and registration have been applied successfully to other problems
such as longitudinal object tracking \citep{paragios_level_2003} and atlas-based
  segmentation \citep{gorthi_active_2011}.
The most common approach to solve the problem simultaneously is optimizing a deformation model
  (registration) that supports the evolution the active contours (segmentation), like
  \citep{paragios_level_2003,yezzi_variational_2003}.
\cite{unal_coupled_2005}, and later \cite{wang_joint_2006}, replaced the linear registration
  transform of \cite{yezzi_variational_2003} with a free-form deformation field.
\cite{droske_mumfordshah_2009} reviewed the existing techniques and proposed two different
  approaches for applying the Mumford-Shah functional \citep{mumford_optimal_1989} during simultaneous
  registration and segmentation by propagating the deformation field from
  the contours onto the whole image definition.
Finally, \cite{gorthi_active_2011} extended the existing methodologies using a multiphase
  level set function to register several active contours during the application
  of atlas-based segmentation.
Unlike the aforementioned registration-and-segmentation methods, \regseg{} uses shape-gradients
  \citep{herbulot_segmentation_2006} computed on a hierarchical set of explicit surfaces
  (triangularized meshes) that substitute the multiphase level sets.
Recently, \cite{guyader_combined_2011} proposed a simultaneous segmentation and
  registration method in 2D using level sets and a nonlinear elasticity smoother on the
  displacement vector field, which preserves the topology even with very large deformations.
\Regseg{} includes an anisotropic regularizer for the displacements field proposed by
  \cite{nagel_investigation_1986}.
This regularization strategy conceptually falls in the midway between the Gaussian smoothing
  generally included in most of the existing methodologies, and the complexity of
  the elasticity smoother of \cite{guyader_combined_2011}.

An important antecedent of \regseg{} is \emph{bbregister} \citep{greve_accurate_2009}.
The tool has been widely adopted as the standard registration method to be used along with the \gls*{epi}
  correction of choice.
It implements a linear mapping, and uses 3D active contours with edges\footnote{A three-dimensional active contour
  is a surface and it is ``with edges'' when the contours evolve in search of abrupt intensity steps
  (the edges) within the target image.} to search for intensity boundaries in the \lowb{} image.
The active contours are initialized using surfaces extracted from the \gls*{t1} using
  \emph{FreeSurfer} \citep{fischl_freesurfer_2012}: the \emph{pial} surface (exterior of the
  cortical \gls*{gm}) and the \emph{white} surface (the \gls*{wm}/\gls*{gm} interface).
The \lowb{} image only includes a detectable frontier for the pial surface, and thus
  \emph{bbregister} is limited to aligning the cortical layer in this
  application.
To overcome the problem of nonlinear distortions, \emph{bbregister} excludes the
  regions that are typically warped from the boundary search.
Indeed, the distortion must be addressed separately because it is not supported by
  the affine transformation model.
\Regseg{} can be seen as a derivative of \emph{bbregister} for the use of active surfaces.
However, the deformation model is nonlinear and the surfaces in \regseg{} look for the
  homogeneity within the regions enclosed, instead of image edges.}

\newcomment[RV\#1(C.5)]{%
Finally, we also contribute proposing a piecewise-smooth segmentation model defined by
  a selection of nested surfaces to partition the multispectral space
  comprehending the \gls*{fa} and the \gls*{adc} maps and ultimately identify these
  structures in \gls*{dmri} space.}
With the definition of appropriate models, \regseg{} may be applied in other fields like
  neonatal brain image segmentation in longitudinal \gls*{mri} studies like \citep{shi_neonatal_2010}
  in which the surfaces obtained in a mature timepoint of the brain can be retrospectively
  propagated to the initial timepoints, regardless the changes of the contrast and spatial
  development between timepoints.

\paragraph*{Accuracy tests}
The hypothesis tested in our study was that reliable image registration can be performed
  by searching for the homogeneous regions defined by a set of nested surfaces in a
  multispectral image
The surfaces correspond to precise contours extracted from another image of the same subject
  (i.e. a \gls*{t1} image, or a different time step).
We demonstrated that active contours without edges can be used successfully to drive a
  deformation supported by B-spline basis functions with digital phantoms.
We randomly deformed four different phantom models to mimic three homogeneous regions
  (\gls*{wm}, \gls*{gm}, and \acrlong*{csf}) and we used them to simulate \gls*{t1} and \gls*{t2}
  images at two resolution levels.
After registration with \regseg{}, we measured the Hausdorff distance between the
  projected contours obtained using the ground-truth warping and our estimates.
We concluded that the errors were significantly lower than the voxel resolution.
We also assessed the 95\% \gls*{ci}, which yielded an aggregate interval of
  0.64--0.66 [mm] for the low resolution phantoms (2.0 mm isotropic voxel) and
  0.34--0.38 [mm] for the high resolution phantoms (1.0 mm isotropic).
Therefore, we also concluded that the error was bounded above by half of the
  voxel spacing.
\newcomment[RV\#1(C.8)]{%}
The distributions of errors along surfaces vary importantly depending on the shape of the
  phantom (see \autoref{fig:phantom}B).
The misregistration error of the ``gyrus'' phantom showed a much lower spread than that
  for the other shapes.
We argue that the symmetry of those other shapes posed difficulties in driving the contours
  towards the appropriate region and producing some ``sliding'' effect between the
  faces of the surfaces and their ground-truth position.
This effect should not be present in real datasets, thanks to the very convoluted cortical
  layer, and the directional restriction of distortion.}

\paragraph*{Application to real data}
We designed \regseg{} as a method for segmenting \gls*{dmri} data by exploiting the
  anatomical information extracted from a structural image (typically a \gls*{t1} image)
  of the subject.
Current pipelines using whole-brain tractography usually solve this problem with a two-step approach.
First, the images are corrected for nonlinear distortions using auxiliary acquisitions
  such as fieldmaps \citep{jezzard_correction_1995}, \glspl*{dwi} with reversed \gls*{pe}
  blips \citep{chiou_simple_2000}, or \gls*{t2} images \citep{kybic_unwarping_2000}.
Second, the segmentation is projected from a reference \gls*{t1} image using linear
  registration \citep{greve_accurate_2009}.
\Regseg{} addresses this joint problem in a single step and it does not require any additional
  acquisition other than the minimal protocol using only \gls*{t1} and \gls*{dmri} images.
This situation is found commonly in historical datasets.
Moreover, since the structural information is projected into the native space of \gls*{dmri}
  reconstruction and tractography can be performed without resampling data to an undistorted
  space.

We evaluated \regseg{} in a real environment using the experimental framework presented
  in \autoref{fig:evworkflows}.
We processed 16 subjects from the \gls*{hcp} database using both \regseg{}
  and an in-house replication of the \acrfull*{t2b} method.
\Regseg{} obtained very high accuracy, with an aggregate 95\% \gls*{ci} of 0.56--0.66 [mm], which was
  below the pixel size of 1.25 mm.
The misregistration error that remained after \regseg{} was significantly lower ($p <$ 0.01) than the
  error corresponding to the \gls*{t2b} correction according to Kruskal-Wallis H-tests
  (\autoref{tab:results_real}).
Visual inspections of all the results (\suppl{section S5}) and the violin plots in
  \autoref{fig:results_real} confirmed that \regseg{} performed better than the \gls*{t2b} method
  in our settings.
We carefully configured the \gls*{t2b} method using the same algorithm and the
  same settings employed in a widely-used tool.
However, cross-comparison experiments are prone to so-called \emph{instrumentation bias}
  \citep{tustison_instrumentation_2013}.
Therefore, these results do not prove that \regseg{} \emph{is better than} \gls*{t2b}.
Our results suggest that \regseg{} is a reliable option in this application field.
In addition, the \gls*{t2b} may introduce an additional (and small) error during the necessary
  registration of \gls*{t2} in the \gls*{t1} space.

% \paragraph*{Prospects}
% First extensions of this work will study more appropriate features to build the energy functional
%   on, enabling to perform \regseg{} directly on the raw \gls*{dmri} data.
% A second outlook covers incorporating knowledge about the distortion by initializing our method
%   with the theory-based displacement field that can be estimated with fieldmaps.


% -*- root: 00-main.tex -*-
\section*{Conclusion}
\label{sec:conclusion}

\emph{Regseg} is a variational framework to simultaneously segment and
  register 3D \gls*{dmri} data of the human brain, using within-subject
  anatomical information as reference.
The registration method segments the target multivariate image in several competing regions
  defined explicitly by their limiting surfaces.
The surfaces are active, and evolve on a free-deformation field supported by B-Splines.
A descent optimization strategy is guided by shape gradients computed on the current partition
  of the target image.
\emph{Regseg} uses active contours without edges, and hence looks for
  homogeneous regions within the image.
We tested \emph{regseg} on digital phantoms simulating \gls*{t1} and \gls*{t2} \gls*{mri},
	warped with smooth random deformations.
The resulting average misplacement of the contours was significantly lower than the
  image resolution of the phantoms.

We propose \emph{regseg} to correct \gls*{dmri} datasets for the typically present distortions
  caused by inhomogeneities of the main magnetic field.
Visual assessment of \emph{regseg} and cross-comparison with a widely-used competing
  technique demonstrated its accuracy.
Moreover, \emph{regseg} does not require additional images to the minimal acquisition protocol
  that includes only \gls*{t1} and \gls*{dmri}.

\section*{Availability and reproducibility statement}
\label{sec:availability}
We considered the reproducibility of our results a design requirement.
Therefore, we used real data fetched from a publicly available repository
  (the Human Connectome Project \citep{essen_human_2012}) and all the software
  involved in this paper is also publicly released.
The \emph{regseg} algorithm is implemented in C++, on top of ITK-4.6
  (Insight Registration and Segmentation Toolkit, \url{http://www.itk.org}).
The evaluation instruments (\autoref{fig:evworkflows}), are implemented in Python using
  \emph{nipype} \citep{gorgolewski_nipype_2011} to assess their reproducibility.
Phantom data are automatically generated by the evaluation framework.
All research elements (data, source code, figures, manuscript sources, etc.) in this paper
  are made publicly available under a unique digital object identifier in a
  package \citep{esteban_acweregistration_2015}.

% ACKs
% -*- root: 00-main.tex -*-
% use section* for acknowledgement
\section*{Author contributions}
All the authors contributed in editing the paper.
OE implemented the method, designed and conducted the experiments, wrote the paper,
  simulated the phantoms and prepared the real data.
DZ devised and drafted the registration method, generated early phantom datasets, and
  collaborated with the implementation of the method.
AD, MBC and MJLC interpreted the results.
AD, MBC, MJLC, JPT and AS advised on all aspects of the work.

\section*{Acknowledgment}
The authors thank V. Estellers for early discussions at the beginning of this project,
  and L. A. Vese for her support during OE's research visits in her laboratory.

This study is supported by: the Spain's Ministry of Science and Innovation
  (projects TEC2010-21619- C04-03, TEC2011-28972-C02-02, CDTI-CENIT
  AMIT and INNPACTO PRECISION), Comunidad de Madrid (ARTEMIS) and
  European Regional Development Funds; the Center for Biomedical Imaging
  (CIBM) of the Geneva and Lausanne Universities and the EPFL, as well as the
  Leenaards and Louis Jeantet foundations.
DZ is supported by the Swiss National Science Foundation (SNF) under grant PBELP2-137727.



% Add appendices
\appendix
\numberwithin{equation}{section}
% -*- root: 00-main.tex -*-
\renewcommand{\theequation}{A.\arabic{equation}}
\renewcommand{\thesubsection}{Appendix \arabic{subsection}}

\section*{Appendix}

\subsection{Simplifying the regularization term}\label{app:reg_term}
The exponentials of the Thikonov regularization prior \eqref{eq:regseg-thikonov} have the general form
  $\vec{v}^T \mathbf{M} \vec{v}$.
If $\mathbf{M}$ is a $n \times n$ diagonal matrix such that $\mathbf{M} = \vec{m} \, \mathbf{I}_n$, 
  then:

\begin{equation*}
\vec{v}^T \mathbf{M} \vec{v} = \vec{m} \cdot (\vec{v}^T \mathbf{I}_n \vec{v}) = \vec{m} \cdot \vec{v}^{\circ2},
\end{equation*}
  where we have introduced the Hadamard power notation\footnote{The Hadamard power of a matrix or a vector
  is the power of its elements $\mathbf{M}^{\circ p} = ({m_{ij}}^{p})$}.

In general, the anisotropy is aligned with the imaging axes, so 
  $\mathbf{A}$ and $\mathbf{B}$ of \eqref{eq:regseg-energy} can be simplified to diagonal matrices, such that
  $\mathbf{A}= \, \boldsymbol{\alpha}\,\vec{I}_n$ and
  $\mathbf{B}= \, \boldsymbol{\beta}\,\vec{I}_n$.
By substituting into equation \eqref{eq:regseg-energy}, we obtain:

  \begin{align}
  E(\vec{u}) &= \const + \underset{l}{\sum} \int_{\Omega_l} \mdist{f'}{l} \,d\vec{r} +   \int_{\Omega} \frac12 \left[ \boldsymbol{\alpha} \cdot \vec{u}^{\circ2} + \boldsymbol{\beta} \cdot (\nabla \vec{u})^{\circ2} \right] \,d\vec{r}.
  \label{eq:regseg-app_energy}
  \end{align} 

%\bibliographystyle{plainnat}
\bibliographystyle{mystyle}
\bibliography{Remote}


\end{document}
