% -*- root: 00-main.tex -*-
\section*{Introduction}\label{sec:introduction}
Nonlinear registration of images is a challenging task, ubiquitous
  in an endless number of image analysis applications.
The process aims to find a spatial mapping $U$ that aligns the information available
  in two different coordinate systems (generally called
  \emph{reference}, $R$, and \emph{moving}, $M$):%

  \begin{align}
  U\colon R \subset \mathbb{R}^n &\to M \subset \mathbb{R}^n \notag\\
  \vec{r} &\mapsto \vec{r}' =\vec{r}+u(\vec{r}),
  \label{eq:transform}
  \end{align}

  where $\vec{r}$ denotes a position in the reference domain $R$, $\vec{r}'$ is
  its corresponding location in $M$, and $n$ the dimensionality of images.
Finally, $\vec{u} = u(\vec{r})$ is the displacement of every point with respect
  to the reference domain.
A comprehensive survey by \cite{sotiras_deformable_2013}
  illustrates the wide variety of existing registration methodologies.
They classify the algorithms by the three principal building blocks that can be identified
  among methods: the matching criteria or similarity function to evaluate the proximity of
  the solution, the deformation model that may implement theoretical properties and constraints
  of the mapping in the selected application, and the optimization method or search strategy.
The possibly most widespread matching criteria are those classified as \emph{iconic
  methods} in \citep{sotiras_deformable_2013}.
These algorithms require the definition of an appropriate function on the voxel-wise information
  content of both images.
A second group, called \emph{geometric methods}, align geometrical features derived from
  corresponding objects defined in both spaces, such as landmarks, shapes, surfaces, etc.
Finally, \emph{hybrid methods} map geometrical features and voxel-wise information.

In this work we propose a hybrid approach to nonlinear registration of anatomically correct
  and precise surfaces to warped and low-resolution \gls*{dmri} of the brain.
Currently, we are witnessing an uprising demand for such algorithms in the quest
  of extracting and characterizing the networks of structural connectivity in the
  human brain \citep{craddock_imaging_2013}.
Our registration method is demonstrated here as a solution to correct for the susceptibility
  distortions that \gls*{dmri} of the head typically present \citep{jezzard_correction_1995}.
The precise segmentation of \gls*{dmri} data is also crucial in such applications, since the
  definition of the nodes in the network is generally derived from a parcellation performed in
  an anatomical \gls*{mri} of the same subject \citep{daducci_connectome_2012}.

\Gls*{dmri} data are usually acquired with \gls*{epi} schemes.
In order to quickly probe  in many different orientations the diffusion process within the living
  brain, these sequences minimize the bandwidth of the \gls*{pe} direction.
This low bandwith originates the artifactual displacement of locations presenting small deviations
  from the main magnetic field $B_0$.
The principal source of these deviations, $\Delta B_0$, is the discontinuity of magnetic
  susceptibility at tissue interfaces.
Tissue/air boundaries show large steps of susceptibility and this artifact is
  particularly evident in the surroundings of the orbito-frontal lobe or the temporal
  bones in both hemispheres.
The displacements field $U_s$ along the \gls*{pe} axis can be estimated theoretically
  \citep{jezzard_correction_1995}:

  \begin{equation}
  \vec{r}' = \vec{r} + \frac{\gamma \, T_{acq}\, s_{PE}}{2\pi}\Delta B_0(\vec{r}) \cdot \hat{\vec{e}}_{PE}.
  \label{eq:fieldmap}
  \end{equation}

where $\vec{r}$ is the location in the reference domain as in \eqref{eq:transform},
  $\gamma$ is the gyromagnetic ratio, $T_{acq}$ is the readout time,
  $s_{PE}$ is the pixel spacing along \gls*{pe}, and $\hat{\vec{e}}_{PE}$ the unitary
  vector along \gls*{pe}.
Several methodologies have been proposed to estimate $U_s$ and fix the artifact.
The first solution was acquiring extra images to estimate $\Delta B_0$ using field mapping
  techniques \citep{jezzard_correction_1995,andersson_modeling_2001}.
A second set of methods aquired extra images reproducing the same \gls*{epi} scheme but switching
  the \gls*{pe} axis \citep{chiou_simple_2000} or reversing the direction of gradient
  increments in the same axis \citep{cordes_geometric_2000,holland_efficient_2010}.
\cite{kybic_unwarping_2000} proposed a correction using nonlinear registration of the \gls*{dmri}
  data versus a \gls*{t2} image.
Further developments \citep{irfanoglu_susceptibility_2011} extended this method to the
  three-dimensional problem.
All methods require extra \gls*{mr} aquisitions. Thus, a tipical protocol for the extraction
  of structural connectivity comprehends \gls*{dmri}, \gls*{t1}, and the acquisition demanded
  by the correction method of choice.

Our method is related to the \gls*{t2b} methods as we also render the correction as a
  registration problem.
Conversely, the method does not require any of the previously mentioned extra acquisitions.
A fundamental antecedent to the present approach is found in \citep{greve_accurate_2009}.
They proposed a registration method to align \gls*{t1} and \gls*{dmri} images
  of the brain using a linear transformation model.
Similarly, we exploit the precise structures that can be extracted with
  widely-used software (i.e. \emph{FreeSurfer}~\citep{fischl_freesurfer_2012}),
  and then map these surfaces into warped data, in which the scale of the
  imaged structures is in the range of the image resolution or above.
Our approach differs from \citep{greve_accurate_2009} in two fundamental choices.
First, we use a nonlinear deformation model that enables the correction of the
  susceptibility-derived distortions \citep{jezzard_correction_1995}.
Second, our matching criteria does not necessarily need for clear intensity
  steps at the tissue interfaces where surfaces must be fitted, thanks to the use of
  active contours without edges \citep{chan_active_2001}.
The hypothesis underpinning our research is that, in those situations,
  image registration can be reliably performed by searching for homogeneous
  regions in one image corresponding to a prior knowledge of the shape of objects in
  the moving coordinate system.
