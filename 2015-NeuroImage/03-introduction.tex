% -*- root: 00-main.tex -*-
\section{Introduction}\label{sec:introduction}
\Gls*{dmri} enables the mapping of microstructure \citep{basser_microstructural_1996}
  and connectivity \citep{craddock_imaging_2013} of the human brain \emph{in-vivo}.
It is generally acquired using \gls*{epi} schemes, since they are very fast at
  scanning a large sequence of images called \glspl*{dwi}, each sensitized with
  a gradient to probe proton diffusion in a certain orientation.
Subsequent processing involves describing local microstructure with one available
  model, which range from the early \gls*{dti} proposed by \cite{basser_microstructural_1996}
  to current models such as \citep{daducci_accelerated_2015}.
The microstructural map is then used to draw the preferential orientations of diffusion
  across regions using tractography \citep{mori_threedimensional_1999}.
Finally, the streamlines found are filtered with a cortical parcellation
  to build the graph representing the corresponding structural
  network \citep{hagmann_mapping_2008}.
Since the proposal of the earliest methodologies to solve the three problems,
  there is a need for precisely identifying the anatomy on the \gls*{dmri} space.
In fact, current trends in reconstruction \citep{jeurissen_multitissue_2014} and
  tractography \citep{smith_anatomicallyconstrained_2012} are increasingly using
  structural information to improve the microstructural mapping and fiber-tracking.

The structural information required by pioneer tractography studies was a simple
  \gls*{wm} mask to provide with a termination criteria for the streamlines.
The first attempts to segment the \gls*{wm} in \gls*{dti} involved thresholding the
  \gls*{fa}\footnote{the \gls*{fa} is a scalar parameter of diffusion derived from
  the \gls*{dti}.} map.
However, the mask and subsequent analyses are highly dependent on the threshold set
  \citep{taoka_fractional_2009}.
To overcome the unreliability of \gls*{fa} thresholding and broaden \gls*{wm} segmentation
  to brain tissue segmentation, \cite{zhukov_level_2003} proposed active contours with
  edges represented by level sets, which evolve on directionally invariant scalar maps
  computed from \gls*{dti}.
\cite{rousson_level_2004} successfully segmented the corpus callosum with
  region-based level sets based on the maps of eigenvalues corresponding to the
  \gls*{dti}.
Many studies have addressed the definition of appropriate features for segmenting
  regions and/or tissues in \gls*{dmri} datasets, such as the 5D representation
  proposed by \cite{jonasson_segmentation_2005}.
Other efforts include mixture models based on sets of directionally invariant maps
  \citep{liu_brain_2007}, iterative clustering of the \lowb{}\footnote{The \lowb{}
  image is one \gls*{epi} volume acquired as signal reference without sensitizing gradient.}
  image \citep{hadjiprocopis_unbiased_2005},
  hierarchical clustering of the \gls*{fa} and \gls*{adc} maps \citep{lu_segmentation_2008},
  and graph-cuts on combinations of scalar maps \citep{han_experimental_2009}.
However, the precise segmentation of \gls*{dmri} is difficult for several limitations.
First, \gls{dmri} images have a resolution that is much lower than that of the imaged
  microstructural features \citep{basser_microstructural_1996}.
Therefore, voxels located in structural discontinuities are affected by partial
  voluming of the signal sources.
Second, the extremely low \gls*{snr} and the high dimensionality of the \glspl*{dwi} prevent
  their direct use in segmentation.
Third, the low contrast between \gls*{gm} and \gls*{wm} in the \lowb{} volumes also makes
  them unsuitable for brain tissue segmentation.

An alternative route to segmentation in native \gls*{dmri} space the mapping of the
  structural information extracted from anatomical MR images, such as \gls*{t1}, through
  an image registration process.
Generally, intra-subject registration of MR images of the brain involves only a linear
  mapping to compensate for head motion between scans.
However, \gls*{epi} images speed up the acquisition process at the cost of introducing geometrical
  distortion, as well as signal degradation and destruction \citep{jezzard_correction_1995}.
The so-called \emph{\gls*{epi} distortion} have major impacts on the anatomy extracted
  from \gls*{dmri}, particularly in certain fiber bundles \citep{irfanoglu_effects_2012}.
To overcome this problem, two broad families of solutions are available.
First, it is possible to use an existing ``\gls*{epi} distortion correction''
  method to compensate for the nonlinear part of the misalignment with the structural space, combined
  with a rigid registration method.
Second, it is possible to apply a nonlinear registration method with specific constraints
  related to the theoretical properties of the \gls*{epi} distortion.
However, both solutions require extra acquisitions included in the \gls*{dmri} scanning protocol,
  such as fieldmaps \citep{jezzard_correction_1995}, \glspl*{dwi} with a different \gls*{pe}
  scheme \citep{cordes_geometric_2000,chiou_simple_2000}, or \gls*{t2} images.
\cite{saad_new_2009} used the Pearson's correlation coefficient to obtain a linear alignment of
  the \gls*{t1} and the \lowb{}.
Similarly, a linear registration method was employed by \emph{bbregister} \citep{greve_accurate_2009},
  which uses active contours with edges to search for intensity boundaries in the \lowb{}
  image.
The active contours are initialized using surfaces extracted from the \gls*{t1} using
  \emph{FreeSurfer} \citep{fischl_freesurfer_2012}: the \emph{pial} surface (exterior of the
  cortical \gls*{gm}) and the \emph{white} surface (the \gls*{wm}/\gls*{gm} interface).
The \lowb{} image only includes a detectable frontier for the pial surface, so
  \emph{bbregister} is limited to aligning the cortical layer in this
  application.
This tool has become the standard method because of its proven robustness, although the
  distortions found in \gls*{dmri} are nonlinear.
To overcome this issue, \emph{bbregister} excludes the
  regions that are typically warped by artifacts from the boundary search.
However, because the distortion is not considered, it must be addressed separately.
Nonlinear registration has been performed successfully between \gls*{t2} and \lowb{}
  images based on their similarity as a unique method for correcting distortion
  \citep{kybic_unwarping_2000,studholme_accurate_2000,wu_comparison_2008,tao_variational_2009}.
However, further registration of the \gls*{t1} and \gls*{t2} images is still required to map the anatomical
  information into the \gls*{dmri} space.

Therefore, given these issues, a method that simultaneously performs
  segmentation in the native \gls*{dmri} space and registration of the corresponding \gls*{t1} image
  onto the \gls*{dmri} data could provide the optimal solution.
To the best of our knowledge, this strategy has never been proposed for the application described above.
Previously, joint segmentation and registration have been applied successfully to other problems
such as longitudinal object tracking \citep{paragios_level_2003} and atlas-based
  segmentation \citep{gorthi_active_2011}.
The most thorough approach integrates active contours during image registration
  methods.
\cite{unal_coupled_2005}, and later \cite{wang_joint_2006},
  improved an existing method \citep{yezzi_variational_2003} based on linear registration
  to the nonlinear case by implementing a free-form deformation field.
\cite{droske_mumfordshah_2009} reviewed the existing techniques and proposed two different
  approaches for applying the Mumford-Shah functional \citep{mumford_optimal_1989} during simultaneous
  registration and segmentation by propagating the deformation field from
  the contours onto the whole image definition.
Recently, \cite{guyader_combined_2011} proposed a simultaneous segmentation and
  registration method in 2D using level sets and a nonlinear elasticity smoother on the
  displacement vector field, which preserves the topology even with very high deformation.
Finally, \cite{gorthi_active_2011} extended the existing methodologies using a multiphase
  level set function to register several active contours during the application
  of atlas-based segmentation.

The hypothesis tested in the present study is that registration and segmentation
  problems in \gls*{dmri} can be solved simultaneously, thereby increasing the geometrical
  accuracy of the process.
Thus, we propose a tool called \regseg{} that exploits the prior information of shapes
  extracted using \gls*{t1} images to register the anatomical reference
  to \gls*{dmri} space, thereby implicitly segmenting these data.
The proposed approach uses active contours without edges \citep{chan_active_2001}, which evolve to drive a
  free-form deformation field of B-spline basis functions.
Optimization is performed using a descent strategy with explicit shape gradients
  \citep{besson_dream2s_2003,herbulot_segmentation_2006}.
Therefore, unlike most previous methods, \regseg{} does not implement level sets.
The nonlinear distortion is aligned with one of the imaging axes (see
  \autoref{sec:human_connectome}), so \regseg{} includes an anisotropic regularizer for
  the displacements field proposed by \cite{nagel_investigation_1986}.
Finally, we evaluated \regseg{} using an extension of our instrumentation framework
  \citep{esteban_simulationbased_2014}, which simulates known and realistic distortions
  based on \gls*{dmri} data.
The evaluation included a comparison with \gls*{t2b} correction by integrating it within the framework
  of an in-house implementation of the method.
