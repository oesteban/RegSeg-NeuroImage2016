% -*- root: 00-main.tex -*-
\section{Introduction}\label{sec:introduction}
The accurate delineation of \gls*{wm} in \gls*{dmri} and the fusion of prior
  anatomical information extracted from a \gls*{t1} image for the same subject
  are crucial in a range of applications based on tractography, such as
  the extraction of structural connectivity \citep{craddock_imaging_2013} or
  tract-based spatial statistics \citep{smith_tractbased_2006}.
However, the precise segmentation of \gls*{dmri} data and the coregistration of anatomical
  images are difficult due to several limitations.
First, \gls{dmri} images have a resolution that is much higher than that of the imaged
  microstructural features \citep{basser_microstructural_1996}.
Therefore, voxels located in structural discontinuities are affected by partial
  voluming of the signal sources.
Second, \gls*{dmri} schemes probe the diffusion process within the brain at
  many angles to obtain \glspl*{dwi}, which are completed by one or more baseline (\emph{b0})
  volumes without directional gradients.
The extremely low \gls*{snr} and the high dimensionality of \glspl*{dwi} prevent their
  direct use in segmentation.
The low contrast between \gls*{gm} and \gls*{wm} in the \emph{b0} volumes also makes them unsuitable for
  segmentation.
Finally, \gls*{dmri} images are acquired using \gls*{epi} to speed up the acquisition process but
  at the cost of introducing geometrical distortion, as well as signal degradation and
  destruction \citep{jezzard_correction_1995}.
The artifacts generated have major impacts on the anatomy extracted
  from \gls*{dmri}, particularly in certain fiber bundles \citep{irfanoglu_effects_2012}.
These limitations prevent segmentation in the native \gls*{dmri} space, including
  registration-based approaches using anatomical images as the sources of structural
  information.


Early attempts to delineate the \gls*{wm} involved thresholding the
  \gls*{fa}\footnote{the \gls*{fa} is a scalar parameter of diffusion derived from
  the \gls*{dmri} data.} map.
However, the mask and subsequent analyses are highly dependent on the threshold set
  \citep{taoka_fractional_2009}.
To overcome the unreliability of \gls*{fa} thresholding,
  \cite{zhukov_level_2003} proposed active contours with edges represented
  by level sets, which evolve on directionally invariant scalar maps.
\cite{rousson_level_2004} successfully segmented the corpus callosum with
  region-based level sets based on the eigenvalues of tensors fitted using the
  \gls*{dmri} data.
Many studies have addressed the definition of appropriate features for clustering,
  such as the 5D representation proposed by \cite{jonasson_segmentation_2005}.
Other efforts include mixed models based on sets of directionally invariant maps
  \citep{liu_brain_2007}, iterative \citep{hadjiprocopis_unbiased_2005} and
  hierarchical \citep{lu_segmentation_2008} clustering,
  graph-cuts \citep{han_experimental_2009},
  and volume fraction modeling \citep{kumazawa_improvement_2013}.

To address the segmentation problem by registration, \cite{saad_new_2009}
  used the Pearson's correlation coefficient to obtain a linear alignment of the \gls*{t1} and
  the \emph{b0}.
Similarly, a linear registration method was employed by \emph{bbregister} \citep{greve_accurate_2009},
  which uses active contours with edges to search for intensity boundaries in the \emph{b0}
  image.
The active contours are initialized using surfaces extracted from the
  \gls*{t1} using \emph{FreeSurfer} \citep{fischl_freesurfer_2012}:
  the \emph{pial} surface (exterior of the cortical \gls*{gm}) and the \emph{white}
  surface (the \gls*{wm}/\gls*{gm} interface).
The \emph{b0} image only includes a detectable frontier for the pial surface, so
  \emph{bbregister} is limited to aligning the cortical layer in this
  application.
This tool has become the standard method because of its proven robustness, although the
  distortions found in \gls*{dmri} are nonlinear.
To overcome this issue, \emph{bbregister} excludes the
  regions that are typically warped by artifacts from the boundary search.
However, because the distortion is not considered, it must be addressed separately.
Nonlinear registration has been performed successfully between \gls*{t2} and \emph{b0}
  images based on their similarity as a unique method for correcting distortion
  \citep{kybic_unwarping_2000,studholme_accurate_2000,wu_comparison_2008,tao_variational_2009}.
However, further registration of the \gls*{t1} and \gls*{t2} images is still required to map the anatomical
  information (and for \gls*{wm} segmentation) into the \gls*{dmri} space.

Therefore, given these issues, a method that simultaneously performs
  segmentation in the native \gls*{dmri} space and registration of the corresponding \gls*{t1} image
  onto the \gls*{dmri} data could provide the optimal solution.
To the best of our knowledge, this strategy has never been proposed for the application described above.
Previously, joint segmentation and registration have been applied successfully to other problems
such as longitudinal object tracking \citep{paragios_level_2003} and atlas-based
  segmentation \citep{gorthi_active_2011}.
The most thorough approach integrates active contours during image registration
  methods.
\cite{unal_coupled_2005}, and later \cite{wang_joint_2006},
  improved an existing method \citep{yezzi_variational_2003} based on linear registration
  to the nonlinear case by implementing a free-form deformation field.
\cite{droske_mumfordshah_2009} reviewed the existing techniques and proposed two different
  approaches for applying the Mumford-Shah functional \citep{mumford_optimal_1989} during simultaneous
  registration and segmentation by propagating the deformation field from
  the contours onto the whole image definition.
Recently, \cite{guyader_combined_2011} proposed a simultaneous segmentation and
  registration method in 2D using level sets and a nonlinear elasticity smoother on the
  displacement vector field, which preserves the topology even with very high deformation.
Finally, \cite{gorthi_active_2011} extended the existing methodologies using a multiphase
  level set function to register several active contours during the application
  of atlas-based segmentation.

The hypothesis tested in the present study is that registration and segmentation
  problems in \gls*{dmri} can be solved simultaneously, thereby increasing the geometrical
  accuracy of the process.
Thus, we propose a tool called \regseg{} that exploits the prior information of shapes
  extracted using \gls*{t1} images to register the anatomical reference
  to \gls*{dmri} space, thereby implicitly segmenting these data.
The proposed approach uses active contours without edges \citep{chan_active_2001}, which evolve to drive a
  free-form deformation field of B-spline basis functions.
Optimization is performed using a descent strategy with explicit shape gradients
  \citep{besson_dream2s_2003,herbulot_segmentation_2006}.
Therefore, unlike most previous methods, \regseg{} does not implement level sets.
The nonlinear distortion is aligned with one of the imaging axes (see
  \autoref{sec:human_connectome}), so \regseg{} includes an anisotropic regularizer for
  the displacements field proposed by \cite{nagel_investigation_1986}.
Finally, we evaluated \regseg{} using an extension of our instrumentation framework
  \citep{esteban_simulationbased_2014}, which simulates known and realistic distortions
  based on \gls*{dmri} data.
The evaluation included a comparison with \gls*{t2b} correction by integrating it within the framework
  of an in-house implementation of the method.
