% @Author: Oscar Esteban
% @Date:   2015-02-20 15:40:09
% @Last Modified by:   Oscar Esteban
% @Last Modified time: 2015-04-08 13:07:44

\documentclass[a4paper]{report}

\usepackage{titlesec}
\newcommand{\sectionbreak}{\clearpage}

\usepackage{aliascnt}
\usepackage{relsize}
%\usepackage{inconsolata}
\usepackage[scaled]{beramono}
\usepackage[procnames]{listings}

\usepackage{natbib}
%\usepackage[style=numeric-comp,doi=false]{biblatex}

% *** GRAPHICS RELATED PACKAGES ***
\usepackage[table]{xcolor}
\usepackage{graphicx}
\usepackage{ifpdf}
\graphicspath{{figures/},{supplemental/}}

% *** MATH PACKAGES ***
\usepackage[cmex10]{amsmath}
\usepackage{amssymb}

\usepackage[T1]{fontenc}
\usepackage{charter}
%\usepackage[expert]{mathdesign}
%\usepackage[libertine,cmintegrals,cmbraces,vvarbb]{newtxmath}

% *** SPECIALIZED LIST PACKAGES ***
%\usepackage{algorithmic}
%\usepackage{algorithm}


% *** ALIGNMENT PACKAGES ***
%\usepackage{array}

% *** SUBFIGURE PACKAGES ***
\usepackage[font={small}]{caption}
%\usepackage{sidecap}

% *** FLOAT PACKAGES ***
\usepackage{floatrow}
\usepackage[framemethod=tikz]{mdframed}
%\usepackage{float}
% \usepackage{fixltx2e}
% \usepackage{stfloats}
% \usepackage{dblfloatfix}


% *** MISC UTILITY PACKAGES ***
%\usepackage{fancyhdr}
\usepackage{multicol}
\usepackage{tabularx}
\usepackage{longtable}
\usepackage{booktabs} % to use \toprule
\usepackage{makeidx}
%\usepackage[scale=2.0]{ccicons}
\usepackage[colorinlistoftodos]{todonotes}
\usepackage[top=2cm,bottom=1.8cm,left=2.1cm,right=1.7cm,marginparwidth=1.2in]{geometry}
\usepackage{changepage}
\usepackage{epigraph}
\usepackage[toc,nomain,acronym,shortcuts,translate=false]{glossaries}

\usepackage{mathtools}
\usepackage{amsmath}
\usepackage{amssymb}
\usepackage{cancel}
%\usepackage{algorithmic}
\DeclareMathOperator{\tr}{tr}
\DeclareMathOperator{\argmax}{argmax}
\DeclareMathOperator{\argmin}{argmin}
\DeclareMathOperator{\diag}{diag}
\DeclareMathOperator{\dist}{dist}
\DeclareMathOperator{\const}{Const.}
\providecommand{\e}[1]{\ensuremath{\times 10^{#1}}}
\providecommand{\mdist}[2]{ \mathcal{D}_{#2}^2(\mathbf{#1}) }
\let\oldhat\hat
\renewcommand{\vec}[1]{\mathbf{#1}}
\providecommand{\nvec}[1]{\hat{\mathbf{#1}}}

\renewcommand\thesection{S\arabic{section}}
\renewcommand\thesubsection{S\arabic{section}.\arabic{subsection}}
\renewcommand\thefigure{S\arabic{figure}}
\renewcommand{\theequation}{SM\arabic{equation}}

% Listings style -----------------------------------------------------------------------------
\definecolor{pink}{RGB}{255,0,90}
\definecolor{comments}{RGB}{50,120,110}
\definecolor{string}{RGB}{160,0,0}
\definecolor{keywords}{RGB}{0,150,0}
\definecolor{listingbg}{gray}{0.95}

\lstset{basicstyle=\ttfamily\smaller\relax}

\newcommand*{\codeinline}[1]{\colorbox{listingbg}{\lstinline!#1!}}
%\newcommand\codeinline{\lstinline}

\lstnewenvironment{bashcode}[1][]
  {\lstset{language={}}\lstset{%
  showstringspaces=false,
  formfeed=\newpage,
  tabsize=4,
  breaklines=true,
  basicstyle=\ttfamily\smaller\relax,
  keywordstyle=\color{keywords}\bfseries,
  commentstyle=\color{comments}\itshape,
  stringstyle=\color{string},
  showstringspaces=false,
  %identifierstyle=\color{green},
  morekeywords={regseg},
  numbers=none,
  frame=single,
  xleftmargin= 10pt,
  xrightmargin= 10pt,
  framexleftmargin=10pt,
  frameround=tttt,
  fillcolor=\color{listingbg},
  backgroundcolor=\color{listingbg}
}}
{}


\lstnewenvironment{pythoncode}[1][]
  {\lstset{language=bash}\lstset{%
  showstringspaces=false,
  formfeed=\newpage,
  tabsize=4,
  breaklines=true,
  basicstyle=\ttfamily\smaller\relax,
  keywordstyle=\color{keywords}\bfseries,
  commentstyle=\color{comments}\itshape,
  stringstyle=\color{string},
  showstringspaces=false,
  identifierstyle=\color{black!70},
  procnamekeys={def,class},
  morekeywords={models, lambda, forms, as, from, import},
  numbers=left,
  numberstyle=\smaller\color{black!60},
  stepnumber=1,
  numbersep=5pt,
  frame=single,
  xleftmargin= 40pt,
  xrightmargin= 20pt,
  framexleftmargin=20pt,
  frameround=tttt,
  fillcolor=\color{gray!10},
  backgroundcolor=\color{gray!10}
}}
{}

\usepackage[hyphens]{url}
\usepackage[hyphenbreaks]{breakurl} %fixes boxes spanning through pages
\usepackage[colorlinks=true,citecolor=blue,urlcolor=blue,linkcolor=blue]{hyperref}
\usepackage{doi}

% -*- root: 00-main.tex -*-
% List of acronyms used in the text
\newacronym{mr}{MR}{magnetic resonance}
\newacronym{mri}{MRI}{magnetic resonance imaging}
\newacronym{dwi}{DWI}{diffusion weighted image}
\newacronym{dmri}{dMRI}{diffusion MRI}
\newacronym{dw}{DW}{diffusion weighted}
\newacronym{dti}{DTI}{diffusion tensor image}
\newacronym{t1}{T1w}{T1 weighted}
\newacronym{t2}{T2w}{T2 weighted}
\newacronym{csf}{CSF}{cerebrospinal fluid}
\newacronym{wm}{WM}{white matter}
\newacronym{gm}{GM}{grey matter}
\newacronym{epi}{EPI}{echo-planar imaging}
\newacronym{gre}{GRE}{gradient echo sequence}
\newacronym{fa}{FA}{fractional anisotropy}
\newacronym{md}{MD}{mean diffusivity}
\newacronym{adc}{ADC}{apparent diffusion coefficient}
\newacronym{acwe}{ACWE}{active contours without edges}
\newacronym{adf}{ADF}{active deformation field}
\newacronym{map}{MAP}{maximum a posteriori}
\newacronym{snr}{SNR}{signal-to-noise ratio}
\newacronym{pve}{PVE}{partial volume effect}
\newacronym{roi}{ROI}{region of interest}
\newacronym{mrf}{MRF}{Markov Random Field}
\newacronym{pde}{PDE}{partial differential equation}
\newacronym{swindex}{sWI}{surface warping index}
\newacronym{wi}{WI}{Warping index}
\newacronym{nof}{NoF}{number of fibers}
\newacronym{t2b}{T2B}{T2w-registration based}
\newacronym{fft}{FFT}{fast fourier transform}
\newacronym{hcp}{HCP}{Human Connectome Project}
\newacronym{pe}{PE}{phase-encoding}

\makeglossaries

%\acrodef{mr}[MR]{magnetic resonance}
%\acrodef{mri}[MRI]{magnetic resonance imaging}
%\acrodef{dwi}[DWI]{diffusion weighted imaging}
%\acrodef{dw}[DW]{diffusion weighted}
%\acrodef{dti}[DTI]{diffusion tensor imaging}
%\acrodef{t1}[T1]{T1-weighted}
%\acrodef{t2}[T2]{T2-weighted}
%\acrodef{csf}[CSF]{cerebrospinal fluid}
%\acrodef{wm}[WM]{white matter}
%\acrodef{gm}[GM]{grey matter}
%\acrodef{epi}[EPI]{echo-planar imaging}
%\acrodef{fa}[FA]{fractional anisotropy}
%\acrodef{md}[MD]{mean diffusivity}
%\acrodef{acwe}[ACWE]{active contours without edges}
%\acrodef{map}[MAP]{maximum a posteriori}
%\acrodef{snr}[SNR]{signal-to-noise ratio}
%\acrodef{pve}[PVE]{partial volume effect}
%\acrodef{roi}[ROI]{region of interest}


\begin{document}
\title{Supplemental Materials: \\ \emph{Active-contours driven registration method for structure-informed segmentation of diffusion MR images}}
\author{Oscar Esteban and Dominique Zosso}
\date{February 2015}

\maketitle
\section{Extensions to the mathematical formulation of the methods}

\subsection{Computing the gradients of shape-priors}\label{sec:shape_priors}
The computation of gradients at the locations of the active contours in the
  instant $t$ is based on the work of \cite{herbulot_segmentation_2006}.
Let $F(\vec{r})$ be an ``arbitrary'' function over the image domain
  $\Omega = \Omega_l \cup \Omega_m$ splitted in two regions $l$ and
  $m$, and $\Gamma_{l,m}$ a closed boundary between them.
We now derive the domain integral w.r.t. $t$:

  \begin{equation}
  \frac{\partial}{\partial t} \int_\Omega F(\vec{r}) d\vec{r} =
  \int_\Omega \frac{\partial}{\partial t}F(\vec{r}) d\vec{r} 
  - \int_{\Gamma_{l,m}} F(\vec{r}) \left\langle \frac{\partial \Gamma_{l,m} }{\partial t},
  N_{\Gamma_{l,m}}\right\rangle d\vec{r},
  \end{equation}
%
  where $\left\langle\frac{\partial\Gamma_{l,m}}{\partial t}, N_{\Gamma_{l,m}}\right\rangle$ is
  the projection of the boundary movement on the unit inward normal $N_{\Gamma_{l,m}}$.
Assuming that the region descriptors $\{\boldsymbol{\mu}_l, \boldsymbol{\Sigma}_l\}$ vary slowly enough, we can consider
  that $\frac{\partial}{\partial t} F(\vec{r}) = 0$ and thus:

  \begin{equation}
  \frac{\partial}{\partial t} \int_\Omega F(\vec{r}) d\vec{r} =
  - \int_{\Gamma_{l,m}} F(\vec{r}) \left\langle \frac{\partial \Gamma_{l,m} }{\partial t},
  N_{\Gamma_{l,m}}\right\rangle d\vec{r}.
  \label{eq:shape_gradients}
  \end{equation}

The equation \eqref{eq:shape_gradients} is discretized as follows.
First, the surface between limiting regions $\{l, m\}$, $\Gamma_{l,m}$ is explicitly represented by
  a discrete set of vertices $\vec{v}_i$, with $i \in \{0, \ldots, N_p -1 \}$.
Consequently, the inwards normal of the surface $N_{\Gamma_{l,m}}$ is represented by the discrete
  set of normals $\hat{\vec{n}}_i$ at each vertex of the mesh.
The resulting summation is, therefore, discrete and the integral operator is replaced by the sum:
  \begin{align}
  \frac{\partial}{\partial t} \int_\Omega F(\vec{r}) d\vec{r} &=
  \underbracket{\cancel{\int_\Omega \frac{\partial}{\partial t}F(\vec{r}) d\vec{r} }}_{\text{Functional's evolution}}
  - \underbracket{\int_{\Gamma_{l,m}} F(\vec{r}) \left\langle \frac{\partial \Gamma_{l,m}}{\partial t},
  N_{\Gamma_{l,m}}\right\rangle d\vec{r}}_{\text{Shape's evolution}} \notag \\
  & = - \underset{p}{\sum} \frac{1}{A_p} \underset{i}{\sum} \, a_i \, F(\vec{v}_i) \left\langle \underbracket{\frac{\partial \vec{v}_i}{\partial t}}_{\text{speed of }\vec{v}_i},
  \hat{\vec{n}}_{i}\right\rangle.
  \label{eq:shape_gradient_orig}
  \end{align}
where $a_i$ is the area corresponding to vertex $\vec{v}_i$, and $A_p = \sum_i a_i$ is the total area of surface $p$.
In the following, we will refer as $w_{p,i} = a_i / A_p $ to the area contribution of $\vec{v}_i$ to the
  total area of the surface it belongs to.
For simplicity, the sum over $p$ can be also removed, as the vertices belong to only one of the total $P$ contours

Then, the speed of $\vec{v}_i$ is discretized using the artificial time-step parameter $\delta$, as the displacement
  $\frac{\partial \vec{v}_i}{\partial t} = \vec{v}_i(\delta = t+1) - \vec{v}_i(\delta = t)$:
  \begin{equation}
  \frac{\partial}{\partial t} \int_\Omega F(\vec{r}) \, d\vec{r} =
  - \underset{i}{\sum} w_{p,i} F(\vec{v}_i) \frac{\partial \vec{v}_i}{\partial t} \cdot \hat{\vec{n}}_i.
  \label{eq:shape_gradient_disc1}
  \end{equation}

Since the energy functional is defined over competing regions, the displacement of $\vec{v}_i$ will cause
  an energy exchange between the limiting regions, and therefore $F(\vec{r})$ must be split in
  two terms, $F_{in}(\vec{r})$ corresponding to the interior region and $F_{out}(\vec{r})$ to the exterior:
  \begin{equation}
  \frac{\partial}{\partial t} \int_\Omega F(\vec{r}) \, d\vec{r} =
  - \underset{i}{\sum} \, \frac{\partial \vec{v}_i}{\partial t} \cdot
  \underbracket{w_{p,i} \, \Big[ F_{out}(\vec{v}_i) - F_{in}(\vec{v}_i) \Big] \hat{\vec{n}}_i}_{\bar{s}_i \text{ in Figure 1}}.
  \label{eq:shape_gradient_disc2}
  \end{equation}

\subsection{Gradient-descent optimization}\label{sec:gradient_descent}
The energy functional to be optimized in \emph{regseg} is presented in Eq. 7.
After the simplifications described in equation (A.1) of Appendix 1, we obtain the
  following energy functional:
  \begin{equation}
  E(\vec{u}) = C + \underbracket{\underset{l}{\sum} \int_{\Omega_l}
  \mdist{f'}{l} \,d\vec{r}}_{\text{Data term } (E_{data})}
  + \underbracket{\underset{\Omega}{\int} \left[ \boldsymbol{\alpha} \cdot \vec{u}^{\circ2}
  + \boldsymbol{\beta} \cdot (\nabla \vec{u})^{\circ2} \right] \,d\vec{r}}_{\text{Regularization term } (E_{reg})}.
  \label{eq:energy}
  \end{equation}

To search for the minimum of $E(\vec{u})$ w.r.t. the coefficients $\vec{u}_k$, we use a gradient descent strategy.
In Eq. (9) we introduced the derivative of \eqref{eq:energy}:
  \begin{equation}
  \frac{\partial E(\vec{u})}{\partial \vec{u}_k} =
  \frac{ \partial }{\partial \vec{u}_k} \Big\{
  \underset{l}{\sum} \int_{\Omega_l} \mdist{f'}{l} \,d\vec{r}
  + \int_{\Omega} \frac12 [ \boldsymbol{\alpha} \cdot \vec{u}^{\circ2}
  + \boldsymbol{\beta} \cdot (\nabla \vec{u})^{\circ2} ] \,d\vec{r}
  \Big\}.
  \label{eq:gradient_descent}
  \end{equation}

We split \eqref{eq:gradient_descent} in the derivatives of its data and regularization terms.
Let $\frac{\partial E_{data}}{\partial \vec{u}_k} = \vec{g}_k$ for simplicity, 
  we compute the derivative of the data term and discretize the domain $\Omega$ as follows:
\begin{equation}
  \vec{g}_k =
  \frac{ \partial }{\partial \vec{u}_k} \Big\{
  \underset{l}{\sum} \int_{\Omega_l} \mdist{f'}{l} \Big\} =
  \frac{ \partial }{\partial \vec{u}_k} \Big\{ \underset{l}{\sum} \underset{\vec{x} \in \Omega_l}{\sum} \, \mdist{f'}{l} \Big\},
  \label{eq:data_derivative}
\end{equation}
  where we can apply the shape-gradients \eqref{eq:shape_gradient_disc2} introduced
  in \autoref{sec:shape_priors}, and ultimately avoid implementing level sets:

  \begin{align}
  \vec{g}_k &= \underset{i}{\sum} \left\langle \frac{\partial \vec{v}_i'}{\partial \vec{u}_k}, \bar{s}_i'\right\rangle, \\
  \text{with }
  \bar{s}_i' &= - w_i \left[ \mdist{f_i'}{out} - \mdist{f_i'}{in} \right] \, \hat{\vec{n}}_i, \\
  \text{and } 
  \frac{\partial \vec{v}_i'}{\partial \vec{u}_k} &=
  \frac{\partial}{\partial \vec{u}_k} \left\{ \vec{v}_i + \sum_k \psi_k(\vec{v}_i) \vec{u}_k \right\} = \psi_k(\vec{v}_i)\, \hat{\vec{e}}.
  \label{eq:gradient_wshape}
  \end{align}%
  where $\hat{\vec{e}}$ is the coordinates system's unit vector.
Therefore, the shape gradients projected to the grid of B-Spline control points is:
\begin{equation}
  \vec{g}_k = - \underset{i}{\sum} \bar{s}_i \cdot \psi_k(\vec{v}_i) \, \hat{\vec{e}} = - \underset{i}{\sum} \vec{g}_{i,k}.
  \label{eq:shape_gradient_final}
\end{equation}

It is also necessary to obtain and discretize the derivatives of the regularization term of \eqref{eq:gradient_descent}:
  \begin{equation}
  \frac{\partial E_{reg}(\vec{u})}{\partial \vec{u}_k} = \frac{ \partial }{\partial \vec{u}_k} \Big\{
  \int_{\Omega} \frac12 [ \boldsymbol{\alpha} \cdot \vec{u}^{\circ2}
  + \boldsymbol{\beta} \cdot (\nabla \vec{u})^{\circ2} ] \,d\vec{r}
  \Big\} = \boldsymbol{\alpha} \cdot \vec{u}_k - \boldsymbol{\beta} \cdot \Delta \vec{u}_k.
  \label{eq:gradient_regterm}
  \end{equation}

Inserting \eqref{eq:shape_gradient_final} and \eqref{eq:gradient_regterm} into \eqref{eq:gradient_descent} we get the
  final evolution equation:
  \begin{align}
  \frac{\partial \vec{u}}{\partial t} &\propto - \frac{\partial E(\vec{u})}{\partial \vec{u}_k}, \notag\\
  \frac{\partial \vec{u}}{\partial t} &\propto - \Big( \vec{g}_k - \boldsymbol{\alpha} \cdot \vec{u}_k + \boldsymbol{\beta} \cdot \Delta \vec{u}_k \Big).
  \label{eq:gradient_final}
  \end{align}


\subsection{Obtaining the update equation} To solve the differential equation in \eqref{eq:gradient_final},
  we use a semi-implicit Euler scheme, referring to the discrete step size as $\delta$ and where the
  shape-gradients $\vec{g}_k$ are explicit:
\begin{align}
  \vec{u}_k^{t+1} &= \vec{u}_k^{t} + \delta \, \Big( - \vec{g}_k^{t} - ( \boldsymbol{\alpha} - \boldsymbol{\beta} \Delta) \, \vec{u}_k^{t+1} \Big) \notag \\
  (1 + \delta \, \boldsymbol{\alpha} - \delta \, \boldsymbol{\beta} \Delta)\, \vec{u}_k^{t+1} &= \vec{u}_k^{t} - \delta\, \vec{g}_k^{t}
  \label{eq:discrete_derivative}
\end{align}

This expression is easily translated into Fourier domain as follows:

  \begin{align}
  \vec{u}_k^{t+1} = \mathcal{F}^{-1}\left\{ \frac{\mathcal{F}\{\delta^{-1} \, \vec{u}_k^t - \vec{g}_k^t\} }%
                  {\mathcal{F}\{(\delta^{-1} + \boldsymbol{\alpha})\, I-\boldsymbol{\beta}\Delta\}} \right\},
  \label{eq:update_equation}
  \end{align}
  where $I$ denotes the identity operator.
Here, we rewrite the Laplacian as a linear combination of the identity and shift operators:
  \begin{equation}
  \Delta = \Big\{\sum_{d=1}^n \mathcal{S}_d^- + \mathcal{S}_d^+\Big\}  - 2n \mathcal{I}
  \end{equation}
  where $\mathcal{S}_d^{\pm}$ stands for the forward ($+$) and backward ($-$) shift operator along axis $d$, of which the Fourier transform is found easily as
  \begin{equation}
  \mathcal{F}\{\mathcal{S}_d^{\pm}\} = e^{\pm i\omega_d},
  \end{equation}
  where $\omega_d$ is the normalized pulsation along direction $d$.
Accordingly, the Fourier transform of the discrete Laplacian is found as
  \begin{align}
  \mathcal{F}\{\Delta\} &= \sum_d e^{-i\omega_d} + e^{i\omega_d} - 2n = n \, \left( \sum_d \cos(\omega_d) - 2\, \right)
  \end{align}

The remaining transforms are trivial or can be computed using FFT as in \citep{estellers_efficient_2012}.

%Then, introducing \eqref{eq:intp_kernel} into \eqref{eq:nodes_tfm} and replacing
%  $\psi$ by the actual kernel function, the transformation writes:
%
%  \begin{equation}
%    \vec{v}_i' = \vec{v}_i + \sum_k \left[ \vec{u}_k \, \underset{d}{\prod}
%      \beta_3( (\vec{v}_i - \vec{r}_k) \cdot \hat{\mathbf{e}}_d ) \right],
%  \label{eq:transformation}
%  \end{equation}
%%
%  with $\hat{\mathbf{e}}_d$ being the unitary vector along axis $d$.

\newpage
\section{Parameter settings and implementation details of \emph{regseg}}

\subsection{Implementation}
\subsection{General} The \emph{regseg} registration and segmentation tool is written in C++, using ITK-4.6 as
  implementation core.
We designed a modular implementation, enabling multithreading in several pieces of the software,
  as the process is computationally expensive.
The tool generates a log-file in JSON format to easily inter-operate with secondary tools (such
  as the convergence report generation, \autoref{sec:convergence_evidence}).

\subsection{Efficient interpolation using sparse matrices}
During the registration process, every iteration requires computing the product of all the gradients
  $\vec{g}_k$ associated to the control point $k$, and computed at the current position
  $\vec{v}_i$ by the corresponding weights $\psi_{ik} = \psi_k(\mathbf{v}_i)$ of interpolating functions
  \eqref{eq:shape_gradient_final}.
In order to optimize multiplications and summations, all the $\psi_{ik}$ are collected in a
  matrix $\boldsymbol{\Psi} = (\psi_{ik})$.
Given the limited support of the basis function $\psi$, $\boldsymbol{\Psi}$ will hold the property
  of being sparse, as only few $\psi_{ik} > 0$ in the surroundings of $\vec{v}_i$.
Then, the gradients $\vec{g}_k$ are easily computed using the matrix product:
\begin{equation}
  \big(\vec{g}_k \big) = \boldsymbol{\Psi} \cdot \big( w_i \, \bar{s}_i \big)^T
  \label{eq:sparse_matrix}
\end{equation}

As these weights can be computed once in the beginning of the process and they do not change along
  it, $\boldsymbol{\Psi}$ can be pre-cached.

\subsection{Assessment of the segmentation model}
\begin{figure}[!ht]
  \floatbox[{\capbeside\thisfloatsetup{capbesideposition={right,top},capbesidewidth=0.5\textwidth}}]{figure}[\FBwidth]
  {\caption{\textbf{Assessment of the segmentation model}: 
  Preliminarily, we investigated the aptness of the segmentation model.
  For five test datasets, we uniformly sampled the space of distortions
    $\hat{U} = \epsilon \cdot U_{true} = \vec{r} + \epsilon \, u_\text{PE}$
    (with $\epsilon \in [-1.1, 1.1]$ and $u_\text{PE}$ from Equation 13 in the
    paper), and evaluated the data-term of the cost function \eqref{eq:energy}.
  The metric consistently displayed its minimum at the ground-truth location ($\epsilon=0.0$)
    for all the cases, indicating that the segmentation model is appropriate for
    registration.}}
  {\includegraphics[width=\linewidth]{Suppl-figure04.pdf}}
  \label{fig:energymap}
\end{figure}

\subsection{Interface and Settings}\label{sec:interface_settings}

\paragraph{Command-line interface}
The command line interface of \emph{regseg} supports general settings and level-wise settings.
For each multi-resolution level, its corresponding settings are added between brackets.

\begin{bashcode}
regseg -F fa.gz adc.nii.gz -P white.vtk pial.vtk -o myprefix [ -a 0.00000 -b 0.00000 --convergence-energy -t 1.0e-06 -w 60 --adaptative-descriptors --grid-spacing 16.0 -i 500 -s 0.001] [ -a 0. -b 0. --convergence-energy -t 1.e-08 -w 5 --grid-spacing 8.0 -i 250 -s 0.01]
\end{bashcode}


It is possible to get the description of available options running
  \codeinline{regseg -h}:

\begin{bashcode}
Usage:

General options:
  -h [ --help ]                         show help message
  -F [ --fixed-images ] arg             fixed image file
  -P [ --surface-priors ] arg           shape priors
  -T [ --surface-target ] arg           final shapes to evaluate metric (only
                                        testing purposes)
  -M [ --fixed-mask ] arg               fixed image mask
  -L [ --transform-levels ] arg         number of multi-resolution levels for
                                        the transform
  -o [ --output-prefix ] arg (=regseg)  prefix for output files
  -l [ --logfile ] arg                  log filename
  -v [ --monitoring-verbosity ] arg (=1)
                                        verbosity level of intermediate results
                                        monitoring ( 0 = no output; 5 = verbose
                                        )

Optimizer options (by levels):
  -a [ --alpha ] arg              alpha value in regularization
  -b [ --beta ] arg               beta value in regularization
  -s [ --step-size ] arg          step-size value in optimization
  -g [ --gradient-scales ] arg    alpha value in regularization
  -r [ --learning-rate ] arg      learning rate to update step size
  -i [ --iterations ] arg         number of iterations
  -w [ --convergence-window ] arg number of iterations of convergence window
  -t [ --convergence-thresh ] arg convergence value
  --grid-size arg                 size of control points grid
  --grid-spacing arg              spacing between control points
  -u [ --update-descriptors ] arg frequency (iterations) to update descriptors
                                  of regions (0=no update)
  --adaptative-descriptors        recomputes descriptors more often at the
                                  beginning of the process
  --convergence-energy            disables lazy convergence tracking: instead
                                  of fast computation of the mean norm of the
                                  displacement field, it computes the full
                                  energy functional

Functional options (by levels):
  --smoothing arg               apply isotropic smoothing filter on target
                                image, with kernel sigma=S mm.
  --smooth-auto                 apply isotropic smoothing filter on target
                                image, with automatic computation of kernel
                                sigma.
  --uniform-bg-membership       consider last ROI as background and do not
                                compute descriptors.
  -d [ --decile-threshold ] arg set (decile) threshold to consider a computed
                                gradient as outlier (ranges 0.0-0.5)

\end{bashcode}

\paragraph{\emph{Nipype} interface}
Our registration algorithm is released with a \emph{nipype Interface} packaged in
  \codeinline{pyacwereg.interfaces.acwereg}.
This interface has been comprehensively used in the evaluation workflows.

\begin{pythoncode}
from pyacwereg.interfaces.acwereg import ACWEReg
regseg = ACWEReg()
regseg.inputs.in_fixed = ['T1w.nii.gz', 'T2w.nii.gz']
regseg.inputs.in_pior = ['csf.vtk', 'white_lh.vtk', 'white_rh.vtk',
                         'pial_lh.vtk', 'pial_rh.vtk']
ifresult = regseg.run()
\end{pythoncode}


\subsection{Convergence evidencing}\label{sec:convergence_evidence}

In order to track the evolution of the registration process, several internal variables
  are saved in the JSON log-file.
Using the JSON log-file as input for the \emph{nipype Interface}
  \codeinline{ACWEReport}, it is straightforward to obtain
  a visual assessment document presenting the convergence.
\begin{pythoncode}
from pyacwereg.interfaces.acwereg import ACWEReport
report = ACWEReport()
report.inputs.in_log = `myprefix.log'
ifresult = report.run()
\end{pythoncode}

Online checking is also possible as the algorithm writes to the standard output as well.
A sample report is found in \autoref{fig:convreport}.

\begin{figure*}[b]
  \includegraphics[width=\linewidth]{Suppl-figure02.pdf}
  \caption{The evolution of the registration and segmentation process can be
    checked using the \emph{Convergence report},
    easily generated using the appropriate \emph{nipype Interface}.
  The report comprehends several plots tracking the evolution of the algorithm and several
    features to help researchers tune up the algorithm in their application.}%
    \label{fig:convreport}
\end{figure*}

\newpage
\section{Instruments for evaluation}

This work is supported by two \emph{nipype Workflows} in order to ensure the reproducibility
  of the results.
All the intermediate results and figures in this paper have been encapsulated into
  the workflows and are available in \citep{esteban_acweregistration_2015}.

An overview of the workflow applied in phantoms is presented in the Figure 2 of the paper.
Subsequently, Figure 3 reproduces the evaluation on real data.
The two figures have been extremely simplified for the best of visualization.
In this section, we review the main elements of the evaluation pipelines.

\subsection{Extraction of structural information}
\begin{figure}[!ht]
  \includegraphics[width=\linewidth]{surfaces.pdf}
  \caption{\textbf{Generating Surfaces}.
  Each surface extracted in the structural reference
    (in both phantoms and real datasets) is obtained with a unitary pipeline,
    for instance, the \emph{PialSurface} block.
  In this figure, the composite workflow to extract the six surfaces that define the
    segmentation model (\autoref{fig:jointplot}) is represented.
  }\label{fig:wf_surfaces}
\end{figure}

\subsection{Processing diffusion MRI data}
\begin{figure}[!ht]
  \floatbox[{\capbeside\thisfloatsetup{capbesideposition={right,top},capbesidewidth=.5\linewidth}}]{figure}[\FBwidth]
  {\includegraphics[width=.9\linewidth]{wf_dti.pdf}}
  {\caption{\textbf{Workflow to generate \gls*{fa} and \gls*{adc} maps from \gls*{dmri} data}.
    We use \emph{MRtrix} \citep{tournier_mrtrix_2012} to generate a \gls*{dti} from data and
      computing the two feature maps of interest (\gls*{fa} and \gls*{adc}).
    This workflow is performed in both undistorted and warped data.}\label{fig:wf_dti}}
\end{figure}

\subsection{T2-weighted registration to \emph{b0} -based (T2B) method}
\begin{figure}[!ht]
  \floatbox[{\capbeside\thisfloatsetup{capbesideposition={right,top},capbesidewidth=0.3\textwidth}}]{figure}[\FBwidth]
  {\includegraphics[width=.85\linewidth]{wf_t2b.pdf}}
  {\caption{\textbf{In-house implementation of the T2B method}.
    Each surface extracted in the structural reference
      (in both phantoms and real datasets) is obtained with a unitary pipeline,
      for instance, the \emph{PialSurface} block.
    In this figure, the composite workflow to extract the six surfaces that define the
      segmentation model (\autoref{fig:jointplot}) is represented.}\label{fig:wf_t2b}}
\end{figure}

\subsection{Complete workflow for evaluation on real data}
\begin{figure}[!ht]
  \includegraphics[width=\linewidth]{hcp_wf.pdf}
  \caption{\textbf{The \emph{nipype} design of the workflow for real data}.}\label{fig:wf_hcp}
\end{figure}

\newpage
\section{Model considerations}

\begin{figure*}[!ht]
	\includegraphics[width=\linewidth]{Suppl-figure01.pdf}
	\caption{Evaluating the joint distribution}\label{fig:jointplot}
\end{figure*}

\newpage
\section{Extended results on real data}

\immediate\write18{./visual_report.sh}
\input{suppl-realdata.tex}

\bibliographystyle{mystyle}
\bibliography{Remote}

\end{document}




%{\color{red} {The main diverging points with respect to
%\citep{gorthi_active_2011} are: 1) there is no need for an explicit level set function
%$\Phi_G$, as we replace the level set gradient computation $N_{\Phi_G}$ with shape
%gradients \citep{besson_dream2s_2003,herbulot_segmentation_2006};
%2) regularization is also based on linear diffusion smoothing \citep{thirion_image_1998},
%but we replace the Gaussian filtering by other constraints
%studied in \citep{nagel_investigation_1986} to better the problem;
%3) optimization is applied in the spectral
%domain, observing anisotropic and inhomogeneous mappings along each direction.
%With respect \citep{guyader_combined_2011}, the main differences are
%the distance function, and the spectral solution to the optimization updates,
%as we shall cover in \autoref{sec:numerical_implementation}.}}



%In this paper we formulate the joint registration-segmentation
%problem as follows.
%such that the known contours in anatomical space $T$ optimally segment
%the diffusion space $D$.
%
%
%Whereas related \glspl*{adf} introduced in \autoref{sec:methods_background}
%make use of explicit level-set formulations to solve \eqref{eq:gradient_descent},
%we alternatively use \emph{shape-gradients}
%\cite{besson_dream2s_2003,herbulot_segmentation_2006}.

%
%Let us denote by $\vec{x}$ the voxel and $F(\vec{x}) = [ f_1, f_2, \ldots, f_N]^T$
%  its associated feature vector in the following.
%
%In our application, these surfaces are precise tissue interfaces of interest extracted
%  from a high-resolution, anatomically correct reference volume using
%  the well-established
%
%In this paper we propose a novel registration framework to simultaneously
%solving the segmentation, distortion and cortical parcellation challenges,
%by exploiting as strong shape-prior the detailed morphology extracted
%from high-resolution and anatomically correct \gls{mri}.
%Indeed, hereafter
%we assume this segmentation problem in anatomical images is reliably and
%accurately solved with readily available tools (e.g.
%\citep{fischl_freesurfer_2012}).
%After global alignment with \gls{t1} using existing approaches, the remaining
%spatial mismatch between anatomical and diffusion space is due to susceptibility
%distortions.
%Finally, we need to establish precise spatial correspondence between the
%surfaces in both spaces, including the tangential direction for parcellation.
%Therefore, we can reduce the problem to finding the differences of spatial
%distortion in between anatomical and \gls{dwi} space.
%We thus reformulate the segmentation problem as an inverse problem, where we
%seek for an underlying deformation field (the distortion) mapping
%from the structural space into the diffusion space, such that the structural
%contours segment optimally the \gls{dwi} data.
%In the process, the one-to-one
%correspondence between the contours in both spaces is guaranteed, and projection
%of parcellisation to \gls{dwi} space is implicit and consistent.
%
%We test our proposed joint segmentation-registration model on two different
%synthetic examples.
%The first example is a scalar sulcus model, where the
%\gls{csf}-\gls{gm} boundary particularly suffers from \gls{pve} and can only be
%segmented correctly thanks to the shape prior and its coupling with the inner,
%\gls{gm}-\gls{wm} boundary through the imposed deformation field regularity.
%The second case deals with more realistic \gls{dwi} data stemming from
%phantom simulations of a simplistic brain data.
%Again, we show that the
%proposed model successfully segments the \gls{dwi} data based on two derived
%scalar features, namely \gls{fa} and \gls{md}, while establishing an estimate
%of the dense distortion field.
%
%The rest of this paper is organized as follows.
%First, in \autoref{sec:methods}
%we introduce our proposed model for joint multivariate segmentation-registration.
%Then we provide a more detailed description of the data and experimental setup in
%\autoref{sec:experiments}.
%We present results in \autoref{sec:results} and conclude
%in \autoref{sec:conclusion}.
%
%
%The properties of the reconstructed tensors and derived scalar maps have
%been studied by \cite{ennis_orthogonal_2006}. Based on their
%findings, \gls{fa}~\eqref{eq:fa} and \gls{md}~\eqref{eq:md} are
%considered complementary features, and therefore we selected them for the
%energy model \eqref{eq:complete_energy} in driving the
%registration-segmentation process.
%Whereas \gls{fa} informs mainly about the \emph{shape} of diffusion,
%the \gls{md} is more related to the \emph{magnitude} of the process:
%
%\begin{align}
%\mathrm{FA} &= \sqrt{ \frac{1}{2}}\,\frac{\sqrt{ (\lambda_1 - \lambda_2)^2 + (\lambda_2 - \lambda_3)^2 + (\lambda_3 - \lambda_1)^2}}{\sqrt{ {\lambda_1}^2 + {\lambda_2}^2 + {\lambda_3}^2}} \label{eq:fa} \\
%\mathrm{MD} &= ( \lambda_1 + \lambda_2 + \lambda_3 ) / 3 \label{eq:md}
%\end{align}
%where $\lambda_i$ are the eigenvalues of the diffusion tensor
%associated with the diffusion signal $S(\vec{q})$. There exist
%two main reasons to justify their choice.
%First, they are well-understood and standardized in clinical routine.
%Second, together they contain most of the information that is
%usually extracted from the \gls{dwi}-derived scalar maps
%\cite{ennis_orthogonal_2006}.
%