% -*- root: 00-main.tex -*-
\section*{Discussion}
\label{sec:discussion}
In section \nameref{sec:methods_background}, we framed our method within the current
  state of art and defined the contributions in our approach.
We chose active contours without edges as we were interested in the homogeneous
  regions of diffusion properties within the brain.
Iteratively, the active contours are updated following the instant shape gradients
  (see \nameref{app:shape_priors} in Appendix) with a gradient descent strategy.
The aproach is valid for image processing applications in which a precise segmentation is
  available (from an atlas or within subject anatomical data), and match any or all
  the following circumstances:
  i) reference and target images do not present appropriate contrasts to
  	perform volume-based registration reliably;
  ii) the low resolution of the target volume produces strong partial volume effects
  	that exclude the use of voxel-wise segmentation algorithms and hinder volume-based
  	registration; and,
  iii) the estimation of the unknown warping is explicitly needed and the
    deformation field is smooth enough to be represented with a sufficient number
  	of B-Spline kernels (i.e. the distortion in \gls*{dmri} images,
  	object tracking in time series, etc.).
To support our claim that images can be simultaneously segmented and registered to a
  reference coordinate system, we conducted evaluation experiments on digital
  simulated phantoms and real datasets.

The algorithm is designed for the proposed application on correcting distortions of
  \gls*{dmri} data, a required preliminary step to extract the structural connectivity
  networks.
Connectome extraction protocols generally include the acquisition of \gls*{t1} images
  to provide with prior information about anatomy with high accuracy.
Our method solves in a single step a joint problem usually addressed in two steps.
The first solves the linear registration of the \gls*{t1} and \gls*{dmri} data
  \citep{greve_accurate_2009}, whereas the second corrects for the nonlinear distortions
  derived from the inhomogenety of magnetic susceptibility across tissues
  \citep{jezzard_correction_1995}.



\paragraph*{Prospects}
Future extensions or improvements of \emph{regseg} similar to the evolution of existing
  alternative methodologies.
Incorporating knowledge about the distortion initializing our method with the theory-based
  correction using fieldmaps would be the most profitable.
The main pitfall using fieldmap correction is the challenging problem of unwrapping the
  range of the phase map, as it is clipped in the range $(-\pi, \pi]$.
The most extended solution to this is \emph{prelude} \citep{jenkinson_fast_2003}, and some
  alternatives have been proposed \citep{daga_susceptibility_2014} to circumvent the
  inherent unreliability of the method.
The \emph{regseg} algorithm would naturally correct missregistered areas due to errors in
  phase unwrapping, and errors derived to the important smoothing processes that are
  typically to clean up field maps.
This integration has been recently applied in the \emph{b0}-to-\gls*{t2} approach
  by \cite{irfanoglu_susceptibility_2011}, studied along with the virtues of using
  nonregular B-Spline grids.
This is also a potential extension to our deformation model.

Finally, we are currently studying substitute features to build the energy functional on,
  enabling to perform \emph{regseg} directly on the raw \gls*{dmri} data.