% -*- root: 00-main.tex -*-
\section*{Discussion}
\label{sec:discussion}
We propose a joint registration and segmentation algorithm designed for biomedical
  image analyses that are affected by all or some of the following circumstances:
  i) reference and target images do not present appropriate contrasts to
  	perform volume-based registration reliably;
  ii) the low resolution of the target volume produces strong partial volume effects
  	that exclude the use of voxel-wise segmentation algorithms and hinder volume-based
  	registration; and,
  iii) the estimation of the unknown warping is explicitly needed and the
    deformation field is smooth enough to be represented with a sufficient number
  	of B-Spline kernels (i.e. the distortion in \gls*{dmri} images,
  	object tracking in time series, etc.).
The registration method performs the segmentation of the target image in homogeneous
  and competing regions defined by explicit active contours.
We chose active contours without edges as we were interested on segmenting homogeneous
  regions of diffusion properties within the brain.
Finally, a B-Spline deformation field supports and keeps track of the evolution of the
  active contours.
Iteratively, the active contours are updated following the instant shape gradients
  \citep{herbulot_segmentation_2006} with a gradient descent strategy.
In section \nameref{sec:methods_background}, we framed our method within the current
  state-of-art and defined the novelty of our approach.

To support our claim that images can be simultaneously segmented and registered to a
  reference coordinate system, we conducted evaluation experiments on digital
  simulated phantoms and real datasets.

The algorithm is designed for the proposed application on correcting distortions of
  \gls*{dmri} data, a required preliminary step to extract the structural connectivity
  networks.
As a consequence, our approach relies on the availability of precise segmentations or
  surfaces of the objects that are to be segmented on the target volume(s).
Connectome extraction protocols generally include the acquisition of \gls*{t1} images
  to provide with prior information about anatomy with high accuracy.
\todo[inline]{Add sentence about comment RW\#1 of IPMI2013: incorporate knowledge of
the EPI distortions into regularization}
Our method solves in a single step a joint problem usually addressed in two steps.
The first solves the linear registration of the \gls*{t1} and \gls*{dmri} data (i.e.
  \cite{greve_accurate_2009}), whereas the second corrects for the nonlinear distortions
  derived from the inhomogenety of magnetic susceptibility across tissues
  \citep{jezzard_correction_1995}.


\paragraph*{Availability and reproducibility statement}
{\color{red} We also considered the reproducibility of results a design requirement.
Therefore, real data are fetched from a publicly available repository
  (the Human Connectome Project \citep{essen_human_2012}) and all the software
  involved in this paper is also publicly released.
\autoref{fig:evworkflows} describes the general structure of the workflow we implemented
  as evaluation instrument, using \emph{nipype} \citep{gorgolewski_nipype_2011} to ensure
  reproducibility.

We implemented the registration method upon the widely used Insight Registration and Segmentation
	Toolkit (\url{http://www.itk.org}) with the aim to release a useful and free software bundle.
The instrumentation framework is also distributed with the registration method,
  and has been implemented using \emph{nipype} \citep{gorgolewski_nipype_2011}.
The pipelines include the automatic generation of the ``cortex'' phantom.
The real datasets are publicly available under the \gls*{hcp} project.
Thus, all the experiments and results presented in this paper are intended to be
  fully reproducible.}