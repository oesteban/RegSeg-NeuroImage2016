% -*- root: 00-main.tex -*-
\section{Discussion}
\label{sec:regseg-discussion}
We present \regseg{}, a simultaneous segmentation and registration method that
  maps a set of nested surfaces into a multivariate target-image.
The nonlinear registration process evolves driven by the fitness of the
  picewise-smooth classification of voxels in the target volume imposed
  by the current mapping of the surfaces.
We propose \regseg{} to incorporate anatomical information extracted from \gls*{t1}
  images into the corresponding \gls*{dmri} of the same subject.
Previously, joint segmentation and registration have been applied successfully to
  other problems such as longitudinal object tracking \citep{paragios_level_2003}
  and atlas-based segmentation \citep{gorthi_active_2011}.
The most common approach to solve the problem simultaneously is optimizing a deformation
  model (registration) that supports the evolution the active contours (segmentation),
  like \cite{paragios_level_2003,yezzi_variational_2003}.
\Regseg{} can be seen as particular case of atlas-based segmentation, replacing the atlas
  by the structural image of the subject.
However, the majority of atlas-based segmentation methods, such as \citep{gorthi_active_2011},
  use a multiphase level set function to register several active contours.
Alternatively, \regseg{} implements the active contours with a hierarchical set of explicit
  surfaces (triangular meshes) that substitute the multiphase level sets, and registration
  is driven by shape-gradients \citep{herbulot_segmentation_2006}.

An important antecedent of \regseg{} is \emph{bbregister} \citep{greve_accurate_2009}.
The tool has been widely adopted as the standard registration method to be used along with the \gls*{epi}
  correction of choice.
It implements a linear mapping, and uses 3D active contours \emph{with edges} to
  search for intensity boundaries in the \lowb{} image.
The active contours are initialized using surfaces extracted from the \gls*{t1} using
  \emph{FreeSurfer} \citep{fischl_freesurfer_2012}: the \emph{pial} surface (exterior of the
  cortical \gls*{gm}) and the \emph{white} surface (the \gls*{wm}/\gls*{gm} interface).
The \lowb{} image only includes a detectable frontier for the pial surface, and thus
  \emph{bbregister} is limited to aligning the cortical layer in this
  application.
To overcome the problem of nonlinear distortions, \emph{bbregister} excludes the
  regions that are typically warped from the boundary search.
Indeed, the distortion must be addressed separately because it is not supported by
  the affine transformation model.
The deformation model of \regseg{} is nonlinear and the surfaces are active
  contours \emph{without edges} \citep{chan_active_2001}.
The contours are without edges because the \gls*{fa} and \gls*{adc} maps do
  not present steep image gradients (edges) but the anatomy can be identified
  by looking for piece-wise smooth homogeneous regions.


Recently, \cite{guyader_combined_2011} proposed a simultaneous segmentation and
  registration method in 2D using level sets and a nonlinear elasticity smoother on the
  displacement vector field, which preserves the topology even with very large deformations.
\Regseg{} includes an anisotropic regularizer for the displacements field proposed by
  \cite{nagel_investigation_1986}.
This regularization strategy conceptually falls in the midway between the Gaussian smoothing
  generally included in most of the existing methodologies, and the complexity of
  the elasticity smoother of \cite{guyader_combined_2011}.
Other minor features that differ from current methods in joint segmentation and registration are
  the support of multivariate target-images and the efficient computation of the shape-gradients
  implemented with sparse matrices.

We verified that precise segmentation and registration of a set of surfaces into multivariate
  data is possible on digital phantoms.
We randomly deformed four different phantom models to mimic three homogeneous regions
  (\gls*{wm}, \gls*{gm}, and \acrlong*{csf}) and we used them to simulate \gls*{t1} 
  and \gls*{t2} images at two resolution levels.
After registration with \regseg{}, we measured the Hausdorff distance between the
  projected contours obtained using the ground-truth warping and our estimates.
We concluded that the errors were significantly lower than the voxel resolution.
We also assessed the 95\% \gls*{ci}, which yielded an aggregate interval of
  0.64--0.66 [mm] for the low resolution phantoms (2.0 mm isotropic voxel) and
  0.34--0.38 [mm] for the high resolution phantoms (1.0 mm isotropic).
Therefore, we also concluded that the error was bounded above by half of the
  voxel spacing.
\newcomment[RV\#1(C.8)]{%}
The distributions of errors along surfaces vary importantly depending on the shape of the
  phantom (see \autoref{fig:regseg-phantom}B).
The misregistration error of the ``gyrus'' phantom showed a much lower spread than that
  for the other shapes.
We argue that the symmetry of those other shapes posed difficulties in driving the contours
  towards the appropriate region and producing some ``sliding'' effect between the
  faces of the surfaces and their ground-truth position.
This effect should not be present in real datasets, thanks to the very convoluted cortical
  layer, and the directional restriction of distortion.}

\todo{real data}
This application is necessary in the extraction of structural connectivity from
  \gls*{dmri}, and has been typically solved in a two-step approach.
First, the \glspl*{dwi} are corrected for \emph{\gls*{epi} distortions} by estimating
  the nonlinear-deformation field from extra MR acquisitions
  \citep{jezzard_correction_1995,chiou_simple_2000,cordes_geometric_2000,
  kybic_unwarping_2000}.
Second, mapping the structural information from the corresponding \gls*{t1} image
  using a linear registration tool like \emph{bbregister} \citep{greve_accurate_2009}.
numerous attempts to delineate the structures of interest in diffusion
  space have been proposed.
Current activity on improving correction methods \citep{irfanoglu_drbuddi_2015} and
  segmenting the \gls*{dmri} in its native space \citep{jeurissen_tissuetype_2015}
  proofs the open interest of the application.
\todo{real data old}
We designed \regseg{} as a method for segmenting \gls*{dmri} data by exploiting the
  anatomical information extracted from a structural image (typically a \gls*{t1} image)
  of the subject.
Current pipelines using whole-brain tractography usually solve this problem with a two-step approach.
First, the images are corrected for nonlinear distortions using auxiliary acquisitions
  such as fieldmaps \citep{jezzard_correction_1995}, \glspl*{dwi} with reversed \gls*{pe}
  blips \citep{chiou_simple_2000}, or \gls*{t2} images \citep{kybic_unwarping_2000}.
Second, the segmentation is projected from a reference \gls*{t1} image using linear
  registration \citep{greve_accurate_2009}.
\Regseg{} addresses this joint problem in a single step and it does not require any additional
  acquisition other than the minimal protocol using only \gls*{t1} and \gls*{dmri} images.
This situation is found commonly in historical datasets.
Moreover, since the structural information is projected into the native space of \gls*{dmri}
  reconstruction and tractography can be performed without resampling data to an undistorted
  space.

We evaluated \regseg{} in a real environment using the experimental framework presented
  in \autoref{fig:regseg-evworkflows}.
We processed 16 subjects from the \gls*{hcp} database using both \regseg{}
  and an in-house replication of the \acrfull*{t2b} method.
\Regseg{} obtained very high accuracy, with an aggregate 95\% \gls*{ci} of 0.56--0.66 [mm], which was
  below the pixel size of 1.25 mm.
The misregistration error that remained after \regseg{} was significantly lower ($p <$ 0.01) than the
  error corresponding to the \gls*{t2b} correction according to Kruskal-Wallis H-tests
  (\autoref{tab:results_real}).
Visual inspections of all the results (\suppl{section S5}) and the violin plots in
  \autoref{fig:regseg-results_real} confirmed that \regseg{} performed better than the \gls*{t2b} method
  in our settings.
We carefully configured the \gls*{t2b} method using the same algorithm and the
  same settings employed in a widely-used tool.
However, cross-comparison experiments are prone to so-called \emph{instrumentation bias}
  \citep{tustison_instrumentation_2013}.
Therefore, these results do not prove that \regseg{} \emph{is better than} \gls*{t2b}.
Our results suggest that \regseg{} is a reliable option in this application field.
In addition, the \gls*{t2b} may introduce an additional (and small) error during the necessary
  registration of \gls*{t2} in the \gls*{t1} space.



\newcomment[RV\#1(C.3)]{%
\paragraph*{Contributions}\label{sec:regseg-related_work} %

}

\newcomment[RV\#1(C.5)]{%
Finally, we also contribute proposing a piecewise-smooth segmentation model defined by
  a selection of nested surfaces to partition the multispectral space
  comprehending the \gls*{fa} and the \gls*{adc} maps and ultimately identify these
  structures in \gls*{dmri} space.}
\newcomment[New]{%
With the definition of appropriate models, \regseg{} may be applied in other fields like
  neonatal brain image segmentation in longitudinal \gls*{mri} studies like \citep{shi_neonatal_2010}
  in which the surfaces obtained in a mature timepoint of the brain can be retrospectively
  propagated to the initial timepoints, regardless the changes of the contrast and spatial
  development between timepoints.}

% \paragraph*{Prospects}
% First extensions of this work will study more appropriate features to build the energy functional
%   on, enabling to perform \regseg{} directly on the raw \gls*{dmri} data.
% A second outlook covers incorporating knowledge about the distortion by initializing our method
%   with the theory-based displacement field that can be estimated with fieldmaps.