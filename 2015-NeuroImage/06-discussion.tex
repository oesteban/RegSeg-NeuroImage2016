% -*- root: 00-main.tex -*-
\section{Discussion}
\label{sec:discussion}

\paragraph*{Accuracy tests}
The hypothesis tested in our study was that reliable image registration can be performed
  by searching for the homogeneous regions defined by a set of nested surfaces in a
  multispectral image
The surfaces correspond to precise contours extracted from another image of the same subject
  (i.e. a \gls*{t1} image, or a different time step).
We demonstrated that active contours without edges can be used successfully to drive a
  deformation supported by B-spline basis functions with digital phantoms.
We randomly deformed four different phantom models to mimic three homogeneous regions
  (\gls*{wm}, \gls*{gm}, and \acrlong*{csf}) and we used them to simulate \gls*{t1} and \gls*{t2}
  images at two resolution levels.
After registration with \regseg{}, we measured the Hausdorff distance between the
  projected contours obtained using the ground-truth warping and our estimates.
We concluded that the errors were significantly lower than the voxel resolution.
We also assessed the 95\% \gls*{ci}, which yielded an aggregate interval of
  0.64--0.66 [mm] for the low resolution phantoms (2.0 mm isotropic voxel) and
  0.34--0.38 [mm] for the high resolution phantoms (1.0 mm isotropic).
Therefore, we also concluded that the error was bounded above by half of the
  voxel spacing.
The distributions of errors along surfaces vary importantly depending on the shape of the
  phantom (see \autoref{fig:phantom}B).
The misregistration error of the ``gyrus'' phantom showed a much lower spread than that
  for the other shapes.
We argue that the symmetry of those other shapes posed difficulties in driving the contours
  towards the appropriate region and producing some ``sliding'' effect between the
  faces of the surfaces and their ground-truth position.
This effect should not be present in real datasets, thanks to the very convoluted cortical
  layer, and the directional restriction of distortion.

\paragraph*{Application to real data}
We designed \regseg{} as a method for segmenting \gls*{dmri} data by exploiting the
  anatomical information extracted from a structural image (typically a \gls*{t1} image)
  of the subject.
Current pipelines using whole-brain tractography usually solve this problem with a two-step approach.
First, the images are corrected for nonlinear distortions using auxiliary acquisitions
  such as fieldmaps \citep{jezzard_correction_1995}, \glspl*{dwi} with reversed \gls*{pe}
  blips \citep{chiou_simple_2000}, or \gls*{t2} images \citep{kybic_unwarping_2000}.
Second, the segmentation is projected from a reference \gls*{t1} image using linear
  registration \citep{greve_accurate_2009}.
\Regseg{} addresses this joint problem in a single step and it does not require any additional
  acquisition other than the minimal protocol using only \gls*{t1} and \gls*{dmri} images.
This situation is found commonly in historical datasets.
Moreover, since the structural information is projected into the native space of \gls*{dmri}
  reconstruction and tractography can be performed without resampling data to an undistorted
  space.

We evaluated \regseg{} in a real environment using the experimental framework presented
  in \autoref{fig:evworkflows}.
We processed 16 subjects from the \gls*{hcp} database using both \regseg{}
  and an in-house replication of the \acrfull*{t2b} method.
\Regseg{} obtained very high accuracy, with an aggregate 95\% \gls*{ci} of 0.56--0.66 [mm], which was
  below the pixel size of 1.25 mm.
The misregistration error that remained after \regseg{} was significantly lower ($p < 0.01$) than the
  error corresponding to the \gls*{t2b} correction according to Kruskal-Wallis H-tests
  (\autoref{tab:results_real}).
Visual inspections of all the results (\suppl{section S5}) and the violin plots in
  \autoref{fig:results_real} confirmed that \regseg{} performed better than the \gls*{t2b} method
  in our settings.
We carefully configured the \gls*{t2b} method using the same algorithm and the
  same settings employed in a widely-used tool.
However, cross-comparison experiments are prone to so-called \emph{instrumentation bias}
  \citep{tustison_instrumentation_2013}.
Therefore, these results do not prove that \regseg{} \emph{is better than} \gls*{t2b}.
Our results suggest that \regseg{} is a reliable option in this application field.
In addition, the \gls*{t2b} may introduce an additional (and small) error during the necessary
  registration of \gls*{t2} in the \gls*{t1} space.

% \paragraph*{Prospects}
% First extensions of this work will study more appropriate features to build the energy functional
%   on, enabling to perform \regseg{} directly on the raw \gls*{dmri} data.
% A second outlook covers incorporating knowledge about the distortion by initializing our method
%   with the theory-based displacement field that can be estimated with fieldmaps.