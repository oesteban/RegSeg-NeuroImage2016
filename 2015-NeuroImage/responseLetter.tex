\documentclass[9pt]{memoir}

\usepackage[numbers,sort&compress]{natbib}
\usepackage{xcolor}
\usepackage[T1]{fontenc}
\usepackage{charter}

\usepackage{xcite}
\externalcitedocument{00-main}

\usepackage[hidelinks]{hyperref}
\usepackage{nameref}

\newcounter{reviewpoint}
\makeatletter
\newenvironment{reviewpoint}%
{\refstepcounter{reviewpoint}\par\medskip\vspace{3ex}\hrule\vspace{1.5ex}\par\noindent%
   {\fontseries{b}\selectfont Comment \arabic{reviewpoint}:}
   \begingroup%
   \color{black!60}
   \fontshape{it}\selectfont %

}
{\endgroup\label{com:\thereviewpoint}\par\medskip}
\def\reviewpointautorefname{Comment}
\makeatother


\newcommand{\reply}{\par\fontshape{n}\selectfont \noindent \textbf{Reply}:\ }

\setlrmarginsandblock{3cm}{2.5cm}{*}
\setulmarginsandblock{2.5cm}{2.5cm}{*}
\checkandfixthelayout

\begin{document}

\section*{Response letter to the manuscript no. NIMG-15-1151}
\medskip
Original title: \emph{Active contours-driven registration method for the structure-informed segmentation of
  diffusion MR images}.

\noindent New title: \emph{\color{red} Segmentation of homogeneous regions of diffusion MR features by mapping
  same-subject surfaces extracted in structural space}.

\bigskip
\noindent We thank the editors and the reviewers for revising our manuscript no. NIMG-15-1151 submitted to NeuroImage.
We appreciate the comments made by both reviewers, and considered them fully to improve our paper.
Please, find in the following our point-by-point answer to the concerns raised by the reviewers,
  along with the description of the corresponding changes in the manuscript.

\bigskip
\bigskip
\subsection*{Reviewer \#1:}
\begin{reviewpoint}
The paper describes an approach for the delineation of selected anatomical structures ([...]) in multi-shell diffusion MR images
  by registering a segmented T1 weighted reference image with fractional anisotropy and apparent diffusion coefficient maps derived
  from the diffusion weighted images.

The approach corresponds fully to the usual registration of an atlas to the image to be segmented, leaving me rather clueless why do
  they call the method ``simultaneous registration and segmentation''.

\end{reviewpoint}
\reply{
The first sentence of the comment is a rather accurate interpretation of our work.
However, it is important to clarify that our method maps anatomically-correct surfaces of the same subject into dMRI space,
  thereby segmenting this target space.
The method works regardless how the surfaces were extracted, or if they were extracted from T2w images or manually delineated.
In this work, the surfaces are extracted from T1w for two reasons: 1) it is the widely-accepted MR sequence for such extraction,
  and 2) the structural information used in reconstruction, tractography and network definition of connectivity extraction pipelines
  is usually defined in T1w-space as well.
We understand that the manuscript was not clear to this respect, and therefore included the following changes:
\begin{enumerate}
\item Changed the title
\item Changed abstract: {\color{red} say what here}
\item Changed introduction: {\color{red} say what here}
\end{enumerate}

Regarding the second sentence, we agree on that our method can be seen as a registration of an atlas to the
  image being segmented.
However, we need to point out some particular features of the method.
First is that surfaces in our method play the role of \emph{strong} priors, since the method does not allow
  for changes in their topology (affecting the segmentation side of the approach) and they need to be
  extremely accurate so that the mapping from structural space is correct (registration).
This might not be the case on atlas-based segmentation, which generally needs to deal with problems by the
  inappropriate topology of surfaces in the atlas, and the result of registration is relatively tolerant
  to the imperfect location of surfaces from the template.
Our method can be seen as a segmentation framework using active contours without edges in which the evolution
  of the contours is governed by an elastic registration method;
or as a nonlinear registration method which cost function is built on top of a segmentation model of active contours
  without edges.
In our opinion, the presented approach is mainly a segmentation method.
However, since the surfaces are supposed to be very accurate and defined in a different coordinate system,
  the method implicitly registers that space to the target image.
Some comments of Reviewer \#2 are very similar to this one (particularly \autoref{com:11}).

In order to clarify this issue, we have modified the manuscript as follows: {\color{red} say what here}.
}

\begin{reviewpoint}
The paper describes an approach [...] (quite some are listed as possible examples but results are only reported for the white matter,
  deep gray matter and the pial surface) in multi-shell [...].
\end{reviewpoint}
\reply{
We have noticed that the way we explained the surfaces necessary to segment real dMRI datasets
  was not appropriate.
On one hand, in this paper we propose a segmentation model to cluster the multivariate space of features
  provided by the FA and ADC map.
As correctly interpreted by reviewer \#2 mentioning it explicitly, and potentially reviewer \#1 noticed,
  the model is given by a set of nested surfaces.
However, it was probably unclear in the manuscript that those surfaces and their nesting is the cornerstone
  of the method, while the set of surfaces selected in validation are only those that will have an important
  impact in downstream processing, such as reconstruction or tractography.

In order to separate the concepts of validation and the definition of the model, we have modified the
  manuscript as follows: {\color{red} say what here}.
}

\begin{reviewpoint}
The proposed algorithm is rather standard, relying on a region-based elastic contour deformation approach.
The cost function is derived in the usual Bayesian framework, relying on a multivariate normal distribution prior for the image
  intensities and a Tikhonov regularizer enforcing a predefined normal distribution over the distortions along the imaging axes.
In this respect the novelty of the paper can be questioned, while the application context and the very promising results may still
  justify a publication in NeuroImage.
\end{reviewpoint}
\reply{
We agree with the reviewer on that the novelty of the methodological elements of our approach is very limited.
However, to our knowledge there exists no method with a paired combination of the following elements:
\begin{itemize}
\item Region based (i.e. without edges) active contours model
\item The use of surfaces as prior knowledge to initialize the segmentation model,
  with property of allowing precise registration if the surface is accurately representative
  of structures being aligned.
  Moreover, the multi-phase definition of levelsets is dual to our surface-nesting model.
\item The anisotropic regularization model of choice, since most of the levelset frameworks
  use Gaussian smoothing of the deformation field.
\item Even though the shape-gradients theory was proposed back in the early 2000's, atlas-based
  registration, and registration-segmentation methods have been generally implemented on levelsets.
\end{itemize}
To our knowledge, the application of a framework as such is new and an important contribution is done
  with the proposed segmentation model of the FA and ADC maps.
We would have appreciated if the reviewers would have provided specific references being close to our approach
  beyond those given in lines 43-57 of the original paper, so that we could point to the differences of our
  method.
In this regard, we have investigated some more potential work related to ours, modifying the manuscript as
  follows: {\color{red} say what here}.
}

\begin{reviewpoint}
However, in such a case the paper should be thoroughly revised to address the following major issues:
1. The assumption of independence between pixels is quite usual but obviously wrong.
The authors may want to comment on the consequences.
\end{reviewpoint}
\reply{
  The reviewer is right, assuming that pixels are i.i.d. is wrong but a widely accepted simplification
that performs correctly with the following details:
\begin{enumerate}
\item Pixels are not generally independent, and this issue is generally addressed imposing prior
  information over neighboring, such as Markov Random Fields.
In our framework, local dependency of pixels is implicitly addressed by the nested surfaces that impose
  the piecewise smoothness of the model.
\item Pixels are not identically distributed, and this issue posed important problems when defining
  our segmentation model.
However, we empirically found that the surfaces described in {\color{red} say line here} and their
  nesting made a good representation of the distribution of each region.
Actually, this reason may be at basis of the poor performance of most of the methods originally referenced
  in lines 19-28.
Our segmentation model ensures that pixels belonging to a certain region are ``almost'' identically distributed.
The bivariate approach using FA and ADC facilitates this, as demonstrated in Figure 4.
\end{enumerate}

Nonetheless, we have specifically introduced this discussion in our manuscript {\color{red} say where}
  and improved Figure 4 to better describe this point.
}

\begin{reviewpoint}
2. It is unclear, under which conditions enforcing the optimality condition guarantees that the region identified corresponds
  to the targeted one (it probably strongly depends on its intensity properties and those of neighbouring structures).
A corresponding analysis of the goal function and the presence of (undesired) local minima is missing.
Figure S1 is not particularly helpful in this respect.
Some information about the initialization of the distortion map should also be given.
\end{reviewpoint}
\reply{

}

\begin{reviewpoint}
3. I miss a reasonable description of the parameter settings necessary for generating the results presented.
It is not even clear if all experiments have been performed by the same parameter settings.
A reasonable sensitivity analysis is also missing.
\end{reviewpoint}
\reply{
We have made a strong effort in providing reproducible experiments using openly available data.
Therefore, parameter settings are publicly available in GitHub.
Particularly, under folder \url{https://github.com/oesteban/RegSeg/tree/master/Scripts/pyacwereg/data}
  there are several configuration files.
Some parameters are also discussed in the Supplemental Materials, Section S2.
However, we understand that a better description of the parameter settings should be included within
  the manuscript and modified it accordingly as follows: {\color{red} say what here}.
}

\begin{reviewpoint}
4. I could not understand how the ground truth can be generated for the real datasets.
As far as I can see, in these cases there is a (probably unknown) distortion between the T1 weighted and the
  diffusion weighted images.
How an emulated distortion can get around this fact?
\end{reviewpoint}
\reply{
As commented in line 160, we are using some ``minimally preprocessed'' images from the database of the
  Human Connectome Project (HCP).
Particularly, the dMRI datasets from these preprocessed images are given corrected for distortions
  and registered in structural space.
Therefore, there could be a residual misregistration between the T1w and the dMRI data, but it should be
  very minimal.
We appreciate the comment, and modified the manuscript to clarify this point {\color{red} say where here}.
We also modified Figure 3 to reflect this aspect.
}

\begin{reviewpoint}
5. The vertex-wise error distributions are not only skewed, but quite clearly structured im many cases
  (especially for the phantom images).
What is the underlying reason?
\end{reviewpoint}
\reply{}

\bigskip
\bigskip
\subsection*{Reviewer \#2:}
\begin{reviewpoint}
This paper seems to present a method that can perform wM-GM segmentation and simultaneously estimate a deformation field to
  correct for EPI distortion.
Apparently, the idea is to use nested WM-GM surfaces, as well as other surfaces, to provide better cortical registration than
  a competing method, which is not clearly described.
The overall goal and motivation of the paper are not clearly explained, and almost all related references for EPI distortion
  correction are missing.
It is unclear if the deformation field solution is restricted to the phase encode direction.
It is possible that this is a very useful method, as the results seem to indicate, but the paper is unfortunately too unclear to
  make a full judgement about the method.
\end{reviewpoint}
\reply{}

\begin{reviewpoint}
abstract: Not immediately clear what this means: ``active contours without edges extracted from structural images''
  (Where do the edges come from? What modalities are actually input to regseg?)
\end{reviewpoint}
\reply{Active contours without edges \citep{chan_active_2001} is a framework for subpixel segmentation of images in
  which the contours evolve enclosing homogeneous regions inside.
Region-based active contours is an alternative terminology.
As previously described in \autoref{com:1} and \autoref{com:2}, we have made an effort to clearly state that the outputs
  to our method are a set of nested surfaces and a target multivariate image, that for the case of dMRI data comprehends
  FA and ADC.

However, and for the sake of clarity, we have modified the manuscript as follows:  {\color{red} say what here}.}

\begin{reviewpoint}
I note that the text in the abstract seems to have no correspondence with the graphical abstract.
The text abstract mentions nothing about diffusion MRI, FA, ADC, or phase maps, so it is unclear how these may be used in the method.
These images figure prominently in the graphical abstract, which may be a better overview than the text abstract.
\end{reviewpoint}
\reply{We agree with the reviewer, we have updated the text abstract, and proposed a completely new graphical abstract.}

\begin{reviewpoint}
Intro line 6-7. I am sure you mean to say that the resolution of dMRI is much LOWER than the images microstructure (not higher).
  larger voxels == lower resolution
\end{reviewpoint}
\reply{Absolutely, we have fixed the manuscript accordingly.}

\begin{reviewpoint}
Intro 9. Orientations is the usual term for diffusion weighting, rather than ``angles''.
\end{reviewpoint}
\reply{We agreed, and updated the manuscript accordingly.}

\begin{reviewpoint}
``These limitations prevent segmentation in the native dMRI space''  I disagree.
In fact, many methods have been used to segment fiber tracts, then to quantitatively measure them,
  in the dMRI space.
It is true the correspondence with T1 is imperfect, but segmentation of fiber tracts is possible.
\end{reviewpoint}
\reply{}

\begin{reviewpoint}
Please define what you mean here by segmentation.
It seems there is an assumption that it means specifically segmentation via registration, where the
  structures are defined using T1 images, but that is not stated anywhere.

Now on line 19 it seems the goal of the segmentation may simply be the segmentation of the WM.
So far, I have no idea what the overall goal of the approach is.
\end{reviewpoint}
\reply{}

\begin{reviewpoint}
Line 24. clustering of what?
\end{reviewpoint}
\reply{}

\begin{reviewpoint}
Now in the paragraph beginning on line 29, it seems the segmentation of WM/GM interface is of interest.
Still, I have no idea if this is the goal of the approach, or why.

Line 39. I note that there are many approaches, in addition to nonlinear T2->b0, that have been proposed
  for correction of EPI distortions in dMRI.
If that is the goal of this approach, the entire intro should be rewritten focusing on this related work.
However, I do not know yet what the goal is.

The entire Introduction should be rewritten to improve organization and clearly state the paper's goal much
  earlier (in abstract and introduction).
  Truly related work should be described in more detail, and all of the unrelated references should be removed.

Line 58. It seems the paper goal may be stated here. It is still completely unclear what type of segmentation should be achieved.
\end{reviewpoint}
\reply{}

\begin{reviewpoint}
It seems by line 69 that EPI distortion correction and segmentation by registration are to be performed jointly
  in some fashion.
There are many, many references for distortion correction that are missing from this introduction and that must
  be added.
See for example multiple papers by Carlo Pierpaoli's group, including this most recent:

Irfanoglu, M. Okan, et al. ``DR-BUDDI (Diffeomorphic Registration for Blip-Up blip-Down Diffusion Imaging) method for correcting echo planar imaging distortions.''
  NeuroImage 106 (2015): 284-299.
\end{reviewpoint}
\reply{}

\begin{reviewpoint}
The majority of the EPI distortion correction methods use T2 and B0 due to similar contrast. Why was T1 chosen in this framework?

Suddenly in Section 2 it becomes clear that there are surfaces being registered.
Perhaps this should have been obvious from the title, which mentioned active contours, but before line 72 it was not clearly
  stated that the method was posed as a surface registration problem.
\end{reviewpoint}
\reply{}

\begin{reviewpoint}
Was the warp restricted to the phase encode direction?
If so, the r in eq (1) should be restricted to this direction.
I do not see this mentioned anywhere near the equations in the Methods section.
If the solution found for the deformation field was not restricted to the phase encode direction,
  then this is not a reasonable way to perform EPI distortion correction. Please clarify.

[...]

I still see nothing in the results indicating that the estimated distortions were restricted to the phase encode direction.
If not, the added degrees of freedom might result in improved surface segmentation, but the distortion field would not correspond at
  all to the physical problem under study.
Furthermore, the dMRI gradients would have to be re-oriented independently at each voxel, and this is not currently possible in any
  software I know of.
\end{reviewpoint}
\reply{
The displacements field supporting the deformation model is not restricted to the phase encoding direction by default.
Therefore, equations in Methods section are presented under such conditions.
The proof of concept presented in the first experiment with phantoms are conducted without preferential directions
  for the deformation.
However, the point raised about the distortions derived from susceptibility inhomogeneity in dMRI data is also true.
Therefore, experiments with real data mimicking such distortions are performed with the deformation aligned with the phase encoding
  axis.
We understand that this point should be clearly stated in the manuscript and modified it as follows:
{\color{red} say what here}.
}

\begin{reviewpoint}
The experiments seem reasonable and clearly presented in general. But far too little detail is given on the B0-T2 comparison method.
\end{reviewpoint}
\reply{}

\begin{reviewpoint}
Please define the phrase ``active contours without edges''.
\end{reviewpoint}
\reply{We hope this point is answered in \autoref{com:10}.}

\begin{reviewpoint}
Figure 3: What are the arrows in images 5? DEformations? Indications of important regions?
\end{reviewpoint}
\reply{
The arrows in Figure 3, panel 5, are intended to show regions subjected to large deformations, where the contours
  do not fit data before applying \emph{regseg}.
They also happen to show areas of the forceps minor, one fiber bundle importantly affected
  by the susceptibility distortions.
We have modified the caption of Figure 3 to describe the meaning of the arrows correctly.
}

\bibliographystyle{mystyle}
\bibliography{Remote}

\end{document}
