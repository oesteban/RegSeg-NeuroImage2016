\documentclass[9pt]{memoir}

\usepackage[numbers,sort&compress]{natbib}
\usepackage{xcolor}
\usepackage[T1]{fontenc}
\usepackage{charter}

\usepackage{xcite}
\externalcitedocument{00-main}

\usepackage[colorlinks=true]{hyperref}
\usepackage{nameref}

\usepackage{amsmath}
\usepackage{amssymb}

\newcounter{reviewpoint}
\makeatletter
\newenvironment{reviewpoint}%
{\refstepcounter{reviewpoint}\par\medskip\vspace{3ex}\hrule\vspace{1.5ex}\par\noindent%
   {\fontseries{b}\selectfont Comment \arabic{reviewpoint}:}
   \begingroup%
   \color{black!60}
   \fontshape{it}\selectfont %

}
{\endgroup\label{com:\thereviewpoint}\par\medskip}
\def\reviewpointautorefname{Comment}
\makeatother


\newcommand{\reply}{\par\fontshape{n}\selectfont \noindent \textbf{Reply}:\ }

\setlrmarginsandblock{3cm}{2.5cm}{*}
\setulmarginsandblock{2.5cm}{2.5cm}{*}
\checkandfixthelayout

\begin{document}
\hypersetup{linkcolor=black!60, citecolor=black!60, urlcolor=black!60}

\section*{Response letter to the manuscript no. NIMG-15-1151}
\medskip
Original title: \emph{Active contours-driven registration method for the structure-informed segmentation of
  diffusion MR images}.

\noindent New title: \emph{{\color{blue!70} Surface}-driven registration method for the structure-informed segmentation of
  diffusion MR images}.

\bigskip
\noindent We thank the editors and the reviewers for revising our manuscript no. NIMG-15-1151 submitted to NeuroImage.
We appreciate the comments made by both reviewers, and considered them fully to improve our paper.
Please, find in the following paragraphs our point-by-point answer to the concerns raised by the reviewers,
  along with the description of the corresponding changes in the manuscript.
Text modifications in the manuscript are highlighted with a new font color, and attached notes indicate which reviewer comment
  the corresponding change is addressing to.

\bigskip
\bigskip
\subsection*{Reviewer \#1:}
\begin{reviewpoint}
The paper describes an approach for the delineation of selected anatomical structures ([...]) in multi-shell diffusion MR images
  by registering a segmented T1 weighted reference image with fractional anisotropy and apparent diffusion coefficient maps derived
  from the diffusion weighted images.

The approach corresponds fully to the usual registration of an atlas to the image to be segmented, leaving me rather clueless why do
  they call the method ``simultaneous registration and segmentation''.

\end{reviewpoint}
\reply{%
The description of the method given by both reviewers (see \autoref{com:10}, Reviewer \#2) lead us to think that the paper
  did not properly describe the actual inputs and outputs the proposed methodology.
In general, we have almost-completely reformulated the abstract, introduction and discussion to define this
  point clearly.
Particularly, the following sections have been introduced to give an answer to this particular comment:
\begin{itemize}
\item Lines 45--53 (p. 2).
\item Lines 244--246 (p. 13), here we introduce the previously missing discussion to call the
  method ``simultaneous registration and segmentation''.
\item Lines 250--252 (p. 13), we comment here on the similarity of \emph{regseg} to the atlas-based segmentation
  using joint registration and segmentation methods.
\end{itemize}
}

\begin{reviewpoint}
The paper describes an approach [...] (quite some are listed as possible examples but results are only reported for the white matter,
  deep gray matter and the pial surface) in multi-shell [...].
\end{reviewpoint}
\reply{
To clarify this point, we have stated more clearly that the choice of surfaces to segment the target data is
  crucial (lines 46--47, p.2).
We have better defined the segmentation model that drives the registration process (lines 94--95, p. 5).
We have also improved the former Figure 4 (now it is Figure 2, page 5), and brought it earlier in the paper to
  illustrate earlier the segmentation model.
Since the former visualization of the conditional (region-wise) joint distribution of FA and ADC was a bit clumsy,
  we have extracted the conditional distributions to the right side of each panel.
We expect these modifications make clear the difference between the surfaces involved in driving the registration
  process and the surfaces in which we evaluate the error of the method (lines 217--218, p. 10) are a subset of the
  former.
}

\begin{reviewpoint}
The proposed algorithm is rather standard, relying on a region-based elastic contour deformation approach.
The cost function is derived in the usual Bayesian framework, relying on a multivariate normal distribution prior for the image
  intensities and a Tikhonov regularizer enforcing a predefined normal distribution over the distortions along the imaging axes.
In this respect the novelty of the paper can be questioned, while the application context and the very promising results may still
  justify a publication in NeuroImage.
\end{reviewpoint}
\reply{
We agree with the reviewer on that the novelty of the methodological elements of our approach is limited.
However, to our knowledge there exists no alternative method combining the same elements to segment
  multivariate data.
We have moved the related methods and our methodological contributions from the Introduction to a more appropriate
  place in the Discussion (lines 248--275), additionally satisfying \autoref{com:9} (Reviewer \#2) suggesting to include
  in the introduction only the references related to the focus of the paper.
The particular changes can be listed as follows:
\begin{itemize}
  \item Lines 252--260 (p. 13), we describe the methodological differences w.r.t. \cite{gorthi_active_2011},
    one of the latest atlas-based segmentation methods using simultaneous registration and segmentation.
  \item Lines 261--270 (p. 13), we compare the methodological features of our method and \emph{bbregister},
    an antecedent and standardized tool used on the same application.
  \item Lines 271--273 (p. 13), we introduce the methodological differences with a recent alternative method in
    terms of regularization.
  \item Lines 273--278 (p. 13) mention some other minor methodological contributions.
\end{itemize}
}

\begin{reviewpoint}
However, in such a case the paper should be thoroughly revised to address the following major issues:
1. The assumption of independence between pixels is quite usual but obviously wrong.
The authors may want to comment on the consequences.
\end{reviewpoint}
\reply{
The reviewer is right, assuming that pixels are i.i.d. is wrong but a widely accepted simplification.
We have supported the assumptions inserting the new lines 88--90 (p. 3).}

\begin{reviewpoint}
2. It is unclear, under which conditions enforcing the optimality condition guarantees that the region identified corresponds
  to the targeted one (it probably strongly depends on its intensity properties and those of neighbouring structures).
A corresponding analysis of the goal function and the presence of (undesired) local minima is missing.
Figure S1 is not particularly helpful in this respect.
Some information about the initialization of the distortion map should also be given.
\end{reviewpoint}
\reply{
We have improved former Figure 4 (now, it is Figure 2, page 5) to clarify that the segmentation model is inherently defined by
  the choice of surfaces and their nesting.
This is an actual contribution of the work, as stated in lines 304--307 (p. 14).
A comment on the convergence issues follows in lines 307--311 (p. 14).
Finally, we described the initialization of the surfaces with the insertion of lines 196--198 (p. 10).
}

\begin{reviewpoint}
3. I miss a reasonable description of the parameter settings necessary for generating the results presented.
It is not even clear if all experiments have been performed by the same parameter settings.
A reasonable sensitivity analysis is also missing.
\end{reviewpoint}
\reply{
We have made a strong effort in providing reproducible experiments using openly available data.
Therefore, parameter settings are publicly available in GitHub.
However, we understand that a better description of the parameter settings should be included within
  the manuscript and modified it accordingly, in lines 200--204 (p. 10).
}

\begin{reviewpoint}
4. I could not understand how the ground truth can be generated for the real datasets.
As far as I can see, in these cases there is a (probably unknown) distortion between the T1 weighted and the
  diffusion weighted images.
How an emulated distortion can get around this fact?
\end{reviewpoint}
\reply{
We have clarified this point with the following edits:
\begin{itemize}
  \item Lines 171--173 (p. 8): we indicate that we use the ``minimally preprocessed'' images of the
    HCP database, and thus, these images are acquired with specialized protocols and are
    already corrected for the distortion.
  \item Figure 4 (former fig. 3), page 9, has been modified to clearly indicate this feature.
\end{itemize}
}

\begin{reviewpoint}
5. The vertex-wise error distributions are not only skewed, but quite clearly structured im many cases
  (especially for the phantom images).
What is the underlying reason?
\end{reviewpoint}
\reply{
We have added a discussion on this in the corresponding section, lines 287-293 (p. 14).
Thanks for pointing this systematic error out.
}

\bigskip
\bigskip
\subsection*{Reviewer \#2:}
\begin{reviewpoint}
This paper seems to present a method that can perform wM-GM segmentation and simultaneously estimate a deformation field to
  correct for EPI distortion.
Apparently, the idea is to use nested WM-GM surfaces, as well as other surfaces, to provide better cortical registration than
  a competing method, which is not clearly described.
The overall goal and motivation of the paper are not clearly explained, and almost all related references for EPI distortion
  correction are missing.
It is unclear if the deformation field solution is restricted to the phase encode direction.
It is possible that this is a very useful method, as the results seem to indicate, but the paper is unfortunately too unclear to
  make a full judgement about the method.
\end{reviewpoint}
\reply{
We thank the reviewer for pointing out that the overall goal is not clearly explained.
We have rewritten the abstract, introduction and discussion sections almost fully.
Particular comments included within this general comment are addressed later: \autoref{com:18}
  as regards EPI distortions, and \autoref{com:20} for the phase-encoding direction point.
In order to better focus the application field, we have introduced lines 11--14 (p. 2) and lines
  312--317 (p. 14).
We have also added potential applications in the new lines 331--340 (p. 15) to better illustrate
  the method.
}

\begin{reviewpoint}
abstract: Not immediately clear what this means: ``active contours without edges extracted from structural images''
  (Where do the edges come from? What modalities are actually input to regseg?)
\end{reviewpoint}
\reply{%
In our adaptation of ``active contours without edges'' \citep{chan_active_2001}, the active contours are
  ``extracted from structural images''.
The particular sentence has been removed from the abstract, and we expect that the comprehensive reformulation
  of the descriptive sections of the paper give a successful answer to this issue.
Regarding the actual inputs of \emph{regseg}, we have introduced modifications in the abstract, in lines 45--50 (p. 2), and
  in lines 243--243 (p. 13).
}

\begin{reviewpoint}
I note that the text in the abstract seems to have no correspondence with the graphical abstract.
The text abstract mentions nothing about diffusion MRI, FA, ADC, or phase maps, so it is unclear how these may be used in the method.
These images figure prominently in the graphical abstract, which may be a better overview than the text abstract.
\end{reviewpoint}
\reply{We agree with the reviewer. We have proposed new graphical and text abstracts.}

\begin{reviewpoint}
Intro line 6-7. I am sure you mean to say that the resolution of dMRI is much LOWER than the images microstructure (not higher).
  larger voxels == lower resolution
\end{reviewpoint}
\reply{Absolutely, we have fixed the manuscript accordingly.}

\begin{reviewpoint}
Intro 9. Orientations is the usual term for diffusion weighting, rather than ``angles''.
\end{reviewpoint}
\reply{We agree and have updated the manuscript accordingly.}

\begin{reviewpoint}
``These limitations prevent segmentation in the native dMRI space''  I disagree.
In fact, many methods have been used to segment fiber tracts, then to quantitatively measure them,
  in the dMRI space.
It is true the correspondence with T1 is imperfect, but segmentation of fiber tracts is possible.
\end{reviewpoint}
\reply{
The reviewer is right, we have changed the paragraph about dMRI segmentation (lines 15--27, page 2)
  for improved precision.
We agree with the reviewer that there actually exist many methods to segment fiber tracts.
We have supported the applications of the method more clearly on lines 11--14 (p. 2).
Now, it is clearly described that fiber tracks are not available for the task at hand.

}

\begin{reviewpoint}
Please define what you mean here by segmentation.
It seems there is an assumption that it means specifically segmentation via registration, where the
  structures are defined using T1 images, but that is not stated anywhere.

Now on line 19 it seems the goal of the segmentation may simply be the segmentation of the WM.
So far, I have no idea what the overall goal of the approach is.
\end{reviewpoint}
\reply{
This comment is very related to \autoref{com:14}.
We have modified the abstract and introduction accordingly, especially lines 15--27 (p. 2).
}

\begin{reviewpoint}
Line 24. clustering of what?
\end{reviewpoint}
\reply{
This comment is a consequence of the two previous comments.
The reviewer is right in the sense that we did not clearly state what we are segmenting, and therefore
  the references in line 24 are a bit orphaned.
All methods reviewed in lines 15--27 cluster/segment/classify \emph{voxels} in dMRI space based on
  one or another feature derived from dMRI data (lines 18--20, p. 2).
We have clarified the manuscript according to \autoref{com:14} and \autoref{com:15} and this comment.
}

\begin{reviewpoint}
Now in the paragraph beginning on line 29, it seems the segmentation of WM/GM interface is of interest.
Still, I have no idea if this is the goal of the approach, or why.

Line 39. I note that there are many approaches, in addition to nonlinear T2->b0, that have been proposed
  for correction of EPI distortions in dMRI.
If that is the goal of this approach, the entire intro should be rewritten focusing on this related work.
However, I do not know yet what the goal is.

The entire Introduction should be rewritten to improve organization and clearly state the paper's goal much
  earlier (in abstract and introduction).
Truly related work should be described in more detail, and all of the unrelated references should be removed.

Line 58. It seems the paper goal may be stated here. It is still completely unclear what type of segmentation should be achieved.
\end{reviewpoint}
\reply{
Again, the reviewer finds that the main goal of the paper is not clearly stated.
We understand that modifications derived from the previous comments satisfy the answer to the concern,
  and changes include abstract and introduction.
We better support the application of our method (lines 11--14, p. 2), and we have stated more clearly the importance of the set
  of surfaces being registered to produce a reliable segmentation model of the two diffusion features selected: FA and ADC
  (please, see \autoref{com:2}).
}

\begin{reviewpoint}
It seems by line 69 that EPI distortion correction and segmentation by registration are to be performed jointly
  in some fashion.
There are many, many references for distortion correction that are missing from this introduction and that must
  be added.
See for example multiple papers by Carlo Pierpaoli's group, including this most recent:

Irfanoglu, M. Okan, et al. ``DR-BUDDI (Diffeomorphic Registration for Blip-Up blip-Down Diffusion Imaging) method for correcting echo planar imaging distortions.''
  NeuroImage 106 (2015): 284-299.
\end{reviewpoint}
\reply{
Effectively, the simultaneous registration and segmentation enables \emph{regseg}
  to overcome the nonlinear warping introduced by EPI distortion.
We have introduced the following changes:
\begin{itemize}
  \item Lines 28--36 (p. 2): Included description and references
    \citep{jezzard_correction_1995,chiou_simple_2000,kybic_unwarping_2000,studholme_accurate_2000,cordes_geometric_2000} about EPI distortion.
  \item Lines 36--38 (p. 2): Added the suggested reference \citep{irfanoglu_drbuddi_2015} with other
    two relevant citations \citep{holland_efficient_2010,andersson_comprehensive_2012}.
  \item Lines 319--330 (p. 14): we discuss our method as regards EPI correction.
\end{itemize}
}

\begin{reviewpoint}
The majority of the EPI distortion correction methods use T2 and B0 due to similar contrast. Why was T1 chosen in this framework?

Suddenly in Section 2 it becomes clear that there are surfaces being registered.
Perhaps this should have been obvious from the title, which mentioned active contours, but before line 72 it was not clearly
  stated that the method was posed as a surface registration problem.
\end{reviewpoint}
\reply{
We have updated the title to represent this aspect.
We have also included all previous changes to fulfill this request.
}

\begin{reviewpoint}
Was the warp restricted to the phase encode direction?
If so, the r in eq (1) should be restricted to this direction.
I do not see this mentioned anywhere near the equations in the Methods section.
If the solution found for the deformation field was not restricted to the phase encode direction,
  then this is not a reasonable way to perform EPI distortion correction. Please clarify.

[...]

I still see nothing in the results indicating that the estimated distortions were restricted to the phase encode direction.
If not, the added degrees of freedom might result in improved surface segmentation, but the distortion field would not correspond at
  all to the physical problem under study.
Furthermore, the dMRI gradients would have to be re-oriented independently at each voxel, and this is not currently possible in any
  software I know of.
\end{reviewpoint}
\reply{
The general functionality of \emph{regseg} does not restrict distortion if it is not requested using the
  feature of anisotropic regularization.
For the experiments on real data, the ground-truth deformations and the deformation field estimated by \emph{regseg}
  are restricted to be nonzero only in the phase-encoding direction.
To make this more clear, we have introduced the following edits:
\begin{itemize}
  \item Lines 131--134 (p. 7).
  \item Lines 198--199 (p. 10).
\end{itemize}
}

\begin{reviewpoint}
The experiments seem reasonable and clearly presented in general. But far too little detail is given on the B0-T2 comparison method.
\end{reviewpoint}
\reply{
We thank the reviewer for positively weighing our experiments and results.
As stated in line 211 (p. 10), we use the settings from the \emph{ExploreDTI} software package \citep{leemans_exploredti_2009}, that
  internally uses \emph{elastix} \citep{klein_elastix_2010} in registration.
A footnote in line 213 (p. 10) directly links the settings for this method.
}

\begin{reviewpoint}
Please define the phrase ``active contours without edges''.
\end{reviewpoint}
\reply{
Additionally to the edits in \autoref{com:10}, we have more clearly defined the concept with the discussion
  introduced in lines 267--268 (p. 13).
}

\begin{reviewpoint}
Figure 3: What are the arrows in images 5? DEformations? Indications of important regions?
\end{reviewpoint}
\reply{
We have modified the caption of former Figure 3 (now Fig. 4) to describe the meaning of the arrows correctly.
}

\bibliographystyle{mystyle}
\bibliography{Remote}

\end{document}
