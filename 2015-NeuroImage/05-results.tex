  % -*- root: 00-main.tex -*-
\section{Results}
\label{sec:regseg-results}

\subsection{Verification and validation using digital phantoms}
\label{sec:regseg-results_phantom}

The results summarized in \autoref{fig:regseg-phantom} demonstrated that the accuracy was
  high and below the image resolution.
Panel B on \autoref{fig:regseg-phantom} shows the violin plots for each model type corresponding
  to the two sets of resolutions for the generated phantoms.
In order to relate the average misregistration error to the resolution of the moving image,
  we proceeded as follows.
First, we confirmed that the vertex-wise error distributions were skewed by using Shapiro-Wilk's test of
  normality.
All of the distributions of errors in the tests (four phantom types $\times$ two resolutions) were
  nonnormal with $p<$ 0.001.
Consequently, we used the nonparametric Wilcoxon signed-rank test with the Bonferroni
  correction for multiple comparisons ($N$=150, for each phantom type).
The average errors were significantly lower than the voxel size with $p <$ (0.001 / 150)
  in all tests (four phantom types $\times$ two resolutions).
Statistical tests might not be sufficiently conclusive, so we also computed the confidence intervals,
  as shown in \autoref{tab:ci_phantom}.

\begin{figure*}[!h]
  \centering
  \includestandalone[width=\linewidth]{figure05}
  \caption{A. Visual assessment of the results obtained on the digital phantoms.
    The panel shows three coronal slices (indices indicated on the lower left corner) of each phantom
    volume at low resolution:
    ``gyrus'' (top left), ``L'' (top right), ``ball'' (bottom left),
    and ``box'' at (bottom right).
  The contours recovered after registration are represented in yellow.
  \Regseg{} achieved high accuracy because it determined the almost exact locations of the registered
    contours with respect to their ground truth positions (shown in green).
  The partial volume effect makes segmentation of the sulci a challenging problem with voxel-wise
    clustering methods, but they were successfully segmented with \regseg{}.
  B. Quantitative evaluation: the violin plot shows the variability across experiments
    of the average Hausdorff distance measured in each vertex of the corresponding surface, for the
    low (left) and high (right) resolutions. 
    Error averages were consistently below the size of the voxel.
    }\label{fig:regseg-phantom}
\end{figure*}
\begin{figure*}[!h]
  \centering
  \includestandalone[width=\linewidth]{figure06}
  \caption{A. Example of a visual assessment report, which was automatically generated by the evaluation tool.
    Each view shows one component of the input image (in this case, the \gls*{fa} map), the ground-truth locations
    of the surfaces (green contours), and the resulting surfaces obtained with the test method (yellow contours).
  The first two rows show axial slices for \regseg{} and the \acrfull*{t2b} method, while the last two rows
    show the corresponding sagittal views.
  The coronal view is omitted because it was the least informative due to the directional property
    of the distortions.
  Specific regions where \regseg{} outperformed \gls*{t2b} are enlarged.
  B. Violin plots of the error distributions for each surface across the 16 subjects, which show the 
    voxel size of the \gls*{dmri} images (1.25 mm), thereby supporting the improved results obtained
    by \regseg{} with the proposed settings.
  }\label{fig:regseg-results_real}
\end{figure*}

\subsection{Evaluation using real datasets and cross-comparison}\label{sec:regseg-results_hcp}
Finally, we compared the performance of \regseg{} with that of the standard \gls*{t2b}
  method.
Summary reports for visual assessment of the 16 cases are included in the
  \citetalias[section S5]{esteban_useful_2016}.
In \autoref{fig:regseg-results_real}, box A, the visual report is shown for one subject.
We computed the \gls*{swindex} \eqref{eq:regseg-swindex} of every surface after registration
  using both the \regseg{} and \gls*{t2b} methods.
Finally, to compare the results, we performed Kruskal-Wallis H-tests
  (a nonparametric alternative to ANOVA) on the warping indices for the three surfaces of
  interest selected in \autoref{sec:regseg-experiments_evaluation}
  ($\Gamma_\text{VdGM}$, $\Gamma_\text{WM}$, $\Gamma_\text{pial}$).
All of the statistical tests showed that the error distributions obtained with \regseg{} and
  \gls*{t2b} were significantly different, and the violin plots in box B of
  \autoref{fig:regseg-results_real} demonstrate that the errors were always larger with \gls*{t2b}.
We also show the 95\% CIs of the \gls*{swindex} for these surfaces 
  (\autoref{tab:results_real}).
The aggregate CI for \regseg{} was 0.56--0.66 [mm], whereas the \gls*{t2b} method
  yielded an aggregate CI of 2.05--2.39 [mm].
The results of the statistical tests and the CIs are summarized in \autoref{tab:results_real}.

\begin{table}
\centering
\footnotesize
\tabcolsep=0.1cm
\begin{minipage}{.48\textwidth}
    \vskip-4.7ex
    \begin{tabular}{lccccc}
    Res.   & ``gyrus'' & ``ball''  & ``box''   & ``L''     & Aggreg.    \\\hline
    1.0mm  & 0.18--0.38 & 0.31--0.45 & 0.34--0.42 & 0.34--0.40 & 0.34--0.38  \\
    2.0mm  & 0.59--0.60 & 0.65--0.76 & 0.68--0.71 & 0.67--0.77 & 0.64--0.66  \\
    \hline
    \end{tabular}
    \caption{The distributions of vertex-wise Hausdorff distances between the ground-truth surfaces and their
    corresponding estimates obtained with \regseg{} presented a 95\% CI below the half-voxel size for all of
    the phantom types.
    The CIs were computed by bootstrapping using 10$^\text{4}$ samples, with the median as the location statistic.}\label{tab:ci_phantom}
\end{minipage}
\hfill
\begin{minipage}{.48\textwidth}
    \begin{tabular}{cccccc}
    & & $\Gamma_\text{VdGM}$  & $\Gamma_\text{WM}$ & $\Gamma_\text{pial}$ & Aggreg. \\
    \hline
    \multirow{2}{*}{CI}
       & \regseg{} & 0.50--0.78 & 0.50--0.55 & 0.66--0.73 & 0.56--0.66 \\
       & T2B       & 1.78--2.58 & 1.94--2.36 & 2.16--2.58 & 2.05--2.39 \\
    \hline
    \multirow{2}{*}{H-tests}
       & p-value  & 4.1\e{-6} & 2.3\e{-6} & 2.3\e{-6} & 1.8\e{-16} \\
       & H-stat   & 21.20 & 22.31 & 22.31 & 67.85 \\
    \hline
    \end{tabular}
    \caption{Statistical analysis of results obtained using 16 real datasets from 
    the \gls*{hcp}, which show that \regseg{} performed better than the alternative \acrfull{t2b} method.
    The distribution of the errors computed for the surfaces of interest ($\Gamma_\text{VdGM}$, $\Gamma_\text{WM}$, $\Gamma_\text{pial}$)
      and the aggregate of all surfaces (Aggreg. column) had lower 95\% CIs with \regseg{}.
   The CIs in this table were computed by bootstrapping using the mean as the location statistic and with 10$^\text{4}$ samples.
    The Kruskal-Wallis H-tests indicated that there was a significant difference between the results obtained using \regseg{} and
      the \gls*{t2b} method.
    }\label{tab:results_real}
\end{minipage}
\end{table}