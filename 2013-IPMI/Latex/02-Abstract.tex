
\begin{abstract}
Connectomics is the study of fiber connections in the brain, and in particular variations thereof. High-quality reconstructions of the connectome from diffusion weighted MR acquisitions require a precise delineation of the pial surface and the grey-matter-white-matter interface. Also, to enable significant group studies, these surfaces must be located and parcellated consistently across subjects. Precise segmentation and parcellation are readily available on anatomical images obtained e.g., through T1-weighted MR imaging. Diffusion weighted images, however, typically suffer from drastically lower resolution and more important image distortions. Here we propose a joint segmentation-registration framework by exploiting the detailed and consistent anatomy extracted from anatomical MRI as strong shape-prior for DW segmentation. We formulate an instance of Mumford-Shah-like active contours without edges, where we look for a regular deformation field that optimally maps the shape prior on the multivariate features in diffusion space.
%\dots
\keywords{diffusion weighted imaging, connectomics, echo planar imaging, magnetic resonance, segmentation, registration, distortion correction}
\end{abstract}