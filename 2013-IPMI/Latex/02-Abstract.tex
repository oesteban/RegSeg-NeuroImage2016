
\begin{abstract}
Connectomics is the study of fiber connections in the brain. High-quality reconstructions of the connectome from diffusion weighted MR acquisitions require a precise delineation of pial and grey-matter-white-matter interfaces. Furthermore, to enable significant group studies, these surfaces must be located and parcellated consistently across subjects. Precise segmentation and parcellation are readily available on anatomical images, such as T1-weighted MRI. Diffusion weighted images (DWI), however, typically have drastically lower resolution and exhibit more important distortions. Here, we propose a joint segmentation-registration framework that exploits the detailed and consistent anatomy extracted from anatomical MRI as strong shape-prior for DW segmentation. We use the Mumford-Shah (or active contours without edges) model to look for a regular deformation field that optimally maps the shape prior on the multivariate features in diffusion space. Preliminary results on synthetic images and simluated DWI confirm the effectiveness of our approach.
%\dots
\keywords{diffusion weighted imaging, connectomics, echo planar imaging, magnetic resonance, segmentation, registration, distortion correction}
\end{abstract}