\section{Conclusions and outlook}
\label{sec:conclusion}
%
A novel application for the \gls{acwe} framework is proposed,
with the aim at recovering the displacement field underlying 
the \gls{epi} geometrical distortions. Exploiting the segmentation
properties of the \gls{acwe} and optimizing the displacement
field, we describe a registration-segmentation methodology that
simultaneously segmented and restored the distortion on 
\gls{dwi}-like synthetic data. Visual results and quantitative
results are provided.

We implemented the methodology upon the widely used
Insight Registration and Segmentation 
Toolkit~\footnote{\url{http://www.itk.org}} (ITK)
for its computational benefits, the standardized code, and 
with the aim at making the procedure publicly available 
when ready for sharing with the research community.

Once proven the aptness of the methodology to the application
with simplistic synthetic data, in further studies we will 
cover the actual performance on real images and the benefits 
of overcoming the described challenges (segmentation and 
\gls{epi} distortion correction) in one single step. Additional
research lines regard with the use of more adequate 
optimization schemes and the use of an energy model 
better adapted to the specific nature of the \gls{dwi} data.

We conclude recalling the importance of tackling with
the numerous challenges that exist on the \gls{dwi} data 
processing in order to achieve reliable results on the
whole-brain connectivity analysis.
