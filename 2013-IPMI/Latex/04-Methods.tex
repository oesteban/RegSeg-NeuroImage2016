\section{Methods}
\label{sec:methods}
%
\subsection{Simulated datasets}
%
As suggested in \autoref{sec:introduction}, the general situation 
consists of having reliable segmentations on the \ac{t1} reference
space, obtained with 
\emph{FreeSurfer\footnote{\url{http://surfer.nmr.mgh.harvard.edu/}}}.
Therefore, regarding the proposed solution, we will have a precise
location of the tissue interfaces of interest in a reference space.
On the other hand, we have a \ac{dwi} volume, characterized by its
low resolution (typically around $2.2x2.2x3mm^3$). Depending on the
posterior reconstruction methodology and the angular resolution
intended, the \ac{dwi} raw data has to be processed in order to
extract the information in a manageable manner. Particularly, we
will use the \ac{fa} and \ac{md} maps. Whereas \ac{fa} describes
the \emph{shape} of diffusion, the \ac{md} depicts the
\emph{manitude} of the process. There exist to main reasons to 
justify their choice. First, they are well-understood and
standardized in clinical routine. Second, they are statistically 
orthogonal and together contain most of the information that is
usually extracted from the \ac{dwi}-derived scalar maps. \\
In order that demonstrating the functionality of the proposed
methodology and characterize its possibilities, we developed two
synthetic model, generating the \ac{dwi} data as described in
 \citep{tuch_q-ball_2004}. We selected 30 directions, for being
a very common protocol for \ac{dti} reconstruction. The first model
is a set of spherical shapes representing the different brain
tissues. The second model is based on the BrainWeb dataset. We
reconstructed the \ac{dwi} data with standard to approximate the
environment to the real one at maximum. There is no interest on the
anatomical reference, given that with the models we hold \emph{a priori}
precisely located surface of the interfaces of interest. \\

FIGURE OF THE MODELS AND EXTRACTED FA, MD

\subsection{\acl{acwe}-like variational segmentation model}
%
Let us denote $\{c_i\}_{i=1..N_c}$ the nodes of the \ac{wm}-\ac{gm}
interface, and $\{d_i\}_{i=1..N_d}$ the nodes of the pial surface. 
Within the respective volumes, we have sets of sample points, denoted 
$\{w_j\}_{j=1..N_w}$, $\{g_j\}_{j=1..N_g}$, and $\{o_j\}_{j=1..N_o}$ 
for the white matter, grey matter and CSF, respectively. All those 
are given in high-resolution T1 reference coordinates.\\
%
On the other hand, we have low resolution \ac{dwi}. At each voxel 
we have a sampled ODF (or some extracted features).
Let us denote by $x$ the voxel and $f(x) = [ f_1, f_2, \ldots, f_N]^T(x)$ 
its associated feature vector.
%
The transformation from T1 into DW reference coordinate space is 
achieved through a dense deformation field $u(x)$, such that for example
the nodes of the w/g-interface are located in DW-space as follows:
%
\begin{equation}
c_i' = T\{c_i\} = c_i + u(c_i)
\end{equation}
% 
Since the nodes of the anatomical surfaces might lay off-grid, it is 
required to derive $u(x)$ from a discrete set of parameters $\{u_k\}_{k=1..K}$.
Densification is achieved through a set of associated basis functions 
$\psi_k$ (e.g. rbf, interpolation splines):
%
\begin{equation}
u(x) = \sum_k \psi_k(x) u_k
\end{equation}
%
Consequently, the transformation writes
%
\begin{equation}
c_i' = T\{c_i\} = c_i + u(c_i) = c_i + \sum_k \psi_k(c_i)u_k
\end{equation} 
%
Note that, since $c_i$ remains constant in the DW segmentation process,
the values of $\psi_k(c_i)$ can be precomputed. Also, provided compact 
support of the basis functions, the system remains relatively sparse.
%
Based on the region-wise samples $\{w_j\}$, $\{g_j\}$, and $\{o_j\}$, 
and the current estimate of the distortion $u$, we can compute 
``expected samples'' of the respective regions in DW space, written
$\{w_j'\}$, $\{g_j'\}$, and $\{o_j'\}$. Based on those samples, we may 
now estimate region descriptors of the DW features $f(x)$ of the three 
respective regions in DW space. In the simplest case, estimate the 
parameters $\mu_R$ and $\Sigma_R$, i.e. the regions' mean feature 
vector and covariance matrix, for each region $R\in \{w,g,o\}$.
%
Based on these Gaussian region descriptors, we propose an \ac{acwe}-like, 
piece-wise constant, variational image segmentation model (where the 
unknown is the deformation field)\cite{chan_active_2001}:
\begin{align}
E(u) &= \int_{w'(u)} (f-\mu_w)^T\Sigma_w^{-1}(f-\mu_w) dx\nonumber\\
&\quad +\int_{g'(u)} (f-\mu_g)^T\Sigma_g^{-1}(f-\mu_g) dx\\
&\quad +\int_{o'(u)} (f-\mu_o)^T\Sigma_o^{-1}(f-\mu_o) dx\nonumber
\end{align}
where the integral domains depend on the deformation field u. Note that 
minimizing this energy, $\argmin_u\{E\}$, yields the MAP estimate of a 
piece-wise smooth image model affected by Gaussian additive noise. This 
inverse problem is ill-posed \cite{hadamard_sur_1902,bertero_ill-posed_1988}.
In order to account for deformation field regularity and to render the 
problem well-posed, we include limiting and regularization terms into 
the energy functional 
\cite{tichonov_solution_1963,morozov_linear_1975}:
\begin{align}
E(u) &= \int_{w'(u)} (f-\mu_w)^T\Sigma_w^{-1}(f-\mu_w) dx\nonumber\\
&\quad +\int_{g'(u)} (f-\mu_g)^T\Sigma_g^{-1}(f-\mu_g) dx\\
&\quad +\int_{o'(u)} (f-\mu_o)^T\Sigma_o^{-1}(f-\mu_o) dx\nonumber\\
&\quad +\int u^T A u dx + \int \tr\{(\nabla u^T)^T B (\nabla u^T)\} dx\nonumber
\end{align}
These regularity terms ensure, that the segmenting contours in DW space 
are still close to their native shape in T1. In the simplest case, both 
$A=\alpha$ and $B=\beta$ are constant scalars, and the terms rewrite 
$\alpha \int  \|u\|^2 dx + \beta \int \left( \|\nabla u_x\|^2 +
\|\nabla u_y\|^2 + \|\nabla u_z\|^2\right) dx$, which are familiar. 
Later, $A$ and $B$ might be spatially varying and/or $3\times 3$ 
matrices, therefore allowing to incorporate inhomogeneous and 
anisotropic regularization \cite{nagel_investigation_1986}.
%
At each iteration, we update the distortion along the steepest energy descent:
%
\begin{equation}
\frac{\partial u_k^t}{\partial t} = -\frac{\partial E(u)}{\partial u_k^t}
\end{equation}
%
At this point, we interpret the parameter field $u_k$ to be a 
continuous function $u(x)$, sampled at the locations $x_k$, and 
determine the gradient-descent equation\footnote{The same assumption 
was being made above in the minimization of the regularity terms}:
%
\begin{align}
\frac{\partial u^t}{\partial t} &= - \sum_{i=1}^{N_c} \left[(f(c_i')-\mu_g)^T\Sigma_g^{-1}(f(c_i')-\mu_g) - (f(c_i')-\mu_w)^T\Sigma_w^{-1}(f(c_i')-\mu_w)\right]\psi_{c_i'}(x)N_{c_i'}\nonumber\\
&\quad -\sum_{i=1}^{N_d} \left[(f(d_i')-\mu_o)^T\Sigma_o^{-1}(f(d_i')-\mu_o) - (f(d_i')-\mu_g)^T\Sigma_g^{-1}(f(d_i')-\mu_g)\right]\psi_{d_i'}(x)N_{d_i'}\\
&\quad -\alpha u + \beta\Delta u\nonumber
\end{align}
%
where we have swapped $\psi_k(c_i')$ into $\psi_{c_i'}(x)$.
%
This gradient descent step can be efficiently tackled by discretizing 
the time in a forward Euler scheme, and making the right hand side 
semi-implicit in the regularization terms:
%
\begin{align}
\frac{u^{t+1}-u^t}{\tau} &= - \sum_{i=1}^{N_c} \left[(f(c_i')-\mu_g)^T\Sigma_g^{-1}(f(c_i')-\mu_g) - (f(c_i')-\mu_w)^T\Sigma_w^{-1}(f(c_i')-\mu_w)\right]\psi_{c_i'}(x)N_{c_i'}\nonumber\\
&\quad -\sum_{i=1}^{N_d} \left[(f(d_i')-\mu_o)^T\Sigma_o^{-1}(f(d_i')-\mu_o) - (f(d_i')-\mu_g)^T\Sigma_g^{-1}(f(d_i')-\mu_g)\right]\psi_{d_i'}(x)N_{d_i'}\nonumber\\
&\quad -\alpha u^{t+1} + \beta\Delta u^{t+1}
\end{align}
%
where the data terms remain functions of the current estimate $u^t$, i.e. 
all $c_i' = c_i'(u^t)$ and $d_i' = d_i'(u^t)$. Again, we propose a spectral 
approach to solve this implicit scheme:
%
\begin{equation}
u^{t+1} = \mathcal{F}^{-1}\left\{ \frac{\mathcal{F}\{u^t/\delta - \sum_{i=1}^{N_c}(\ldots) - \sum_{i=1}^{N_d}(\ldots)  \}}{\mathcal{F}\{(1/\delta+\alpha)\mathcal{I}-\beta\Delta\}} \right\}
\end{equation}
%
It is easily verified, that the same update can be obtained by plugging the 
AL-update w.r.t. $u$ into the AL-update w.r.t. $v$, and by identifying 
$r = 1/\delta$ (the only exception is the distortion on which the data-term 
is being evaluated).


\subsection{Experiment}
%
For both models, we created manually a sound distortion visually similar
to real \ac{epi} distortions. We interpolated the distortion to a 
dense deformation field, necessary for warping the raw \ac{dwi} simulated
data. Once the signal was deformed, we proceeded to reconstruct the
\ac{dti} and subsequently obtained the scalars of interest (\ac{fa}, \ac{md}).\\

We evaluate the performance of our methodology to estimate the deformation
field, obtaining a precise segmentation on the diffusion space.
