\section{Methods}
\label{sec:methods}
%
\subsection{\Acrlong{acwe} based segmentation model}
%
Let us denote $\{c_i\}_{i=1..N_c}$ the nodes of a shape prior surface. In
our application, a precise \gls{wm}-\gls{gm} interface extracted from a
high-resolution reference volume. All the formulations can be naturally
extended to include more shape priors. On the other hand, we have a 
number of \gls{dwi}-derived features at each
voxel of the volume. Let us denote by $x$ the voxel and 
$f(x) = [ f_1, f_2, \ldots, f_N]^T(x)$ its associated feature vector.\\
%
The transformation from reference into \gls{dwi} coordinate space is 
achieved through a dense deformation field $u(x)$, such that:
%
\begin{equation}
c_i' = T\{c_i\} = c_i + u(c_i)
\end{equation}
% 
Since the nodes of the anatomical surfaces might lay off-grid, it is 
required to derive $u(x)$ from a discrete set of parameters $\{u_k\}_{k=1..K}$.
Densification is achieved through a set of associated basis functions 
$\Psi_k$ (e.g. rbf, interpolation splines):
%
\begin{equation}
u(x) = \sum_k \Psi_k(x) u_k
\end{equation}
%
Consequently, the transformation writes
%
\begin{equation}
\label{eq:transformation}
c_i' = T\{c_i\} = c_i + u(c_i) = c_i + \sum_k \Psi_k(c_i)u_k
\end{equation} 
%
% Comment: maybe this is not for IPMI 2013.
%Note that, since $c_i$ remains constant in the DW segmentation process,
%the values of $\Psi_k(c_i)$ can be precomputed. Also, provided compact 
%support of the basis functions, the system remains relatively sparse.\\
%
Based on the current estimate of the distortion $u$, we can compute 
``expected samples'' within the shape prior projected into the \gls{dwi}.
Thus, we now estimate region descriptors of the \gls{dwi} features 
$f(x)$ of the regions defined by the priors in \gls{dwi} space.
%
Using Gaussian distributions as region descriptors, we propose an
\gls{acwe}-like, piece-wise constant, variational image segmentation
model (where the unknown is the deformation field)
\cite{chan_active_2001}:
\begin{equation}
\label{eq:gaussian_energy}
E(u)= \sum_{\forall{R}} \int_{\Omega_R} (f-\mu_R)^T\Sigma_R^{-1}(f-\mu_R) dx
\end{equation}
where $R$ indexes the existing regions and the integral domains
depend on the deformation field $u$. Note
that minimizing this energy, $\argmin_u\{E\}$, yields the \gls{map} 
estimate of a piece-wise smooth image model affected by Gaussian 
additive noise. This inverse problem is ill-posed
\cite{bertero_ill-posed_1988,hadamard_sur_1902}.
In order to account for deformation field regularity and to render the 
problem well-posed, we include limiting and regularization terms into 
the energy functional \cite{morozov_linear_1975,tichonov_solution_1963}:
%
\begin{align}
\label{eq:complete_energy}
E(u) &= \sum_{\forall{R}} \lbrace \int_{\Omega_R} (f-\mu_R)^T\Sigma_R^{-1}(f-\mu_R) dx \rbrace \nonumber \\
&\quad + \alpha \int  \|u\|^2 dx + \beta \int \left( \|\nabla u_x\|^2 + \|\nabla u_y\|^2 + \|\nabla u_z\|^2\right) dx
\end{align}
%
These regularity terms ensure that the segmenting contours in 
\gls{dwi} space are still close to their native shape. The model
easily allows to incorporate inhomogeneous and anisotropic 
regularization \cite{nagel_investigation_1986} to better regularize
the \gls{epi} distortion. \\
%

At each iteration, we update the distortion along the steepest 
energy descent. This gradient descent step can be efficiently 
tackled by discretizing the time in a forward Euler scheme, 
and making the right hand side semi-implicit in the 
regularization terms:
%
\begin{align}
\frac{u^{t+1}-u^t}{\tau} &= - \sum_{i=1}^{N_c} \left[ e(f(c_i'))  \hat{n}_{c_i'} \Psi_{c_i}(x) \right] -\alpha u^{t+1} + \beta\Delta u^{t+1}
\end{align}
%
where the data terms remain functions of the current estimate 
$u^t$, thus $c_i' = c_i'(u^t)$. For simplicity on notation, we 
restricted the number of priors to only 1. We also defined 
$e(f(c_i')) = E_{out}(f(c_i')) - E_{in}(f(c_i'))$, 
and $E_R(f) = {(f-\mu_R)^T\Sigma_R^{-1}(f-\mu_R)}$.
We applied a spectral approach to solve this implicit scheme:
%
\begin{equation}
u^{t+1} = \mathcal{F}^{-1}\left\{ \frac{\mathcal{F}\{u^t/\tau
- \sum_{i=1}^{N_c} \left[e(f(c_i')) \hat{n}_{c_i'} \Psi_{c_i}(x) \right]  \}}{\mathcal{F}\{(1/\tau+\alpha)\mathcal{I}-\beta\Delta\}} \right\}
\end{equation}
%
