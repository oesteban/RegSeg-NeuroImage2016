\section{Methods}
\label{sec:methods}
%
\subsection{Simulated datasets}
%
As suggested in \autoref{sec:introduction}, the general situation 
consists of having reliable segmentations on the \ac{t1} reference
space, obtained with 
\emph{FreeSurfer\footnote{\url{http://surfer.nmr.mgh.harvard.edu/}}}.
Therefore, regarding the proposed solution, we will have a precise
location of the tissue interfaces of interest in a reference space.
On the other hand, we have a \ac{dwi} volume, characterized by its
low resolution (typically around $2.2x2.2x3mm^3$). Depending on the
posterior reconstruction methodology and the angular resolution
intended, the \ac{dwi} raw data has to be processed in order to
extract the information in a manageable manner. Particularly, we
will use the \ac{fa} and \ac{md} maps. Whereas \ac{fa} describes
the \emph{shape} of diffusion, the \ac{md} depicts the
\emph{intensity} of the process. There exist to main reasons to 
justify their choice. First, they are well-understood and
standardized in clinical routine. Second, they are statistically 
orthogonal and together contain most of the information that is
usually extracted from the \ac{dwi}-derived scalar maps. \\
In order that demonstrating the functionality of the proposed
methodology and characterize its possibilities, we developed two
synthetic model, generating the \ac{dwi} data as described in
 \citep{tuch_q-ball_2004}. We selected 30 directions, for being
a very common protocol for \ac{dti} reconstruction. The first model
is a set of spherical shapes representing the different brain
tissues. The second model is based on the BrainWeb dataset. We
reconstructed the \ac{dwi} data with standard to approximate the
environment to the real one at maximum. There is no interest on the
anatomical reference, given that with the models we hold \emph{a priori}
precisely located surface of the interfaces of interest. \\

FIGURE OF THE MODELS AND EXTRACTED FA, MD

\subsection{\acl{acwe}-like variational segmentation model}

I will summarize here your mathematical formulation, but with simplifications:
- only gradient descent
- no mention to anisotropic and inhomogeneous regularization.

\subsection{Experiment}
%
For both models, we created manually a sound distortion visually similar
to real \ac{epi} distortions. We interpolated the distortion to a 
dense deformation field, necessary for warping the raw \ac{dwi} simulated
data. Once the signal was deformed, we proceeded to reconstruct the
\ac{dti} and subsequently obtained the scalars of interest (\ac{fa}, \ac{md}).\\

We evaluate the performance of our methodology to estimate the deformation
field, obtaining a precise segmentation on the diffusion space.
