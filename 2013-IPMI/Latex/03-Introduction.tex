\section{Introduction}
\label{sec:introduction}
%
\ac{dwi} is a widely used family of \ac{mr} techniques
\citep{sundgren_diffusion_2004} which recently has accounted for a growing
interest in its application to whole-brain structural connectivity analysis.
This emerging field, coined in 2005 as \emph{\ac{mr} Connectomics}
\citep{hagmann_diffusion_2005,sporns_human_2005}, currently includes a
large amount of imaging techniques for acquisition, processing, and analysis
specifically tuned for \ac{dwi} data.

The whole-brain connectivity analysis has given rise to some challenges
towards reliable structural information
about the neuronal tracts from \ac{dwi} \cite{johansen-berg_using_2009,
jones_white_2012}. The earlier stages of these
processing pipelines generally include two necessary steps, brain tissue
segmentation on the diffusion space and the correction of geometrical
distortions inherent to the imaging techniques \citep{hagmann_mr_2012}.

In this work, we will refer as brain tissue segmentation to the precise
delineation of the \ac{csf}-\ac{gm} and \ac{gm}-\ac{wm} interface surfaces.
This segmentation is an important step on which strongly rely further
tasks. In tractography, a high-standard \ac{wm} mask is required. Otherwise,
there is an important risk for the algorithm to lose fiber bundles. This
requirement is usually solved in practice by plainly thresholding the 
\ac{fa}, a well-known scalar map derived from \ac{dwi} which depicts 
the isotropy of water diffusion inside the brain. 
A precise location of the
\ac{gm}-\ac{wm} surface is also essential in the final steps to
achieve a consistent parcellisation of the cortex to represent the nodes 
of the output network. This parcellisation is generally defined in a 
high-resolution and better understood structural \ac{mri} of the same 
subject (e.g. \ac{t1} and/or \ac{t2} weighted acquisitions). %Conversely, this problem is resolved with non-linear registration of a structural \ac{mri} of the subject to the \ac{dwi} data. 
Even though some efforts have addressed
the study of the robustness of tractography with respect to intra-subject variability
\cite{wakana_reproducibility_2007,heiervang_between_2006}, these results are restricted to some regions of the brain, only. Therefore, extremely robust
and precise segmentation methods are required in the whole-brain application. 

The \ac{dwi} data is usually obtained with \ac{epi}
acquisition techniques, that often suffer from severe distortions due to 
local field inhomogeneities. Generally, it is appreciated in the anterior
part of the brain, along the phase-encoded direction. Some methodologies have
been developed and generically named as \emph{\ac{epi}-unwarp} techniques
\cite{holland_efficient_2010,hsu_correction_2009,jezzard_characterization_2005,
reber_correction_2005}. These methods usually 
require the extra acquisition of the magnitude and phase of
the field (field-mapping), a condition which is not always met. Some other 
methodologies do not make use of the field-mapping, compensating the distortion
with non-linear registration from structural \ac{mri} or other means
\citep{andersson_modeling_2001}. To our knowledge, there exists no study
of the impact of the \ac{epi} distortion on the variability of tractography
results. 

It is easy to see, that the problems of precise segmentation in \ac{dwi}-space and the spatial mapping between these contours and the corresponding surfaces in anatomical images bear important redundancy. Once the spatial relationship between \ac{t1} and \ac{dwi} space is established, the contours which are readily available in \ac{t1} space, can simply be projected on to the \ac{dwi}-data. Conversely, if a precise delineation in \ac{dwi}-space was achieved, the spatial mapping with \ac{t1}-space could be derived from one-to-one correspondences on the contours. However, neither segmentation nor registration can be performed flawlessly, if considered independently. The significant benefits of exploiting the anatomical \ac{mri} when segmenting the \ac{dwi} data have been demonstrated \cite{zollei_improved_2010}, justifying the use
of the shape prior information. 

The use of anatomical templates in medical image segmentation is not exactly new---for an early overview see \citep{McInerney1996} and references therein. More recently, we suggest clustering the diverse methods of template-based segmentation methods into three groups. The first group typically adds a shape prior term to the energy functional of an evolving active contour \citep{Rousson2002,Chen2002,Paragios2003,Vemuri2003a,Yezzi2003a,Gastaud2004,Chan2005,Cremers2006,Bresson2006a,Ayvaci2007,Schmid2008}.
These methods have a more or less explicit description of the expected relative boundary locations of the object to be delineated, and some even model the statistical deviations from this average shape. Closely related to this group are atlas-based segmentation methods \citep{Pohl2005,Pohl2006,Wang2006,Gorthi2009,Gorthi2011}, where the prior imposes consistent voxel-based classification of contiguous regions. Here, the presence of more structures than just the actual \ac{roi} helps aligning the target image with the atlas in a hierarchical fashion. Finally, the third group generalizes the atlas to actual images, and the contour is to segment simultaneously two different target images, related by a spatial transform to be co-estimated \citep{Wyatt2003,Yezzi2003}.

In this paper we propose a novel registration framework to simultaneously
solving the segmentation and distortion challenges, by exploiting as strong 
shape-prior the detailed anatomy extracted from anatomical \ac{mri}. Indeed, in our case we can consider the segmentation in anatomical images as a solved problem. Moreover, the shape prior is of very strong nature, since it is very specific to the particular subject. Also, after global alignment using existing approaches, the remaining spatial deformation between anatomical and diffusion space is due to \ac{mr} distortion. Finally, we need to establish precise spatial correspondence between the surfaces in both spaces, including the tangential direction for parcellation. Therefore, we can reduce the problem to finding the differences of spatial distortion in between anatomical and \ac{dwi} space. We thus reformulate the segmentation problem as an inverse problem, where we seek for an underlying deformation field---the distortion---mapping 
from the structural space into the diffusion space, such that the structural contours segment optimally the \ac{dwi} data. In the process, the one-to-one correspondence between the contours in both spaces is guaranteed, and projection of parcellisation is trivial.

We test our proposed joint segmentation-registration model on two different synthetic examples. The first example is a scalar sulcus model, where the \ac{csf}-\ac{gm} boundary particularly suffers from \ac{pve} and can only be segmented correctly thanks to the shape prior and its coupling with the inner, \ac{gm}-\ac{wm} boundary through the imposed deformation field regularity. The second case deals with more realistic \ac{dwi} data stemming from simulations of a simplistic brain data. Again, we can show that the proposed model successfully segments the \ac{dwi} data based on two derived scalar features, namely \ac{fa} and \ac{md}, while establishing an estimate of the dense distortion field.

The rest of this paper is organized as follows. First, in \autoref{sec:methods} we introduce our proposed model for joint multivariate segmentation-registration. Then we provide a more detailed description of the data and experimental setup in \autoref{sec:experiments}. We present results in \autoref{sec:results} and conclude in \autoref{sec:conclusion}.