\documentclass[letter]{memoir}

\newcommand{\pointRaised}[2]{\medskip \hrule \noindent 
               \textsl{{\fontseries{b}\selectfont #1}: #2}} 
\newcommand{\reply}{\noindent \textbf{Reply}:\ } 


\begin{document}

\section*{Response letter}

The authors would like to thank the reviewers for their time and their valuable comments.

\subsection*{Reviewer 1:}
\pointRaised{Genneral comment}%
{This is an excellent paper, addressing an important topic:
  validation and quantification of the correctness of 
  tractography in the presence of susceptibility artifacts.
Study design, execution and document structure are very good to excellent,
  and this is one of the rare attempts to try to model the correctness of
  connectivity analyses.
Very good for presentation at ISBI.}
\reply{No further action or correction is required.
We really appreciate the comments.}

\subsection*{Reviewer 3:}
\pointRaised{On page 2, line 41-43}{Was normally distributed noise added
in all cases or only for the FMB correction method?}
\reply{We agree on the importance this detail, and modified the
manuscript consequently.
All the images included in the phantom dataset have normally distributed
  noise.
As it was not stated on the submitted manuscript, we have added this point
  in the last sentence of sec. 2.1.:
\emph{T1w, T2w and diffusion weighted images (DWIs) were added normally distributed
  noise up to a signal-to-noise ratio (SNR) of 30dB}.
Therefore, all methods were subjected to variability due to noise, the FMB method
  with the noise added to the field map, the REB method with the noise added to
  each DWI, and finally, the T2B method with the noise added to the T2w and the
  \emph{b0} images.
Finally, it is important to mention that the noise-free realization of the field map
  was the one use to generate the deformation map.}
  
\pointRaised{On page 4, lines 28 + 37}{"with respect to"}
\reply{We corrected the manuscript accordingly in these two appearances
and reviewed comprehensively the paper for other typos.}

\pointRaised{General comment}
{The authors reason that the susceptibility correction method has hardly
  any effect on the obtained tractography but instead depends on the
  reconstruction and the tractography algorithm.
Does this conclusion also hold for the point-spread function based correction
  method, which is not considered in this work?
Might there exist reconstruction and/or (deterministic or probabilistic)
  tractography algorithms that are more sensitive to the susceptibility correction
  method than the reconstruction/tractography algorithm employed in this work?
}
\reply{The reviewer is right about the first point.
The conclusion would also hold for the point-spread function based methods, given
  that these methods perform in a very similar manner to the FMB and FEB
  methods on finding the deformation map.
In this regard, they differ on the input data required to seek for the deformation
  (for FMB a field map, for FEB two opposed gradient \emph{b0} images and for the
  point-spread function based methods a mapping, in this case, of the point-spread function
  itself).
If we assumed that the three methods performed equally well, finding a perfect deformation
  field, then they provide different implementations for interpolation and drop-out correction.
This latter step might be of great impact if the point-spread function method provided
  a correction significantly better than the proposed methods, but of course this is
  independent from the deformation map finding strategy.
Finally, some very recent works are starting to apply informed priors to reconstruction.
This could dramatically help minimize the side effects of EPI distortion, avoiding
  the need of a high-standard correction technique.}
  
  
\subsection*{Reviewer 4:}
\pointRaised{General comment}%
{The paper presents a method for evaluating diffusion MRI
  image distortion correction algorithms.
The paper is technically sound and clearly written, however the
  conference template is not followed to format the paper.
}
\reply{We have removed some elements that were overriding the
  latex template distributed for the conference.
  As the text spanned over the 4-pages limit, we had to
  reformulate some paragraphs with minor changes for briefness.
  We also modified Figure 2 to contain a legend, therefore
  shortening the associated caption.}

\pointRaised{1.}{Headings should be "1","2" instead of "I","II" and
  sub-headings should be "1.1", "2.1" instead of "I-A,
  "II-A".}
\reply{Headings are now compliant.}
\pointRaised{2.}{The reference formatting is also incorrect, please use
  the font size and line spacing as shown in the template.}
\reply{We fixed this formatting, and reduced the number of references.
  We removed references of software tools (ANTs, nipype, FSL, Diffusion Toolkit
  and MRTrix), replacing them by URLs to their websites.
  We also removed reference [1]
  
  D. K. Jones et al., “Twenty-five pitfalls in the analysis of diffusion MRI data,” NMR in Biomed., 23(7):803–820,
  2010,
  
  that was cited to illustrate the complexity and drawbacks of pipelines for connectivity analysis.}
  
\pointRaised{Section II-D, 2nd to last line}{consider replacing
  "inexistent connections" with "nonexistent connections"}
\reply{Now it reads "nonexitent connections"}
\pointRaised{Section III-C, paragragh 3, line 2}{"with respect
  distortion" should be "with respect to distortion".}
\reply{As for those errors pointed out by reviewer \#3, we corrected
this one, and reviewed again the whole paper for grammar and typos.}

\end{document}