\section{Introduction}
\label{sec:introduction}
% The very first letter is a 2 line initial drop letter followed
% by the rest of the first word in caps.
% 
% form to use if the first word consists of a single letter:
% \IEEEPARstart{A}{demo} file is ....
% 
% form to use if you need the single drop letter followed by
% normal text (unknown if ever used by IEEE):
% \IEEEPARstart{A}{}demo file is ....
% 
% Some journals put the first two words in caps:
% \IEEEPARstart{T}{his demo} file is ....
% 
% Here we have the typical use of a "T" for an initial drop letter
% and "HIS" in caps to complete the first word.
%
\IEEEPARstart{D}{iffusion} \gls{mri} is a widely used family
of \gls{mri} techniques \citep{sundgren_diffusion_2004} which recently 
has accounted for a growing interest in its application to structural 
connectivity analysis of the brain. This emerging field exploits
\gls{dwi} data to derive the local axonal structure at each imaging voxel 
\citep{basser_microstructural_2011} and estimate a whole-brain mapping of fiber 
tracts represented by trajectories reconstructed from the local information.
This comprehensive map of neural connections of the brain is called the 
\emph{connectome} \citep{hagmann_diffusion_2005,sporns_human_2005}. The connectome
analysis is a promising tool for neuroscience and clinical applications 
(\textbf{FIXME:} why? or references?).


Early \gls{dwi} research focused mainly on the improvements of imaging 
methodologies better understanding the diffusion effect and improving
image reconstruction methodologies. Currently, the connectome extraction 
and analysis relies on a large amount of sophisticated computational techniques
\citep{daducci_connectome_2012,hagmann_mr_2012} including acquisition,
reconstruction, modeling and model fitting, image processing, fibre tracking,
connectivity mapping, visualization, group studies, and inference. This 
growing complexity has given rise to challenging issues towards reliable 
structural information about the neuronal tracts \cite{johansen-berg_using_2009,
jones_white_2012,soares_hitchhikers_2013}, and statistical analysis 
\citep{meskaldji_comparing_????}. Here,we shall address three tasks included within 
the image processing stage in a unified approach: brain tissue segmentation in 
diffusion space (\autoref{sec:dwi_segmentation}), correction of geometrical 
distortions (\autoref{sec:distortion}), and structural image registration to 
diffusion coordinate space (\autoref{sec:registration}). These tasks are generally
solved independently, or combined in pairs. However, there exist fundamental 
coupling relationships that can be exploited to obtain a simultaneous solution to 
the three problems. This joint approach satisfactory impacts the downstream
outcomes of the whole pipeline with the increase of the internal consistency.

\subsection{\gls{dwi} data overview}
\label{sec:dwi_overview}

\todo[inline]{VERY brief look into signal generation HERE and cite reconstruction methods.}

\gls{dwi} data is strongly affected by \glspl{pve} \citep{alexander_analysis_2001},
which appear when several different tissues, or signal emitters, are present
in the same imaging unit, producing an averaged intensity. The effect is directly
related to the low resolution achievable with \gls{dwi} (typically around 
$2.0\times2.0\times2.0mm^3$). An additional complication specific to
\gls{dwi} is the \gls{csf} \emph{contamination} \citep{metzler-baddeley_how_2012},
that is a particular \gls{pve} in which the signal sensed inside the affected voxel is 
linear with respect the \gls{gm} and \gls{wm} contributions, but non-linear with
respect \gls{csf}.

Generally speaking, alongside the difficulty posed by the low resolution, \gls{dwi} 
processing is also challenging due to the \emph{direction dependency} of raw data.
\todo[inline]{Introduce here what are b0, fa, md, and direction dependency problem.}
These low-$b$ (or $b = 0$) volumes are acquired without direction gradient as reference, 
and present a T2-like contrast.



\subsection{\gls{dwi} segmentation}
\label{sec:dwi_segmentation}

A precise delineation of the \gls{csf}, \gls{gm} and \gls{wm} interfacing surfaces
is required with sub-pixel resolution.
The resulting segmentation is necessary to perform the majority of tractography 
algorithms and it is required to filter the resulting tractogram. The \gls{gm}-\gls{wm}
interface is necessary to locate the starting and ending points of the detected
fiber bundles, and \gls{csf}-\gls{wm} surface is critical for pruning spurious 
and discontinued fiber bundles.

A number of methodologies have been proposed for \gls{dwi} segmentation, ranging 
from intensity thresholding to atlas-based segmentation. The first approach is performed 
on the \gls{fa} \citep{ennis_orthogonal_2006}, a well-known scalar map derived from
\gls{dwi} data which depicts the isotropy of water diffusion inside the brain.
Although this methodology was popular among the premier tractography studies,
they were generally limited to certain regions or significant fiber tracts, and thus,
it cannot be applied in whole-brain tractography. Early approaches to \gls{dwi} segmentation 
include level set formulations using scalar maps of direction invariants derived
from the tensor model \citep{zhukov_level_2003}, directly on the diffusion raw data
\citep{rousson_level_2004}, or finding alternative diffusion representations 
\citep{jonasson_representing_2007}. Even though this latter case was restricted to the extraction of
the corpus callosum from a real dataset, the density of the components of the diffusion tensor
are approximated by multivariate Gaussians for first. Iterative clustering performed on the 
\emph{low}-$b$ volumes of \gls{dwi} data was proposed by \citep{hadjiprocopis_unbiased_2005}.
Later studies investigated the application of probabilistic frameworks combining mixtures of 
gaussians, \gls{mrf} and labeling fusion techniques \citep{liu_brain_2007} using as features 
widely-used \gls{dwi}-derived scalar maps as \gls{fa} or \gls{md}. A similar framework 
combining co-registered structural information (T1 weighted) with \emph{independent orthogonal 
invariants} derived from the \gls{dwi} tensor model was proposed by \citep{awate_multivariate_2008}. 
Some proposals suggest the use of the raw diffusion data (directionally dependent) to avoid fitting 
a certain model \citep{lu_segmentation_2008}. In \citep{han_experimental_2009}, graph-cuts 
voxel-based techniques are proposed using the most common diffusion tensor derived features.
Further developments of the probabilistic approach have been proposed adding more scalar maps
as features and a more detailed treatment of \gls{pve} \citep{kumazawa_partial_2010}.

A number of methods have been proposed using features not directly derived from \gls{dwi} data.
Segmentation obtained by co-registering structural T1-weighted images will be covered in 
\autoref{sec:registration}. Some other works delay the segmentation task after the tractogram 
is obtained, performing clustering on features derived on the tracts alignment 
\citep{jonasson_white_2005}, combined tract registration \citep{mayer_supervised_2011} or using
tractography atlases \citep{odonnell_automatic_2007}. However, methods based on tractography
usually address the tractogram segmentation problem, to later combine the solution to 
answer the whole-brain segmentation problem.

None of the presented methods have claimed for definite results, mainly due to the lack of 
a \emph{gold-standard} evaluation methodology. Most of them are tested only on certain
regions \todo{citations}, or do not provide sub-pixel resolution results
\citep{hadjiprocopis_unbiased_2005,liu_brain_2007,awate_multivariate_2008,lu_segmentation_2008,
han_experimental_2009}. Generally, results obtained with high resolution atlas co-registration
(\autoref{sec:registration}) are more compelling, minimizing the activity on this line which is 
currently being considered to be included in reconstruction algorithms 
\citep{kumazawa_improvement_2013}. \emph{Golden}-standard evaluation frameworks have been 
proposed for the segmentation validation \citep{jha_task-based_2012} on the task of lesion 
detection in visceral organs.

\subsection{Correction for susceptibility distortions}
\label{sec:distortion}

\gls{dwi} data are usually acquired with \gls{epi} sequences 
as they allow for very fast acquisitions, but they are known to 
suffer from geometrical distortions and artifacts due to, mainly,
three sources: the subject motion between direction sampling, the 
induced \emph{Eddy currents} on the scanner coils and finally
distortions caused by the magnetic susceptibility inhomogeneity
present at tissue interface. In this paper, we restrict ourselves to
the last one, as it accounts for the major impact in the connectome
analysis. \emph{Susceptibility distortions} happen along the 
phase-encoding direction, and are most appreciable in the front part of 
the brain for the strong air/tissue interface surrounding the frontal sinuses.

One approach to correcting the susceptibility distortions was proposed
with the earliest \gls{epi} sequences used in functional \gls{mri}, and
relies on the acquisition of extra \gls{mri} data. Generally, a \gls{gre} is used to 
obtain a map of magnitude and phase of the actual magnetic field inside the
scanner. Based on this \emph{fieldmap}, a number of methodologies have 
been developed to correct for the distortion, and are generically named as 
\emph{\gls{epi}-unwarp} techniques \cite{holland_efficient_2010,
hsu_correction_2009,jezzard_characterization_2005, reber_correction_2005}. 
Unfortunately, the availability of the corresponding fieldmap is not always
met.\todo[inline]{Mention other problems: low precission, unreliability}

Some other methodologies do not make use of the fieldmaps, compensating the 
distortion with non-linear registration from structural \gls{mri} 
\citep{kybic_unwarping_2000} (see \autoref{sec:registration}), or other means 
\citep{andersson_modeling_2001}. To our knowledge, there exists no study on 
the impact of the \gls{epi} distortion on the variability of tractography 
results. 


\subsection{Structural information co-registration}
\label{sec:registration}

Therefore, the problems of precise segmentation in \gls{dwi}-space and the 
spatial mapping between these contours and the corresponding surfaces in 
anatomical images bear significant redundancy. Once the spatial relationship 
between \gls{t1} and \gls{dwi} space is established, the contours which are 
readily available in \gls{t1} space can simply be projected on to the 
\gls{dwi}-data. Conversely, if a precise delineation in \gls{dwi}-space 
was achieved, the spatial mapping with \gls{t1}-space could be derived 
from one-to-one correspondences on the contours. However, neither segmentation 
nor registration can be performed flawlessly, if considered independently. 
The significant benefits of exploiting the anatomical \gls{mri} when 
segmenting the \gls{dwi} data have been demonstrated \cite{zollei_improved_2010}, 
justifying the use of the shape prior information. 

\subsection{Summary}
\label{sec:contributions}

In this paper we propose a novel registration framework to simultaneously
solving the segmentation and distortion challenges, by exploiting as strong 
shape-prior the detailed morphology extracted from high-resolution anatomical 
\gls{mri}. Indeed, hereafter we assume this segmentation problem in anatomical 
images as a solved by widely used available procedures.
Moreover, the shape prior is of very ``strong'' nature, since it is specific to 
the particular subject. Also, after global alignment using existing approaches, 
the remaining spatial deformation between anatomical and diffusion space is 
due to \gls{epi} distortions. Finally, we need to establish precise spatial 
correspondence between the surfaces in both spaces, including the tangential 
direction for parcellation. Therefore, we can reduce the problem to finding 
the differences of spatial distortion in between anatomical and \gls{dw} space.
We thus reformulate the segmentation problem as an inverse problem, where we 
seek for an underlying deformation field (the distortion) mapping 
from the structural space into the diffusion space, such that the structural 
contours segment optimally the \gls{dwi} data. In the process, the one-to-one 
correspondence between the contours in both spaces is guaranteed, and projection 
of parcellisation to \gls{dw} space is implicit and consistent.

We test our proposed joint segmentation-registration model on two different 
synthetic examples. The first example is a scalar sulcus model, where the 
\gls{csf}-\gls{gm} boundary particularly suffers from \gls{pve} and can only be 
segmented correctly thanks to the shape prior and its coupling with the inner, 
\gls{gm}-\gls{wm} boundary through the imposed deformation field regularity. 
The second case deals with more realistic \gls{dwi} data stemming from 
phantom simulations of a simplistic brain data. Again, we show that the 
proposed model successfully segments the \gls{dwi} data based on two derived 
scalar features, namely \gls{fa} and \gls{md}, while establishing an estimate 
of the dense distortion field.

The rest of this paper is organized as follows. First, in \autoref{sec:methods}
we introduce our proposed model for joint multivariate segmentation-registration.
Then we provide a more detailed description of the data and experimental setup in
\autoref{sec:experiments}. We present results in \autoref{sec:results} and conclude 
in \autoref{sec:conclusion}.