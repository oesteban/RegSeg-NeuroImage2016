\section{Introduction}
\label{sec:introduction}

\IEEEPARstart{D}{iffusion} \gls{mri} is a widely used family
of \gls{mri} techniques \citep{sundgren_diffusion_2004} which 
accounts for a growing interest in its application to structural 
connectivity analyses of the brain at the macro-scale
\citep{craddock_imaging_2013}. This emerging field exploits
\gls{dwi} data to derive the local axonal structure at each imaging voxel 
\citep{basser_microstructural_1996} and estimate a whole-brain mapping of fiber 
tracts \citep{hagmann_mr_2012,johansen-berg_using_2009} represented by 
trajectories reconstructed from the local information.
This comprehensive map of neural connections of the brain is called the 
\emph{connectome} \citep{hagmann_diffusion_2005,sporns_human_2005}. The connectome
analysis is a promising tool for neuroscience \citep{morgan_why_2013} and 
for the clinical application \citep{griffa_structural_2013} as well.

% REMOVE paragraph (Dom, 1st comment)
%Early \gls{dwi} research mainly focused on the improvements of imaging 
%methodologies by a better understanding of the diffusion effect, and improving
%image reconstruction methodologies. Currently, the connectome extraction 
%and analysis relies on a large chain of sophisticated computational techniques
%including acquisition, reconstruction, modeling and model fitting, image segmentation
%and registration, fibre tracking, connectivity mapping, visualization, statistical 
%group studies, and inference \citep{daducci_connectome_2012}. This growing complexity 
%has given rise to challenging issues towards reliable structural information about 
%the neuronal tracts presented in, and also difficulties on the statistical analysis 
%\citep{bullmore_complex_2009,meskaldji_comparing_2013}. To our knowledge, studies 
%regarding the impact of the different sources of inaccuracy and variability along
%the toolchain for connectivity analysis \citep{jones_twenty-five_2010,
%cheng_characteristics_2012,jones_white_2013} are scarce.


\subsection{\Gls{dwi} data overview}
\label{sec:dwi_overview}
\gls{dwi} data are usually acquired with \gls{epi} schemes
as they allow for very fast acquisition of large sets of slices each
acquired after a single excitation \citep{mukherjee_diffusion_2008,
soares_hitchhikers_2013}. \Gls{epi} volumes are collected
in sequence and each one represents one probed diffusion direction.
%This first acquisition is called \emph{raw} data or \emph{diffusion
%signal}. One or more volumes without directional encoding (so-called \emph{baseline},
%$b = 0$, \emph{low $b$}, or just \emph{b0} volumes) are interleaved as
%reference, at the least one is acquired at the beginning of the
%\gls{epi} dataset. These B0 are acquired without direction gradient,
%showing a T2-like contrast.

\Gls{epi} are known to suffer from geometrical distortions and artifacts
due to, mainly, three sources: the subject motion in between acquiring 
different sampling directions, the induced \emph{eddy currents} on the scanner 
coils due to the gradient switching \citep{obrien_3d_2013}, and, finally, 
distortions caused by the magnetic susceptibility inhomogeneity
present at tissue interface. \emph{Susceptibility distortions} happen along the 
phase-encoding direction, and are most appreciable in the frontal part of 
the brain due to the air/tissue interface located around the sinuses.
Two implications are associated to this artifact: the signal loss caused on
highly distorted regions, and a significant added difficulty in 
registering with structural images (e.g. T1-weighted).

After correcting for artifacts, the difficulty of performing analyses directly
on the \emph{directionally dependent} raw data is not affordable. Thus, 
\gls{dwi} data is reconstructed using one of a numerous set of readily 
available algorithms. The outcome is a voxel-wise model of diffusion shape 
(\citep{daducci_quantitative_2013} contains a comprehensive review of 
state-of-art methodologies). On top of the
chosen model, some scalar maps describing tissue features are generally
computed (e.g. \gls*{fa}, and \gls*{adc}).
%For instance, the most popular \gls{dwi}-derived scalar map 
%is \gls{fa} which depicts the isotropy of water diffusion inside the brain.
%Another widely studied map is the \gls{adc} (or \gls{md} that has a dual definition),
%which reflects the intensity of the diffusion effect. Many other
%directionally invariant measurements have been defined. For a comprehensive study, 
%we refer to \citep{ennis_orthogonal_2006}.

Alongside distortions and direction dependency, \gls{dwi} data is strongly 
affected by \glspl{pve} \citep{alexander_analysis_2001} that appears when
several different tissues are present in the same imaging unit, producing
an averaged intensity. The effect is directly related to the resolution,
typically around $2.0\times2.0\times2.0mm^3$ for \gls{dwi}.

% the low resolution achievable with \gls{dwi} 
%(typically around ). An additional complication specific to
%\gls{dwi} is the \gls{csf} \emph{contamination} \citep{metzler-baddeley_how_2012},
%that is a particular \gls{pve} in which the signal sensed inside the affected voxel is 
%linear with respect the \gls{gm} and \gls{wm} contributions, but non-linear with
%respect \gls{csf}.

\subsection{Correction for susceptibility distortions}
\label{sec:distortion}

One approach to correcting the susceptibility distortions was proposed
with the earliest \gls{epi} sequences used in functional \gls{mri}, and
relies on the acquisition of extra \gls{mri} data. Generally, a \gls{gre} is used to 
obtain a map of magnitude and phase of the actual magnetic field inside the
scanner. Based on this \emph{fieldmap} and the theory underlying the 
distortion, \citep{jezzard_correction_1995} proposed a correction 
methodology. A number of forked and improved methodologies have been 
developed to correct for the susceptibility distortion and received the
generic name of ``fieldmap correction'' techniques \cite{hsu_correction_2009,
reber_correction_1998}.

Some other retrospective methodologies do not make use of the fieldmaps, 
as explicitly modeling the distortion \citep{andersson_modeling_2001},
registering with (anatomically correct) T2-weighted \gls{mri} 
\citep{kybic_unwarping_2000,studholme_accurate_2000,li_accounting_2007,
tao_variational_2009}, or acquiring an 
extra B0 image with reversed phase encoding direction 
\citep{andersson_how_2003,holland_efficient_2010}.

A comparison of the diverse correction techniques is found in \citep{wu_comparison_2008}.
This study claims that fieldmap correction methodologies are not entirely accurate
and reliable, even though the method is correct in principle. The conclusion
was later confirmed by \citep{tao_variational_2009}. Additional concerns regarding
fieldmap correction are the requirement of an extra acquisition (that is not always
met or it is impractical), or the accuracy of the measured fieldmap that is sensitive
to various effects (such us respiration, blood flow, etc.). All these factors 
have turned susceptibility distortion in \gls{epi} sequences an active field of 
research for the last 10 years.


\subsection{\Gls{dwi} segmentation}
\label{sec:dwi_segmentation}

A precise delineation of the \gls{csf}, \gls{gm} and \gls{wm} interfacing surfaces
is required with sub-pixel resolution, because whereas tractography algorithms are
very good at estimating the location of bundles in deep \gls{wm}, they are not
god (yet) at identifying where the tracts project into the \gls{gm} 
(limitations on the termination criteria, \citep{jbabdi_tractography:_2011,
craddock_imaging_2013}).
Thus, a \gls{wm} segmentation is required by most of the available tractography 
algorithms to \emph{filter} the resulting tractogram. The \gls{gm}-\gls{wm}
interface is necessary to locate the starting and ending points of the detected
fiber bundles, and \gls{csf}-\gls{wm} surface is critical for pruning spurious 
and discontinued fiber bundles.

A number of methodologies have been proposed for \gls{dwi} segmentation, ranging 
from intensity thresholding to atlas-based segmentation. 
The first approach is performed simply thresholding the \gls{fa} map.
Although this methodology was popular among the pioneer tractography studies,
they were generally limited to certain regions or significant fiber tracts, and thus,
it cannot be applied in whole-brain tractography. Early approaches to \gls{dwi} segmentation 
include level set formulations using scalar maps of direction invariants derived
from the tensor model \citep{zhukov_level_2003}, directly on the diffusion raw data
\citep{rousson_level_2004}, or finding alternative diffusion representations 
\citep{jonasson_representing_2007}. Even though this latter case was restricted to the extraction of
the corpus callosum from a real dataset, the density of the components of the diffusion tensor
are approximated by multivariate Gaussians for first. Iterative clustering performed on the 
B0 volumes of \gls{dwi} data was proposed by \citep{hadjiprocopis_unbiased_2005}.
Later studies investigated the application of probabilistic frameworks combining mixtures of 
gaussians, \gls{mrf} and labeling fusion techniques \citep{liu_brain_2007} using as features 
widely-used \gls{dwi}-derived scalar maps as \gls{fa} or \gls{md}. A similar framework 
combining co-registered structural information (T1 weighted) with \emph{independent orthogonal 
invariants} derived from the \gls{dwi} tensor model was proposed by \citep{awate_multivariate_2008}. 
Some proposals suggest the use of the raw diffusion data (directionally dependent) to avoid fitting 
a certain model \citep{lu_segmentation_2008}. In \citep{han_experimental_2009}, graph-cuts 
voxel-based techniques are proposed using the most common diffusion tensor derived features.
Further developments of the probabilistic approach have been proposed adding more scalar maps
as features and a more detailed treatment of \gls{pve} \citep{kumazawa_partial_2010}.

%A number of methods have been proposed using features not directly derived from \gls{dwi} data.
%Segmentation obtained by co-registering structural T1-weighted images will be covered in 
%\autoref{sec:registration}. 

None of the presented methods have claimed for definite results, mainly due to the lack of 
a \emph{gold-standard} evaluation methodology. Most of them are tested only on certain
regions, or do not provide sub-pixel resolution results
\citep{hadjiprocopis_unbiased_2005,liu_brain_2007,awate_multivariate_2008,lu_segmentation_2008,
han_experimental_2009}. Generally, results obtained with high resolution atlas co-registration
(\autoref{sec:registration}) are more compelling, limiting the activity on \gls{dwi} 
segmentation. However, recently \citep{kumazawa_improvement_2013} proposed to include this
task within reconstruction algorithms. Finally, some \emph{golden}-standard evaluation
frameworks for task-based validation have been proposed for the segmentation results
(e.g. lesion detection in visceral organs \citep{jha_task-based_2012}).
\todo[inline]{reference to a summary table}


\subsection{Structural information co-registration}
\label{sec:registration}

The last addressed task in the image processing stage of the studied 
connectivity analysis pipelines is the structural information co-registration.
The need for this registration step is mainly raised when defining the 
nodes of the connectivity matrix. This definition step is named
\textit{cortical parcellation} as it imposes the regions that will be
considered to cluster the fiber bundles reaching the \gls{wm}-\gls{gm}
surface. The common procedure to accomplish this clustering is using
a predefined parcellation in a high-resolution structural T1-weighted 
atlas (e.g. \citep{greve_accurate_2009}). The second goal that has 
justified the applicability of structural
images co-registration is susceptibility distortion correction
as presented in \autoref{sec:distortion}. In any case, susceptibility 
distortion hinders a rigid-registration solution to this problem.
Additionally, when applying non-linear intensity-based
registration algorithms, other difficulties have to be addressed as
the significant \gls{pve} in the \gls{wm}-\gls{gm} layer, or the
inherent inadequacy of the B0 contrast (the only \textit{directionally
independent} existing in the raw \gls{dwi} data) due to the almost null
difference in intensity between \gls{wm} and \gls{gm} pixels. Early 
registration methods appeared targeting the distortion correction.
They quickly standardized the choice of T2 as anatomically correct source
to be registered against the B0 in \gls{dwi}. B0 images have a very similar
contrast to T2 weighted due to the parallelism of acquisition sequences.
However, parcellation is defined in T1 weighted space, adding an additional 
registration step is generally overlooked by existing literature. Even though this
processing stage has a very low complexity compared to the remaining tasks of 
the pipeline process, it is one additional source of inconsistency
and unreliability.


One of the first proposals is \citep{kybic_unwarping_2000}, where the deformation
field is modeled with B-Splines, the cost function is least squares and optimized
in a multi-resolution gradient descent strategy. Their method is evaluated in 
both synthetic and real 2D images. Similarly,
\citep{studholme_accurate_2000} proposed a spline-based deformation but including
a weighting factor proportional to the Jacobian of the transform to correct the
intensity of the undistorted data. They also use the $\log$ of the signal to
enhance the low-signal regions, and optimize with a gradient descent algorithm
the mutual information of the mapping. This framework set the basis for the following
works. For instance, \citep{wu_comparison_2008} also proposed a B-Spline registration
providing quantitative comparisons with fieldmap correction methods. Recent approaches
take into account the signal loss due to dephasing \citep{li_accounting_2007}, or 
introduce more complex variational frameworks \citep{tao_variational_2009}.

The significant benefits of exploiting the anatomical \gls{mri} when 
segmenting the \gls{dwi} data have been demonstrated \cite{zollei_improved_2010}, 
justifying the use of the shape prior information. To our knowledge, there is
no study simultaneously taking advantage from segmentation or distortion correction
tasks to be applied to the co-registration problem.

\subsection{Proposed model}
\label{sec:contributions}

All the available \gls{dwi}-derived models and scalar maps 
are very sensitive to distortions, \gls{pve}, noise and axonal dispersion 
\citep{jbabdi_tractography:_2011}, what yields numerous challenges in the
processing tasks (artifact correction, segmentation, registration and 
parcellation). Finally, as a result a vast number of challenges and pitfalls 
emerge regarding accuracy (correctness) and precision (reproducibility) of 
tractography-based analyses \citep{jones_white_2013} and connectome
analyses.

Moreover, the problems of precise segmentation in \gls{dwi}-space and the 
spatial mapping between these contours and the corresponding surfaces in 
anatomical images bear significant redundancy. Once the spatial relationship 
between \gls{t1} and \gls{dwi} space is established, the contours which are 
readily available in \gls{t1} space can simply be projected on to the 
\gls{dwi} data. Conversely, if a precise delineation in \gls{dwi} space 
was achieved, the spatial mapping with \gls{t1}-space could be derived 
from one-to-one correspondences on the contours. However, neither segmentation 
nor registration can be performed flawlessly, if considered independently.

% Paragraph taken from first subsection to here
Here, we shall address three tasks included within  the image processing stage in 
a unified approach: brain tissue segmentation in diffusion space 
(\autoref{sec:dwi_segmentation}), 
correction of geometrical distortions (\autoref{sec:distortion}), and structural 
image registration to diffusion coordinate space (\autoref{sec:registration}). 
These tasks are generally
solved independently, or combined in pairs. However, there exist fundamental 
coupling relationships that can be exploited to obtain a simultaneous solution to 
the three problems. This joint approach satisfactory impacts the downstream
outcomes of the whole pipeline with the increase of the internal consistency.

In this paper we propose a novel registration framework to simultaneously
solving the segmentation, distortion and cortical parcellation challenges, 
by exploiting as strong shape-prior the detailed morphology extracted 
from high-resolution and anatomically correct \gls{mri}. Indeed, hereafter 
we assume this segmentation problem in anatomical images is reliably and
accurately solved with readily available tools (e.g. \citep{fischl_freesurfer_2012}).
After global alignment with \gls{t1} using existing approaches, the remaining 
spatial mismatch between anatomical and diffusion space is due to susceptibility
distortions.
Finally, we need to establish precise spatial correspondence between the 
surfaces in both spaces, including the tangential direction for parcellation.
Therefore, we can reduce the problem to finding the differences of spatial 
distortion in between anatomical and \gls{dwi} space.
We thus reformulate the segmentation problem as an inverse problem, where we 
seek for an underlying deformation field (the distortion) mapping 
from the structural space into the diffusion space, such that the structural 
contours segment optimally the \gls{dwi} data. In the process, the one-to-one 
correspondence between the contours in both spaces is guaranteed, and projection 
of parcellisation to \gls{dwi} space is implicit and consistent.

We test our proposed joint segmentation-registration model on two different 
synthetic examples. The first example is a scalar sulcus model, where the 
\gls{csf}-\gls{gm} boundary particularly suffers from \gls{pve} and can only be 
segmented correctly thanks to the shape prior and its coupling with the inner, 
\gls{gm}-\gls{wm} boundary through the imposed deformation field regularity. 
The second case deals with more realistic \gls{dwi} data stemming from 
phantom simulations of a simplistic brain data. Again, we show that the 
proposed model successfully segments the \gls{dwi} data based on two derived 
scalar features, namely \gls{fa} and \gls{md}, while establishing an estimate 
of the dense distortion field.

The rest of this paper is organized as follows. First, in \autoref{sec:methods}
we introduce our proposed model for joint multivariate segmentation-registration.
Then we provide a more detailed description of the data and experimental setup in
\autoref{sec:experiments}. We present results in \autoref{sec:results} and conclude 
in \autoref{sec:conclusion}.