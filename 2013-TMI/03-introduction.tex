\label{sec:introduction}
%
\IEEEPARstart{D}{iffusion} MRI (dMRI) is a widely used family of \gls{mr} techniques
\citep{sundgren_diffusion_2004} which recently has accounted for a growing
interest in its application to whole-brain structural connectivity analysis.
This emerging field, coined in 2005 as \emph{\gls{mr} Connectomics}
\citep{hagmann_diffusion_2005,sporns_human_2005}, currently includes a
large amount of imaging techniques for acquisition, processing, and analysis
specifically tuned for \gls{dwi} data.

The whole-brain connectivity analysis has given rise to some challenges
towards reliable structural information about the neuronal tracts 
from \gls{dwi} \cite{johansen-berg_using_2009,jones_white_2012}. 
Here, we shall address brain tissue segmentation on diffusion space 
and correction of geometrical distortions inherent to the acquisition 
sequence \citep{hagmann_mr_2012}.

In this work, we will refer as brain tissue segmentation to the precise
delineation of the \gls{csf}-\gls{gm} and \gls{gm}-\gls{wm} interface surfaces.
An accurate brain tissue segmentation is required
to filter the fiber bundles obtained with \gls{dwi} tractography. 
This requirement is usually solved in practice by plainly thresholding the 
\gls{fa}, a well-known scalar map derived from \gls{dwi} which depicts 
the isotropy of water diffusion inside the brain. Also,
it is necessary to locate the intersections of fiber bundles and \gls{gm}.
Moreover, a precise location of the \gls{gm}-\gls{wm} surface is also essential 
to finally achieve a consistent parcellisation of the cortex 
to represent the nodes of the output network. This parcellisation 
is generally defined in a high-resolution and better understood structural 
\gls{mri} of the same 
subject (e.g. \gls{t1} and/or \gls{t2} weighted acquisitions). Even though
some efforts have addressed the study of the robustness of tractography with
respect to intra-subject variability \cite{heiervang_between_2006,
wakana_reproducibility_2007}, these results are restricted to certain regions 
of the brain, only. Therefore, robust and precise segmentation methods are 
required in the whole-brain application. The problem is challenging due to
the much lower resolution of \gls{dwi} (typically around 
$2.5\times2.5\times2.5mm^3$) compared to structural \gls{mri}, and the
existence of geometrical distortions.

\gls{dwi} data are usually acquired with \gls{epi} sequences 
as they allow for very fast acquisitions, but they are known to 
suffer from geometrical distortions due to local field inhomogeneities. 
These artifacts happen along the phase-encoding direction, and are most 
appreciable in the front part of the brain for the strong air/tissue 
interface around the frontal sinuses. A number of methodologies have
been developed to correct for the distortion, and are generically named 
as \emph{\gls{epi}-unwarp} techniques
\cite{holland_efficient_2010,hsu_correction_2009,jezzard_characterization_2005,
reber_correction_2005}. These methods usually 
require the extra acquisition of the magnitude and phase of
the field (``field-mapping''), a condition which is not always met. Some other 
methodologies do not make use of the field-mapping, compensating the distortion
with non-linear registration from structural \gls{mri} or other means
\citep{andersson_modeling_2001}. To our knowledge, there exists no study
on the impact of the \gls{epi} distortion on the variability of tractography
results. 

Therefore, the problems of precise segmentation in \gls{dwi}-space and the 
spatial mapping between these contours and the corresponding surfaces in 
anatomical images bear significant redundancy. Once the spatial relationship 
between \gls{t1} and \gls{dwi} space is established, the contours which are 
readily available in \gls{t1} space can simply be projected on to the 
\gls{dwi}-data. Conversely, if a precise delineation in \gls{dwi}-space 
was achieved, the spatial mapping with \gls{t1}-space could be derived 
from one-to-one correspondences on the contours. However, neither segmentation 
nor registration can be performed flawlessly, if considered independently. 
The significant benefits of exploiting the anatomical \gls{mri} when 
segmenting the \gls{dwi} data have been demonstrated \cite{zollei_improved_2010}, 
justifying the use of the shape prior information. 

We suggest clustering the current methodologies of template-based segmentation 
methods into three groups. The first group typically adds a shape prior term to 
the energy functional of an evolving active contour~\citep{Bresson2006a,Chan2005,
Chen2002,Cremers2006,Gastaud2004,Paragios2003,Vemuri2003a,Yezzi2003a}.
These methods generally have a explicit description of the expected relative boundary 
locations of the object to be delineated, and some even model the statistical deviations
from this average shape. Closely related to this group are atlas-based segmentation
methods \citep{Gorthi2011,Gorthi2009,Pohl2005,Pohl2006,Wang2006}, where the prior 
imposes consistent voxel-based classification of contiguous regions. Here, the 
presence of more structures than one unique \gls{roi} helps aligning the target image 
with the atlas in a hierarchical fashion. Finally, the third group generalizes 
the atlas to actual images, and the contour is to segment simultaneously two 
different target images, related by a spatial transform to be co-estimated~\citep{Wyatt2003,
Yezzi2003}.

In this paper we propose a novel registration framework to simultaneously
solving the segmentation and distortion challenges, by exploiting as strong 
shape-prior the detailed morphology extracted from high-resolution anatomical 
\gls{mri}. Indeed, hereafter we assume this segmentation problem in anatomical 
images as a solved by widely used available procedures.
Moreover, the shape prior is of very ``strong'' nature, since it is specific to 
the particular subject. Also, after global alignment using existing approaches, 
the remaining spatial deformation between anatomical and diffusion space is 
due to \gls{epi} distortions. Finally, we need to establish precise spatial 
correspondence between the surfaces in both spaces, including the tangential 
direction for parcellation. Therefore, we can reduce the problem to finding 
the differences of spatial distortion in between anatomical and \gls{dw} space.
We thus reformulate the segmentation problem as an inverse problem, where we 
seek for an underlying deformation field (the distortion) mapping 
from the structural space into the diffusion space, such that the structural 
contours segment optimally the \gls{dwi} data. In the process, the one-to-one 
correspondence between the contours in both spaces is guaranteed, and projection 
of parcellisation to \gls{dw} space is implicit and consistent.

We test our proposed joint segmentation-registration model on two different 
synthetic examples. The first example is a scalar sulcus model, where the 
\gls{csf}-\gls{gm} boundary particularly suffers from \gls{pve} and can only be 
segmented correctly thanks to the shape prior and its coupling with the inner, 
\gls{gm}-\gls{wm} boundary through the imposed deformation field regularity. 
The second case deals with more realistic \gls{dwi} data stemming from 
phantom simulations of a simplistic brain data. Again, we show that the 
proposed model successfully segments the \gls{dwi} data based on two derived 
scalar features, namely \gls{fa} and \gls{md}, while establishing an estimate 
of the dense distortion field.

The rest of this paper is organized as follows. First, in \autoref{sec:methods}
we introduce our proposed model for joint multivariate segmentation-registration.
Then we provide a more detailed description of the data and experimental setup in
\autoref{sec:experiments}. We present results in \autoref{sec:results} and conclude 
in \autoref{sec:conclusion}.