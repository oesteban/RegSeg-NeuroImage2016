% -*- root: 00-main.tex -*-
\section{Introduction}
\label{sec:introduction}

\IEEEPARstart{N}{onlinear} registration of images is a difficult task required
  in an endless number of image analysis applications, producing a vast amount of
  methods \citep{sotiras_deformable_2013} over the past decades.

   especially when the scale of the imaged structures is in the range of the
  image resolution or below.
In the case of the human brain, the cortex has a typical thickness of up to 5 mm
  \citep{fischl_measuring_2000}.
Moreover, as the cortex is densely folded, points located afar in the topological
  manifold of the brain's surface can be very close in the corresponding 3D space.
Consequently, accuracy and precision are very difficult to achieve, but necessary properties
  of registration algorithms.


-> Volume-based registration is a traditional technique

-> We have anatomical surfaces at high resolution that we want to impose to a
lower resolution (typically different sequence), a follow-up image to track the surface
or similar.

-> Similar problem: atlas-based registration.

Atlas-based segmentation Nonlinear registration and  are fundamental techniques applied in
  image processing.

\subsection{Related work}
\label{sec:methods_background}

We suggest clustering the
current methodologies of template-based segmentation methods into three groups.
The first group typically adds a shape prior term to the energy functional of
\citep{chan_level_2005,chen_using_2002,cremers_kernel_2006,gastaud_combining_2004,wyatt_map_2003,paragios_level_2003,yezzi_variational_2003,gorthi_segmentation_2009,pohl_unifying_2005,pohl_bayesian_2006,wang_joint_2006,guyader_combined_2011,besson_dream2s_2003,bertero_illposed_1988,cote_tractometer_2013}
an evolving active contour \citep{bresson_variational_2006, chan_level_2005,
chen_using_2002,cremers_kernel_2006,gastaud_combining_2004}.
These methods generally have an explicit description of the expected relative boundary
locations of the object to be delineated, and some even model the statistical deviations
from this average shape. By including a coordinates mapping, it is possible to perform
active contours based registration between timesteps in a time-series or between different
images \citep{bertalmio_morphing_2000,wyatt_map_2003,paragios_level_2003,vemuri_joint_2003,
yezzi_variational_2003}.
A summary of these second set of techniques is performed in \citep{droske_mumfordshah_2009},
proposing two different approaches to applying the Mumford-Shah \citep{mumford_optimal_1989}
functional in joint registration and segmentation. Finally, a derivation of the
latter groups is composed by atlas-based segmentation methods
\citep{gorthi_segmentation_2009,gorthi_active_2011,pohl_unifying_2005,
pohl_bayesian_2006,wang_joint_2006}, where the prior
imposes consistent voxel-based classification of contiguous regions.
A comprehensive summary of the \gls{acwe}-derived methodologies with special attention
to joint registration-segmentation methods is found in \citep{gorthi_active_2011}.
\Gls{acwe} and level-set based formulations have been widely applied in image
processing, as reviewed in \citep{suri_shape_2002}.

The first joint ``morphing''-segmentation approach (it was not a proper
registration method) was proposed in \citep{bertalmio_morphing_2000}. The first
registration framework is presented by \citep{yezzi_variational_2001} where the
energy functional is defined simultaneously in the moving and target images with
an affine transformation supporting the coordinates mapping (``2 \glspl*{pde} approach'').
\citep{vemuri_joint_2003} proposed the first atlas-based registration based on the
level set framework and using only one \gls*{pde}. \citep{unal_coupled_2005} extended the
idea of \citep{bertalmio_morphing_2000,yezzi_variational_2001} for non-linear
registration with a dense deformation field as mapping function. \citep{droske_mumfordshah_2009}
combining ideas from both branches and proposed the propagation of the
deformation field from the contours to the whole image definition. Finally,
\citep{gorthi_active_2011} present a comprehensive generalization of the methodologies.

Even though the historical evolution
of joint segmentation and registration procedures is strongly linked
to the level sets approximation, our work is based in derivations of
the \gls{acwe} framework \citep{chan_active_2001}, and related to
\citep{guyader_combined_2011}.
{\color{red} {The main diverging points with respect to
\citep{gorthi_active_2011} are: 1) there is no need for an explicit level set function
$\Phi_G$, as we replace the level set gradient computation $N_{\Phi_G}$ with shape
gradients \citep{besson_dream2s_2003,herbulot_segmentation_2006};
2) regularization is also based on linear diffusion smoothing \citep{thirion_image_1998},
but we replace the Gaussian filtering by other constraints
studied in \citep{nagel_investigation_1986} to better the problem;
3) optimization is applied in the spectral
domain, observing anisotropic and inhomogeneous mappings along each direction.
With respect \citep{guyader_combined_2011}, the main differences are
the distance function, and the spectral solution to the optimization updates,
as we shall cover in \autoref{sec:numerical_implementation}.}}

\todo[inline]{The previous paragraphs in red color are marked by Dominique
as ``to rework'', and I totally agree}

\subsection{Our framework}
\label{sec:our_framework}

In this paper we formulate the joint registration-segmentation
problem as follows. We look for a spatial mapping (with, typically,
$n=3$ for the 3-dimensional case):

\begin{align}
U\colon T \subset \mathbb{R}^n &\to D \subset \mathbb{R}^n \notag\\
\vec{x} &\mapsto \vec{x}' =\vec{x}+u(\vec{x}),
\label{eq:transform}
\end{align}
such that the known contours in anatomical space $T$ optimally segment
the diffusion space $D$.


Whereas related \glspl*{adf} introduced in \autoref{sec:methods_background}
make use of explicit level-set formulations to solve \eqref{eq:gradient_descent},
we alternatively use \emph{shape-gradients}
\cite{besson_dream2s_2003,herbulot_segmentation_2006}.
Let $r(\vec{x})$ be an ``arbitrary'' function over the image domain,
and $\Omega$ a certain image region with $\Gamma_{l,m}$ its corresponding
outer boundary as defined in \autoref{sec:methods_map}.
We now derive the domain integral w.r.t. $t$:
\begin{equation*}
\frac{\partial}{\partial t} \int_\Omega r(\vec{x}') d\vec{x} = \int_\Omega \frac{\partial}{\partial t}r(\vec{x}') d\vec{x} - \int_{\Gamma_{l,m}} r(\vec{x}') \left\langle \frac{\partial \Gamma_{l,m} }{\partial t}, N_{\Gamma_{l,m}}\right\rangle d\vec{x}
\end{equation*}
where $\left\langle\frac{\partial\Gamma_{l,m}}{\partial t}, N_{\Gamma_{l,m}}\right\rangle$ is
the projection of the boundary movement on the unit inward normal $N_{\Gamma_{l,m}}$. Assuming
that the region descriptors $\Theta_l$ vary slowly enough, we can consider
that $\frac{\partial}{\partial t} r(\vec{x}') = 0$ and thus:
\begin{equation}
\frac{\partial}{\partial t} \int_\Omega r(\vec{x}') d\vec{x} =
- \int_{\Gamma_{l,m}} r(\vec{x}') \left\langle \frac{\partial \Gamma_{l,m} }{\partial t}, N_{\Gamma_{l,m}}\right\rangle d\vec{x}
\label{eq:shape_gradients}
\end{equation}

Let us denote by $\vec{x}$ the voxel and $F(\vec{x}) = [ f_1, f_2, \ldots, f_N]^T$
  its associated feature vector in the following.

In our application, these surfaces are precise tissue interfaces of interest extracted
  from a high-resolution, anatomically correct reference volume using
  the well-established

In this paper we propose a novel registration framework to simultaneously
solving the segmentation, distortion and cortical parcellation challenges,
by exploiting as strong shape-prior the detailed morphology extracted
from high-resolution and anatomically correct \gls{mri}.
Indeed, hereafter
we assume this segmentation problem in anatomical images is reliably and
accurately solved with readily available tools (e.g.
\citep{fischl_freesurfer_2012}).
After global alignment with \gls{t1} using existing approaches, the remaining
spatial mismatch between anatomical and diffusion space is due to susceptibility
distortions.
Finally, we need to establish precise spatial correspondence between the
surfaces in both spaces, including the tangential direction for parcellation.
Therefore, we can reduce the problem to finding the differences of spatial
distortion in between anatomical and \gls{dwi} space.
We thus reformulate the segmentation problem as an inverse problem, where we
seek for an underlying deformation field (the distortion) mapping
from the structural space into the diffusion space, such that the structural
contours segment optimally the \gls{dwi} data.
In the process, the one-to-one
correspondence between the contours in both spaces is guaranteed, and projection
of parcellisation to \gls{dwi} space is implicit and consistent.

We test our proposed joint segmentation-registration model on two different
synthetic examples.
The first example is a scalar sulcus model, where the
\gls{csf}-\gls{gm} boundary particularly suffers from \gls{pve} and can only be
segmented correctly thanks to the shape prior and its coupling with the inner,
\gls{gm}-\gls{wm} boundary through the imposed deformation field regularity.
The second case deals with more realistic \gls{dwi} data stemming from
phantom simulations of a simplistic brain data.
Again, we show that the
proposed model successfully segments the \gls{dwi} data based on two derived
scalar features, namely \gls{fa} and \gls{md}, while establishing an estimate
of the dense distortion field.

The rest of this paper is organized as follows.
First, in \autoref{sec:methods}
we introduce our proposed model for joint multivariate segmentation-registration.
Then we provide a more detailed description of the data and experimental setup in
\autoref{sec:experiments}.
We present results in \autoref{sec:results} and conclude
in \autoref{sec:conclusion}.


The properties of the reconstructed tensors and derived scalar maps have
been studied by \cite{ennis_orthogonal_2006}. Based on their
findings, \gls{fa}~\eqref{eq:fa} and \gls{md}~\eqref{eq:md} are
considered complementary features, and therefore we selected them for the
energy model \eqref{eq:complete_energy} in driving the
registration-segmentation process.
Whereas \gls{fa} informs mainly about the \emph{shape} of diffusion,
the \gls{md} is more related to the \emph{magnitude} of the process:

\begin{align}
\mathrm{FA} &= \sqrt{ \frac{1}{2}}\,\frac{\sqrt{ (\lambda_1 - \lambda_2)^2 + (\lambda_2 - \lambda_3)^2 + (\lambda_3 - \lambda_1)^2}}{\sqrt{ {\lambda_1}^2 + {\lambda_2}^2 + {\lambda_3}^2}} \label{eq:fa} \\
\mathrm{MD} &= ( \lambda_1 + \lambda_2 + \lambda_3 ) / 3 \label{eq:md}
\end{align}
where $\lambda_i$ are the eigenvalues of the diffusion tensor
associated with the diffusion signal $S(\vec{q})$. There exist
two main reasons to justify their choice.
First, they are well-understood and standardized in clinical routine.
Second, together they contain most of the information that is
usually extracted from the \gls{dwi}-derived scalar maps
\cite{ennis_orthogonal_2006}.