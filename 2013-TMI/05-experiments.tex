\section{Data and experiments}
\label{sec:experiments}
%


\subsection{Image Data}
\label{sec:datasets}

\subsubsection{Simulated digital phantom} %
%
The lack of a widely accepted \emph{gold}-standard in the application
field has been addressed by several authors \citep{cote_tractometer:_2013}
In this work, we use one of the most complete, and publicly available,
digital phantoms (\url{http://hardi.epfl.ch/static/events/2013_ISBI/testing_data.html}).
The phantom is a spherical volume containing a set of fiber bundles, that connect 
one area of a ``cortex'' to another. The model accounts for \gls{pve} using 
a similar approach to \citep{close_software_2009} and for \gls{csf} contamination
as well.

\subsubsection{Real \gls{mri} datasets} %
%
We used image data from XX healthy volunteers with no history of neurological
conditions (ages XX$\pm$XX, X female) to illustrate the applicability of our 
approach. All the subjects were scanned in a 3T MR Scanner (Siemens Magnetom
TrioTim blah blah)
with an X-channel head coil. Subjects were scanned twice with the same protocol,
described hereafter. After being scanned the first time, each subject exited the
scan room for a short break and then reentered for an identical scan session.
To note, there was a full repositioning of the volunteer, coils, blankets and pads
before each scan and re-scan session.

The scan session protocol was as follows:
\begin{enumerate}
\item Triplanar survey (Localizer).
\item Field Mapping: field mapping using GRE sequence was performed before \gls{dwi}
acquisition for susceptibility correction purposes.
\item DTI: \gls{dwi} were acquired with axial in-plane 
isotropic resolution $2mm$, slice thickness $2mm$, $XXX \times XXX \times XXX$ 
image matrix, TR= XXXX ms, TE = XX ms, NEX = X, BW = XXXX Hz/pixel, GRAPPA
acceleration factor 2. The series included images acquired with diffusion
weighting along 30 non-collinear directions ( $b$=700sm$^{-2}$ ) sampled
twice for averaging, and 5 interleaved images acquired without diffusion 
weighting ( $b=0$ ).
\item Field Mapping (same as before DTI)
\item Structural T1: An MPRAGE T1-weighted acquisition, sagittal
GRE sequence, in-plane isotropic resolution 1.0 mm, slice thickness 1.2mm
$XXX \times XXX \times XXX$ image matrix, TR=2300 ms, TE=2.98 ms, FA= 9, NEX=X, BW= 240 Hz/pixel.
\item Structural T2: A T2-weighted acquisition, oblique axial
TSE sequence, in-plane isotropic resolution 1.0 mm, slice thickness 1.2mm
$XXX \times XXX \times XXX$ image matrix, TR=3200 ms, TE= 408 ms, NEX=X, BW= 751 Hz/pixel.
\end{enumerate}


\subsection{Image preprocessing and shape-prior generation}
\label{sec:preprocessing}
%
Regardless the dataset type (simulated or real), all \gls{dwi} datasets were processed
using the standard \gls{dti} reconstruction methods provided by FSL \footnote{DTIFIT, 
included in the FMRIB's Software Library (FSL), 
\url{http://fsl.fmrib.ox.ac.uk/fsl/fsl-4.1.9/fdt/fdt_dtifit.html}} to fit
the tensor model and produce scalar maps of the required features.
The properties of the reconstructed tensors and derived scalar maps have
been studied by \cite{ennis_orthogonal_2006}. Based on their
findings, \gls{fa}~\eqref{eq:fa} and \gls{md}~\eqref{eq:md} are
considered complementary features, and therefore we selected them for the 
energy model \eqref{eq:complete_energy} in driving the 
registration-segmentation process. 
Whereas \gls{fa} informs mainly about the \emph{shape} of diffusion, 
the \gls{md} is more related to the \emph{magnitude} of the process:

\begin{align}
\mathrm{FA} &= \sqrt{ \frac{1}{2}}\,\frac{\sqrt{ (\lambda_1 - \lambda_2)^2 + (\lambda_2 - \lambda_3)^2 + (\lambda_3 - \lambda_1)^2}}{\sqrt{ {\lambda_1}^2 + {\lambda_2}^2 + {\lambda_3}^2}} \label{eq:fa} \\
\mathrm{MD} &= ( \lambda_1 + \lambda_2 + \lambda_3 ) / 3 \label{eq:md}
\end{align}
where $\lambda_i$ are the eigenvalues of the diffusion tensor 
associated with the diffusion signal $S(\vec{q})$. There exist 
two main reasons to justify their choice. 
First, they are well-understood and standardized in clinical routine.
Second, together they contain most of the information that is
usually extracted from the \gls{dwi}-derived scalar maps
\cite{ennis_orthogonal_2006}.


Preprocessing differed between simulated and real datasets in the 
shape-prior surfaces generation, as we describe hereafter.


\subsubsection{Simulated digital phantom} %
A description goes here: surfaces extraction, synthetic susceptibility artifact generation.

\subsubsection {Real \gls{mri} data} %

We used a standard automated method 
available in \emph{FreeSurfer} \citep{fischl_freesurfer_2012} to obtain the
cortical gray/white boundary from the T1 scan \citep{greve_accurate_2009}.
Parcellations used to evaluate the repeatability of the method are also
obtained from \emph{FreeSurfer}.


\subsection{Experiments and evaluation}
\label{sec:experiments_evaluation}
%
\subsubsection{Validation on the simulated digital phantom} %
\label{sec:validation_phantom}
We firstly evaluated our approach on the simulated data, using
the distorted data with a deformation field similar to the
susceptibility artifact that affects real \gls{dwi} data. To 
this end, we report the following indices:
\begin{itemize}
\item \gls{se}, a distance between the one-to-one corresponding
vertices, weighted by their respective Voronoi area.
\begin{equation}
MSE = \sum\limits_k \sum\limits_j^M w_j\,\| \mathbf{x}_j - \hat{\mathbf{x}}_j \|
\end{equation}
where $\mathbf{x}_j$ are the locations of the $M$ vertices of the $k$ priors,
$\hat{\mathbf{x}}_j$ are the corresponding locations recovered, and $w_j$ the
weighting factor as the relative surface of the Voronoi area.
\item \gls{wi}, L2-distance between the theoretical and the recovered
deformation field.
\begin{equation}
WI = \frac{1}{N} \sum\limits_i^N \| \mathbf{d}_i - \hat{\mathbf{d}}_i \|
\end{equation}
where $\mathbf{d}_i$ is the theoretical displacement vector at position $i$
and $\hat{\mathbf{d}}_i$ is the recovered one at the same index position.
\item Parcellation agreement, the \gls{se} averaged by defined \glspl{roi}
between the theoretical and the recovered parcellations.
\item \gls{nof} agreement, between the fibers recovered on the original
data and the processed data.
\end{itemize}

Additionally, the entropy of data is studied in both
original and recovered datasets to draw another possible basis for the 
assessment of the real datasets. We also report the same indices for the
outcome of a widely-used methodology that combines field-map susceptibility 
correction and T1-T2-\gls{dwi} registration (\textbf{FIXME}: I suggest to give 
it a compact name and define it in the pre-processing section or even before
in the intro).

\subsubsection{Evaluation on real datasets}
For the real datasets there is no published \emph{gold}-standard to validate results.
Thus, visual results are provided to let compare the performance with the NICEACRONYM
standard methodology. Additionally, cross-comparison of repeatability results are 
provided. In this second evaluation strategy, all the indices defined in 
\autoref{sec:validation_phantom} are reported.
