% -*- root: 00-main.tex -*-
\section{Methods}
\label{sec:methods}
%

\subsection{Cost-function derivation}
\label{sec:methods_map}

In a Bayesian framework, the mappings $U$ in \autoref{eq:transform} are
  evaluated based on their posterior probability given the observed data
  $F$.
Using the Bayes' rule, the posterior likelihood is computed as:
  \begin{equation}
  P(U \mid F,\Gamma_{l,m}) = \frac{P(F \mid U,\Gamma_{l,m})\, P(U)}{P(F)},
  \label{eq:bayes_rule}
  \end{equation}
  where $P(F \mid U,\Gamma_{l,m})$ is the data-likelihood, and
  $\Gamma_{l,m}$ are a set of surfaces corresponding to the interfaces
  between piecewise smooth regions $\Omega_i$, such that
  $\Gamma_{l,m} = \partial \Omega_l \cap \partial \Omega_m$ is the
  contour between tissues $l$ and $m$.
Then, the \gls*{map} criterium \citep{leemput_automated_1999} is used
  to find the mapping estimate $\hat{U}$ that aligns $\Gamma$ into $F$:
  \begin{align}
  \hat{U} &= \underset{U}{\argmax} \left\{ P(U \mid F,\Gamma_{l,m}) \right\} \notag\\
   &= \underset{U}{\argmax} \left\{ P(F \mid U,\Gamma_{l,m})\, P(U) \right\}.
  \label{eq:map_u}
  \end{align}

First, we assume independence between pixels, and thus break down the
  global data likelihood into a product of pixel-wise conditional probabilities:
  \begin{equation}
  P(F \mid U,\Omega) = \underset{l}{\prod} \underset{\vec{r}\in \Omega_l}{\prod}
    P\left( \vec{f}' \mid U \right),
  \label{eq:bayes_aposteriori}
  \end{equation}
  where $\vec{f}' = F(\vec{r}')$ is the feature vector at the displaced
  position $\vec{r}'$ (\autoref{eq:transform}) of the multispectral target
  volume.
For convenience, and because it has been shown to be appropriate approximations
  \citep{cuadra_comparison_2005}, we introduce two assumptions for each
  region $\Omega_l$:
  1) the features are i.i.d.; and 2) they can be modeled by multivariate normal
  distributions:
  \begin{equation}
  P\left( \vec{f}' \mid U,\Omega_l \right) = \mathcal{N} \left( \vec{f}' \mid \Theta_l \right),
  \label{eq:likelihood}
  \end{equation}
 	where $\Theta_l = \lbrace \boldsymbol{\mu}_l, \boldsymbol{\Sigma}_{l} \rbrace$ are the
 	descriptors of region $\Omega_l$.
We then insert \autoref{eq:likelihood} into \autoref{eq:bayes_rule} to obtain the full
  formulation of the \emph{data term} of our registration model:
 	\begin{align}
  P( F \mid U, \Omega ) &= \underset{l}{\prod} \underset{\vec{r} \in \Omega_l}{\prod}
  \mathcal{N} ( \vec{f} \mid \boldsymbol{\mu}_l, \boldsymbol{\Sigma}_{l} ) \notag\\
  &= \underset{l}{\prod} \underset{\vec{r} \in \Omega_l}{\prod} \frac{1}{ \sqrt{(2\pi)^{C}\,\left|\boldsymbol{\Sigma}_{l}\right|}}\,{e^{\left(-\frac{1}{2}
  \mdist{f'}{l} \right)}},
  \label{eq:pdf}
  \end{align}
  using $\mdist{f}{l}$ to denote the squared \emph{Mahalanobis distance} of $\vec{f}$ with respect
  to the descriptors of region $l$ as
  $\mdist{f}{l} = (\vec{f} - \boldsymbol{\mu}_l)^T \, {\boldsymbol{\Sigma}_l}^{-1} \, (\vec{f} - \boldsymbol{\mu}_l)$.


We choose a Thikonov regularization prior as follows:
  \begin{equation*}
  P(U) = \underset{\vec{r}}{\prod} p(u(\vec{r})) =
  \underset{\vec{r}}{\prod} p_0(u(\vec{r})) \, p_1(u(\vec{r})),
  \end{equation*}
  with further:
  \begin{align*}
  p_0(u(\vec{r})) &= \mathcal{N}( u(\vec{r}) \mid 0, A^{-1}), \notag\\
  p_1(u(\vec{r})) &= \mathcal{N}(  \nabla \cdot u(\vec{r}) \mid 0, B^{-1}),
  \end{align*}
  imposing that the distortion and its gradient have zero
  mean and variance governed by $A$ and $B$.
Since the anisotropy is generally aligned with the imaging axes, these will be simplified
  in the following for the sake of clarity:
  \begin{align}
    p_0(u(\vec{r})) &= \underset{\vec{r}}{\sum} \mathcal{N}( u(\vec{r}) \mid 0,
      (\boldsymbol{\alpha}^{\circ\frac12}\,\vec{I}_n)^{-1}), \notag\\
    p_1(u(\vec{r})) &= \underset{\vec{r}}{\sum} \mathcal{N}( \nabla \cdot u(\vec{r}) \mid 0,
      (\boldsymbol{\beta}^{\circ\frac12}\,\vec{I}_n)^{-1}).
  \label{eq:priors}
  \end{align}

Finally, the \gls{map} problem is adapted into a variational one applying the
  following log-transform:
  \begin{align}
  E(F \mid U) &= -\log \underset{l}{\prod}
  \underset{\vec{r} \in \Omega_l}{\prod}
  \mathcal{N} \left( \vec{f}' \mid \Theta_l \right)\,p_0( u(\vec{r}))\,p_1( u(\vec{r})) \notag\\
  &= C + \underset{l}{\sum}
  \underset{\vec{r} \in \Omega_l}{\sum}
  \mdist{f'}{l} \notag\\
  &+ \underset{\vec{r} \in \Omega}{\sum} \boldsymbol{\alpha} \cdot u(\vec{r})^{\circ2}
  + \underset{\vec{r} \in \Omega}{\sum} \boldsymbol{\beta} \cdot (\nabla \cdot u(\vec{r}))^{\circ2},
  \label{eq:energy}
  \end{align}
  that is the dual expression to the energy functional corresponding
  to a discrete \gls*{acwe} framework \citep{chan_active_2001}
  with anisotropic regularization as studied in
  \citep{nagel_investigation_1986}.
We provide a proof of duality in {\color{red} [REF figshare?]}.


\subsection{Numerical Implementation}
\label{sec:numerical_implementation}

\subsubsection{Deformation model}
\label{sec:deformation_model}
Let us denote $\{\vec{c}_i\}_{i=1 \ldots N_c}$ the nodes of one or several prior
  surface(s).
In our application, these surfaces are extracted using \emph{FreeSurfer}
  \citep{fischl_freesurfer_2012} in the case of real datasets.
The transformation from the structural space into the coordinate space of the
  target image is achieved through a dense deformation field $u(\vec{r})$, such that:
  \begin{equation}
  \vec{c}_i' = U\{\vec{c}_i\} = \vec{c}_i + u(\vec{c}_i),
  \label{eq:nodes_tfm}
  \end{equation}
  where $U$ is defined in \eqref{eq:transform}.
Since the nodes of the anatomical surfaces likely lay off-grid, it is required to
  derive $u(\vec{r})$ from a discrete set of parameters $\{\vec{u}_k\}_{k=1 \ldots K}$.
Densification is achieved through a set of associated basis functions $\psi_k$:
  \begin{equation}
  u(\vec{r}) = \sum_k \psi_k(\vec{r}) \vec{u}_k.
  \label{eq:intp_kernel}
  \end{equation}
%
In our implementation, $\psi_k$ is chosen to be a tensor-product B-Spline kernel
  of degree 3 ($B_3$).
Then, introducing \eqref{eq:intp_kernel} into \eqref{eq:nodes_tfm} and replacing
  $\psi$ by the actual kernel function, the transformation writes:
  \begin{equation}
    \vec{c}_i' = \vec{c}_i + \sum_k \left[ \vec{u}_k \, \underset{d}{\prod}
      B_3( (\vec{c}_i - \vec{r}_k) \cdot \hat{\mathbf{e}}_d ) \right],
  \label{eq:transformation}
  \end{equation}
  with $\hat{\mathbf{e}}_d$ being the unitary vector along axis $d$.


\subsubsection{Optimization}
\label{sec:gradient_descent}
To find the minimum of the energy functional \eqref{eq:energy},
  we propose a gradient-descent approach with respect to the underlying
  deformation field through the following \gls*{pde}:
  \begin{equation}
  \frac{\partial u(\vec{r},t)}{\partial t} \propto - \frac{\partial E(\vec{u})}{\partial \vec{u}_k},
  \label{eq:general_gradient_descent}
  \end{equation}
  with $t$ being an artificial time parameter of the contour
  evolution, and $\vec{u}_k$ the parameters supporting the estimate
  $\hat{U}$ of the transformation at the current time point.
Now, we introduce \eqref{eq:energy} in \eqref{eq:general_gradient_descent}:
  \begin{align}
  \frac{\partial E(\vec{u})}{\partial \vec{u}_k} &=
  \frac{ \partial }{\partial \vec{u}_k} \Big\{
  C + \underset{l}{\sum}
  \underset{\vec{r} \in \Omega_l}{\sum} \mdist{f'}{l} \notag\\
  &+ \underset{\vec{r} \in \Omega}{\sum} \boldsymbol{\alpha} \cdot u(\vec{r})^{\circ2}
  + \underset{\vec{r} \in \Omega}{\sum} \boldsymbol{\beta} \cdot (\nabla \cdot u(\vec{r}))^{\circ2}
  \Big\}.
  \label{eq:gradient_descent}
  \end{align}

Therefore, we can apply a discretized interpretation of \eqref{eq:shape_gradients}
  to compute the data term in \eqref{eq:gradient_descent} as follows:
  \begin{align}
  \frac{\partial E_{data}(\vec{u})}{\partial \vec{u}_k} &=
  \underset{l}{\sum} \frac{ \partial }{\partial \vec{u}_k} \left\{
   \underset{\vec{x} \in \Omega_l}{\sum} \mdist{f'}{l} \right\} \notag\\
  &= \underset{l,m}{\sum} \underset{i}{\sum}
  \left[ \mdist{f_i'}{l} - \mdist{f_i'}{m} \right]
  \left\langle \frac{\partial \vec{c}_i'}{\partial \vec{u}_k}, \vec{n_i}'\right\rangle,
  \label{eq:gradient_wshape}
  \end{align}
  in this case, the formulation has been adapted to the non-binary case, $l,m$
  being any pair of neighboring regions, and $\Gamma_{l,m}$ the contour separating
  them such that $\vec{x}' = \vec{c}' \in\Gamma_{l,m} \iff \vec{x}\in \partial\Omega_i \cap \partial\Omega_j$
  and $\vec{n_i}'$ is the unit inward normal to the contour at $c_i'$.

Finally, we can compute:
  \begin{align}
  \frac{\partial \vec{c}_i'}{\partial \vec{u}_k} &= \frac{\partial}{\partial \vec{u}_k}
  \left\{ \vec{c}_i + \sum_k \psi_k(\vec{c}_i) \vec{u}_k \right\}
  = \psi_k(\vec{c_i})\, \hat{\vec{e}}
  \end{align}
  where $\hat{\vec{e}}$ is the coordinates system's unit vector.
  Projecting this gradient onto the surface normal,
  $\left\langle \frac{\partial}{\partial \vec{u}_k}{\vec{c}_i}', \vec{n_i}'\right\rangle
  = \psi_k(\vec{c_i})\, \hat{\vec{n}}_i$, then the
  full gradient evolution equation \eqref{eq:gradient_descent} yields:
  \begin{align}
  \frac{\partial E(\vec{u}_k)}{\partial \vec{u}_k} =
  &- \underset{l,m}{\sum} \underset{i}{\sum}
  \left[ \mdist{f_i'}{l} - \mdist{f_i'}{m} \right]
  \psi_k(\vec{c}_i)\, \hat{\vec{n}}_i \notag\\
  &+2\, \boldsymbol{\alpha} \vec{u}_k
  -2\, \boldsymbol{\beta} \Delta \vec{u}_k,
  \label{eq:gradient_final}
  \end{align}

\subsubsection{Region descriptor reestimation}
In regular intervals, after certain number of iterations,
the parameters $\Theta_l$ of the regions can be reestimated
based on the shifted volumetric samples
$\vec{x}' = \vec{x_0} + u(\vec{x})$.

\subsubsection{Convergence}
{\color{red} discuss choice of $\tau$, Courant-Friedrichs-Lewy (CFL) condition / Wolfe conditions, multi-resolutions on the bspline grid, image samples and surface sampling,  etc.}

\subsubsection{Efficient field densification}
As long as the dense deformation field is iteratively interpolated
from the same set of control points that define the $L-1$ contours,
it is possible to pre-cache all the $\psi_k(c_i)$ weights into a
sparse matrix for fast densification. As well, we achieve a
diffeomorphic transform by


\subsection{Image Data and preprocessing}
\label{sec:datasets}

\subsubsection{Simulated digital phantom} %
\label{sec:digital_phantoms}
%
We first present the proof of the concept on a simplistic digital phantom,
  simulating a spherical crust of thin tissue with a folding resembling
  a gyrus of the brain cortex folding (see \autoref{fig1}).
We simulated 2.0mm isotropic \gls*{t1} and \gls*{t2} images at
  several rician noise levels, with parameters TE/TR=10/1500ms and
  TE/TR=90/5000ms respectively using \cite{caruyer_phantomas_2014}.
The reference surfaces were extracted using the marching-cubes algorithm
  on high-resolution (1.0mm isotropic) binary versions of the corresponding
  maps of tissue volume fractions used during the simulation step.

\subsubsection{Real \gls{mri} datasets} %
\label{sec:human_connectome}
%
For the evaluation of the algorithm with real brains with \gls*{dmri} data,
  we randomly selected 10 datasets from the minimally preprocessed
  \citep{glasser_minimal_2013} cohort of the \gls*{hcp}.
These datasets comprehend the preprocessed (corrected for artifacts,
  brain-extracted and spatially normalized) \gls*{t1}, \gls*{t2} and
  multi-shell \gls*{dmri} images, along with the most prominent results of
  the standard processing with \emph{FreeSurfer}: \emph{aseg} and
  \emph{aparc} segmentations, and the surfaces corresponding to the cortical
  \gls*{gm} exterior (\emph{pial}) and the \gls*{wm}/\gls*{gm} interface
  (\emph{white}).
This cohort is especially convenient for their careful preprocessing,
  making available high-standard \gls*{mri} data accurately corrected
  for distortions and artifacts.
For the exact details of the scanning parameters we refer the reader to
  \citep{essen_human_2012}.
Selecting the appropriate labels in the \emph{aseg} segmentation, we applied
  the marching-cubes algorithm again to extract the surfaces containing the
  \gls*{csf} of the ventricular system.

As we propose the \gls*{fa} and the \gls*{md} as target features, we
  implemented a simplified pipeline for processing \gls*{dmri} using
  \emph{MRtrix} \citep{tournier_mrtrix_2012}.
\todo[inline]{insert here the summary of parameters}

\subsection{Experiments and evaluation}
\label{sec:experiments_evaluation}
%
\subsubsection{Synthetic distortion}
\label{sec:distortion}



\subsubsection{Evaluation indices} %
\label{sec:evaluation}
We report the following indices:
  \begin{itemize}
  \item \gls{se}, a distance between the one-to-one corresponding
    vertices, weighted by their respective Voronoi area.
    \begin{equation}
    MSE = \sum\limits_k \sum\limits_j^M w_j\,\| \mathbf{x}_j - \hat{\mathbf{x}}_j \|
    \end{equation}
    where $\mathbf{x}_j$ are the locations of the $M$ vertices of the $k$ priors,
    $\hat{\mathbf{x}}_j$ are the corresponding locations recovered, and $w_j$ the
    weighting factor as the relative surface of the Voronoi area.
  \item \gls{wi}, L2-distance between the theoretical and the recovered
    deformation field.
    \begin{equation}
    WI = \frac{1}{N} \sum\limits_i^N \| \mathbf{d}_i - \hat{\mathbf{d}}_i \|
    \end{equation}
    where $\mathbf{d}_i$ is the theoretical displacement vector at position $i$
    and $\hat{\mathbf{d}}_i$ is the recovered one at the same index position.
  \item Parcellation agreement, the Dice overlap index of the \glspl*{roi}
    contained by the parcellation.
  \end{itemize}
