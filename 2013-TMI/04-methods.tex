\section{Methods}
\label{sec:methods}
%
\subsection{Related work}
\label{sec:methods_background}

We suggest clustering the 
current methodologies of template-based segmentation methods into three groups. 
The first group typically adds a shape prior term to the energy functional of 
an evolving active contour \citep{bresson_variational_2006,
chan_level_2005,chen_using_2002,cremers_kernel_2006,gastaud_combining_2004}.
These methods generally have an explicit description of the expected relative boundary 
locations of the object to be delineated, and some even model the statistical deviations
from this average shape. By including a coordinates mapping, it is possible to perform
active contours based registration between timesteps in a time-series or between different
images \citep{bertalmio_morphing_2000,wyatt_map_2003,paragios_level_2003,vemuri_joint_2003,
yezzi_variational_2003}.
A summary of these second set of techniques is performed in \citep{droske_mumfordshah_2009},
proposing two different approaches to applying the Mumford-Shah \citep{mumford_optimal_1989}
functional in joint registration and segmentation. Finally, a derivation of the 
latter groups is composed by atlas-based segmentation methods 
\citep{gorthi_segmentation_2009,gorthi_active_2011,pohl_unifying_2005,
pohl_bayesian_2006,wang_joint_2006}, where the prior 
imposes consistent voxel-based classification of contiguous regions.
A comprehensive summary of the \gls{acwe}-derived methodologies with special attention
to joint registration-segmentation methods is found in \citep{gorthi_active_2011}.
\Gls{acwe} and level-set based formulations have been widely applied in image
processing, as reviewed in \citep{suri_shape_2002}.

The first joint ``morphing''-segmentation approach (it was not a proper
registration method) was proposed in \citep{bertalmio_morphing_2000}. The first
registration framework is presented by \citep{yezzi_variational_2001} where the
energy functional is defined simultaneously in the moving and target images with
an affine transformation supporting the coordinates mapping (``2 PDEs approach'').
\citep{vemuri_joint_2003} proposed the first atlas-based registration based on the 
level set framework and using only one PDE. \citep{unal_coupled_2005} extended the 
idea of \citep{bertalmio_morphing_2000,yezzi_variational_2001} for non-linear 
registration with a dense deformation field as mapping function. \citep{droske_mumfordshah_2009}
combining ideas from both branches and proposed the propagation of the 
deformation field from the contours to the whole image definition. Finally,
\citep{gorthi_active_2011} present a comprehensive generalization of the methodologies.

Even though the historical evolution
of joint segmentation and registration procedures is strongly linked
to the level sets approximation, our work is based in derivations of
the \gls{acwe} framework \citep{chan_active_2001}, and related to
\citep{le_guyader_combined_2011}.
{\color{red} {The main diverging points with respect to
\citep{gorthi_active_2011} are: 1) there is no need for an explicit level set function
$\Phi_G$, as we replace the level set gradient computation $N_{\Phi_G}$ with shape
gradients \citep{jehan-besson_dream2s:_2003,herbulot_segmentation_2006};
2) regularization is also based on linear diffusion smoothing \citep{thirion_image_1998},
but we replace the Gaussian filtering by other constraints
studied in \citep{nagel_investigation_1986} to better the problem;
3) optimization is applied in the spectral
domain, observing anisotropic and inhomogeneous mappings along each direction.
With respect \citep{le_guyader_combined_2011}, the main differences are
the distance function, and the spectral solution to the optimization updates,
as we shall cover in \autoref{sec:numerical_implementation}.}}

\todo[inline]{The previous paragraphs in red color are marked by Dominique
as ``to rework'', and I totally agree}


\subsection{Proposed model}
\label{sec:methods_map}
In this paper we formulate the joint registration-segmentation
problem as follows. We look for a spatial mapping (with, typically,
$n=3$ for the 3-dimensional case):
\begin{align*}
U\colon T \subset \mathbb{R}^n &\to D \subset \mathbb{R}^n \\
\mathbf{x} &\mapsto \mathbf{x}' =\mathbf{x}+u(\mathbf{x}),
\end{align*}
such that the known contours in anatomical space $T$ optimally segment
the diffusion space $D$.

In a Bayesian framework, different such mappings $U$ between ``T1-''
and ``\gls*{dwi}-'' spaces are evaluated based on their posterior
probability given the observed data $F$. Using the Bayes' rule, this
posterior \emph{likelihood} can be computed as:
\begin{equation}
P(U \mid F,\Gamma_{ij}) = \frac{P(F \mid U,\Gamma_{ij})\, P(U)}{P(F)},
\label{eq:bayes_rule}
\end{equation}
where $P(F \mid U,\Gamma_{ij})$ is the data-likelihood, and
$\Gamma_{ij}$ are a set of contours in T1-space that correspond to
two interfacing regions $\Omega$, such that 
$\Gamma_{ij} = \partial \Omega_i \cap \partial \Omega_j$.

We are interested in a mapping $\hat{U}$ which maximizes this
posterior probability (\gls*{map} estimate, \citep{bishop_pattern_2006}):
\begin{align}
\hat{U} &= \underset{U}{\argmax} \left\{ P(U \mid F,\Gamma_{ij}) \right\} \notag\\
 &= \underset{U}{\argmax} \left\{ P(F \mid U,\Gamma_{ij})\, P(U) \right\}.
\label{eq:map_u}
\end{align}

First, we assume independence between pixels, and thus break down the
global data likelihood into a product of pixel-wise conditional probabilities:

\begin{equation}
P(F \mid U,\Gamma_{ij}) = 
\underset{\mathbf{x}\in \Omega}{\prod} P\left( F(\mathbf{x}') \mid U(\Gamma_{ij}) \right).
\label{eq:bayes_aposteriori}
\end{equation}

Further, within each region $\Omega_k$ defined by the contours $\Gamma$
trough duality, we assume the features to be i.i.d.:
\begin{equation}
P\left( F(\mathbf{x}') \mid U,\Gamma_{ij} \right) = p_k( F(\mathbf{x}')) \quad
\forall \mathbf{x} \in \Omega_k.
\label{eq:likelihood}
\end{equation}

For convenience, and because this has been shown to be an appropriate
approximation \citep{bishop_pattern_2006}, each such region is modeled
by a multivariate normal distribution with region descriptors
$\Theta_k = \lbrace \boldsymbol{\mu}_k, \boldsymbol{\Sigma}_{k} \rbrace$:

\begin{equation}
p_k( F(\mathbf{x}') ) = \mathcal{N} ( f \mid \Theta_k ).
\label{eq:multivariate_normal}
\end{equation}

\todo{I modified this slightly}
Thus, defining the feature observed as $\mathbf{f}'=F(\mathbf{x}')$ and 
the squared \emph{Mahalanobis distance} of $\mathbf{f}'$ with respect
of the descriptors of region $k$ as
$\Delta^2_k (\mathbf{f}' \mid \Theta_k ) = (\mathbf{f}' - \boldsymbol{\mu}_k)^T \, \boldsymbol{\Sigma}^{-1}_k \, (\mathbf{f}' - \boldsymbol{\mu}_k)$,
we write:
\begin{align}
P( F \mid U, \Omega ) &= \underset{k}{\prod} \underset{\mathbf{x} \in \Omega_k}{\prod}
\mathcal{N} ( f \mid \boldsymbol{\mu}_k, \boldsymbol{\Sigma}_{k} ) \notag\\
&= \frac{1}{ \sqrt{(2\pi)^{C}\,\left|\boldsymbol{\Sigma}_{k}\right|}}\,{e^{\left(-\frac{1}{2}  \Delta^2_l (\mathbf{f}' \mid \Theta_k ) \right)}}.
\label{eq:pdf}
\end{align}

We choose a Thikonov regularization prior as follows:

\begin{equation*}
P(U) = \underset{\mathbf{x}}{\prod} p(u)
\end{equation*}
with further:
\begin{align*}
p(u) &= p_0(u) \, p_1(u) \notag\\
p_0(u) &= \mathcal{N}( u \mid 0, A^{-1}) \notag\\
p_1(u) &= \mathcal{N}( Du \mid 0, B^{-1}),
\end{align*}
that consider that the distortion and its gradient have zero
mean and variance governed by $A$ and $B$. Since the anisotropy
is technically aligned with the imaging axes, these can
be simplified:
\begin{align}
p_0(u) &= \underset{\mathbf{x}}{\sum} \mathcal{N}( u(\mathbf{x}) \mid 0, \frac{1}{\sqrt{\alpha_{x,y,z}}}) \notag\\
p_1(u) &= \underset{\mathbf{x}}{\sum} \mathcal{N}( \nabla_{x,y,z}u(\mathbf{x}) \mid 0, \frac{1}{\sqrt{\beta_{x,y,z}}})
\label{eq:priors}
\end{align}

Finally, we can turn the \gls{map} problem into 
a variational one
applying the following log-transform:
\begin{align}
E(F \mid U) &= -\log \underset{k}{\prod}
\underset{\mathbf{x} \in \Omega_k}{\prod} 
p_k( \mathbf{f}')\,p_0( u(\mathbf{x}))\,p_1( u(\mathbf{x})) \notag\\
&= C + \underset{k}{\sum}
\underset{\mathbf{x} \in \Omega_k}{\sum}
\Delta^2_k (\mathbf{f}' \mid \Theta_k ) \notag\\
&+ \underset{d=\{x,y,z\}}{\sum} \alpha_d u^2_d
+ \underset{d=\{x,y,z\}}{\sum} \beta_d \|\nabla u_d\|^2,
\label{eq:energy}
\end{align}
that is the dual expression to the energy functional corresponding
to a discrete \gls*{acwe} framework \citep{chan_active_2001}
with anisotropic regularization as studied in
\citep{nagel_investigation_1986}.

\subsection{Numerical Implementation}
\label{sec:numerical_implementation}
%
Let us denote $\{c_i\}_{i=1 \ldots N_c}$ the nodes of one or several shape-prior
surface(s). In our application, precise tissue interfaces of interest 
extracted from a high-resolution, anatomically correct reference volume. 
On the other hand, we have a number of \gls{dwi}-derived features at each
voxel of the volume. Let us denote by $\mathbf{x}$ the voxel and $f(\mathbf{x}) = 
[ f_1, f_2, \ldots, f_N]^T(\mathbf{x})$ its associated feature vector.

\subsubsection{Deformation field}
\label{sec:deformation_field}
The transformation from reference into \gls{dwi} coordinate space is 
achieved through a dense deformation field $u(\mathbf{x})$, such that:
\begin{equation}
c_i' = T\{c_i\} = c_i + u(c_i),
\end{equation}
what is equivalent to \eqref{eq:omega_tf}. Since the nodes of the anatomical 
surfaces might lay off-grid, it is required to derive $u(\mathbf{x})$ from a discrete 
set of parameters $\{u_k\}_{k=1 \ldots K}$. Densification is achieved through 
a set of associated basis functions $\psi_k$ (e.g. rbf, interpolation splines):
%
\begin{equation}
u(\mathbf{x}) = \sum_k \psi_k(\mathbf{x}) u_k
\end{equation}
%
In our implementation, $\psi_k$ is chosen to be a tensor-product B-Spline kernel.
Then, the transformation writes
%
\begin{equation}
\label{eq:transformation}
c_i' = T\{c_i\} = c_i + u(c_i) = c_i + \sum_k \psi_k(c_i)\,u_k
\end{equation} 

Based on the current estimate of the distortion $u$, we can compute 
``expected samples'' within the shape prior projected into the \gls{dwi}.
Thus, we now estimate region descriptors of the \gls{dwi} features 
$f(\mathbf{x})$ of the regions defined by the priors in \gls{dwi} space.
%
Using the region descriptors derived in \autoref{sec:methods_map}, we propose
an \gls{adf}-like, \gls{acwe}-based, piece-wise constant, image segmentation
model (where the unknown is the deformation field)
\cite{chan_active_2001} with the energy functional obtained in 
\eqref{eq:final_map_energy}. This inverse problem is ill-posed
\cite{bertero_ill-posed_1988,hadamard_sur_1902}.
In order to account for deformation field regularity and to render the 
problem well-posed, we include limiting and regularization terms into 
the energy functional \cite{morozov_linear_1975,tichonov_solution_1963}:
%
\begin{multline}
E(u) = \sum\limits_l \int_{\Omega'_l} \Delta^2_l (\mathbf{f}) \,d\mathbf{x} \\
+ \int u^T \, A \, u \, d\mathbf{x} + \int \tr\{(\nabla u^T)^T B (\nabla u^T)\} d\mathbf{x}
\label{eq:complete_energy}
\end{multline}
%
These regularity terms ensure that the segmenting contours in 
\gls{dwi} space are still close to their native shape. The model
easily allows to incorporate inhomogeneous and anisotropic 
regularization \cite{nagel_investigation_1986} to better regularize
the \gls{epi} distortion.


\subsubsection{Shape gradients}
To compute the gradient-descent of the data-term domain integrals
in \eqref{eq:complete_energy} with respect to the underlying 
deformation field, we use shape gradients 
\cite{jehan-besson_dream2s:_2003,herbulot_segmentation_2006}.

Let $\Omega$ be an image domain and $d\Omega$ its boundary. Further, 
$r(\mathbf{x})$ is an ``arbitrary'' function over the image domain. 
We now derive the domain integral w.r.t. the contour evolution 
parameter $t$ (time):
\begin{equation}
\frac{\partial}{\partial t} \int_\Omega r(\mathbf{x}) d\mathbf{x} = \int_\Omega \frac{\partial r}{\partial t}(\mathbf{x}) d\mathbf{x} - \int_{d\Omega} r(\mathbf{x}) \left\langle \frac{\partial{d\Omega}}{\partial t}, N_{d\Omega}\right\rangle d\mathbf{x}
\label{eq:shape_gradients}
\end{equation}
where $\left\langle\frac{\partial{d\Omega}}{\partial t}, N_{d\Omega}\right\rangle$ is 
the projection of the boundary movement on the unit inward normal. We recall
here that equation \eqref{eq:shape_gradients} stands the bridge between the 
explicit level set formulations surveyed in \autoref{sec:methods_background}
by updating the distortion along the steepest energy descent:
\begin{equation}
\frac{\partial u(\mathbf{x},t)}{\partial t} = - \frac{\partial E(u)}{\partial u},
\end{equation}
what closes the loop with the general expression of the \glspl{adf} introduced in
\eqref{eq:active_deformation_fields}.

\subsubsection{Operator splitting}
In order to make the energy minimization computationally more tractable, 
we propose the following operator splitting: Let us optimize the data terms 
and the regularity terms on separate copies of the deformation field, 
now called $u$ and $v$, constrained to be equal:
\begin{multline}
E(u,v) = \sum\limits_l \int_{\Omega'_l(u)} \Delta^2_l (\mathbf{f}) \,d\mathbf{x} \\
+ \int v^T \, A \, v \, d\mathbf{x} + \int \tr\{(\nabla v^T)^T B (\nabla v^T)\} d\mathbf{x}
\label{eq:operator_splitting}
\end{multline}
and now
\begin{equation}
\min_{u,v} \{ E \} \quad s.t. \quad u = v.
\end{equation}

In order to take the equality constraint into account, we may make use of 
augmented Lagrangians (a combination of Lagrangian multipliers and penalty 
terms on the constraint) \cite{bertsekas_multiplier_1976,
glowinski_augmented_1989,nocedal_numerical_2006}:
\begin{equation}
AL(u,v,\lambda,r) = E(u,v) + \langle \lambda, u-v \rangle + \frac{r}{2} \| u - v \|_2^2
\end{equation}

To solve the constraint minimization problem, we may now optimize the 
Augmented Lagrangian in an iterative way:
\begin{equation}
\left\lbrace 
\begin{array}{rcl}
u^{t+1} &=& \argmin_{u} AL(u,v^t,\lambda^t,r)\\
v^{t+1} &=& \argmin_{v} AL(u^{t+1},v,\lambda^t,r)\\
\lambda^{t+1} &=& \lambda^t + \rho(u^{t+1}-v^{t+1})
\end{array}\right.
\end{equation}
where typically $0 < \rho < r$. The two sub-minimization problems will 
now be much easier to handle than the original complete problem.

\subsubsection{Minimization w.r.t. $u$}
The first minimization problem optimizes the data-term. The problem is equivalent 
to minimizing the following energy:
\begin{multline}
E(u,v) = \sum\limits_l \int_{\Omega'_l(u)} \Delta^2_l (\mathbf{f}) \,d\mathbf{x}
+ \langle \lambda, u-v \rangle + \frac{r}{2} \| u - v \|_2^2
\end{multline}
where we identify one instance of domain integrals of the form $\int_\Omega r(\mathbf{x}) 
d\mathbf{x}$ for each region $l$. Optimality requires the derivative of this energy 
with respect to the parameters $u$ to be null. At this point, we decide to ignore 
the influence of the boundary shift on the statistics of the regions (i.e. moving the 
boundary does not significantly impact the $\mu$ and $\Sigma$ descriptors). This means 
that we can drop the derivative of $r(\mathbf{x})$ w.r.t. contour evolution. 
What remains are the surface integrals at the subdomain interfaces, plus the Lagrangian 
and penalty terms:

\begin{multline}
\frac{\partial E}{\partial u_k} = \sum\limits_l \int_{d\Omega'_l} \left[ \Delta^2_{O(l)} (\mathbf{f}) - \Delta^2_l (\mathbf{f}) \right]
\left\langle\frac{\partial{d\Omega'_l}}{\partial t}, N_{d\Omega'_l}\right\rangle \,d\mathbf{x} \\
+ \lambda_k + r(u_k - v_k) = 0
\label{eq:energy_gradient}
\end{multline}
where $O(l)$ is a function that returns the neighboring region at $\mathbf{x}$ (i.e.
the region \emph{outside} the current location in the contour). For the shake of
simple notation, we will refer as $c_i$ to the positions 
of points belonging to any of the existing contours $d\Omega_l$ (with 
$i \in [1 \ldots N_c]$ and $N_c$ the total number of control points).
Given the deformation field interpolation stated in \autoref{sec:deformation_field}, 
the boundary moves according to
\begin{equation}
\frac{\partial c_i}{\partial u_k} = \psi_k(c_i)\,e_a
\end{equation}
where $e_a$ is the unit vector along direction $a \in \mathbb{R}^n$ and $n$ the dimension
of the image. Thus, if we name the inward unit normal of the corresponding transformed 
contour at position $c'_i$ as $\hat{\mathbf{n}}_{c'_i}$,
\begin{equation}
\left\langle\frac{\partial c_i}{\partial u_k}, N_{c_i}\right\rangle = \psi_k(c_i)\,\hat{\mathbf{n}}_{c'_i}.
\end{equation}
The discrete implementation of \eqref{eq:energy_gradient} is straightforward as
we have a triangularized mesh representation of the interfacing surfaces:
\begin{multline}
\frac{\partial E}{\partial u_k} = \sum\limits_{N_c} \left[ \Delta^2_{O(l)} (\mathbf{f}) - \Delta^2_l (\mathbf{f}) \right]
\, \psi_{c_i}(\mathbf{x}_k)\,\hat{\mathbf{n}}_{c'_i} \\
+ \lambda_k + r(u_k - v_k) = 0,
\label{eq:discrete_energy_gradient}
\end{multline}
where we have swapped $\psi_k(c_i)$ by $\psi_{c_i}(\mathbf{x}_k)$ for
convenience in implementation, taking the advantage that they are
equivalent. Subsequently, the optimal distortion $u_k$ 
is found at each iteration as:
\begin{multline}
u_k^{t+1} = v_k^t - \frac{1}{r}\lambda_k^{t} \\
- \frac{1}{r} \sum\limits_{N_c} \left[ \Delta^2_{O(l)} (\mathbf{f}) - \Delta^2_l (\mathbf{f}) \right]
\, \psi_{c_i}(\mathbf{x}_k)\,\hat{\mathbf{n}}_{c'_i}
\label{eq:update_u}
\end{multline}


\subsubsection{Minimization w.r.t. $v$}
For the optimization w.r.t. $v$, the relevant energy writes
\begin{align}
E(v) &= \int v^T A v d\mathbf{x} + \int \tr\{(\nabla v^T)^T B (\nabla v^T)\} d\mathbf{x}\\
&\quad + \langle \lambda, u-v \rangle + \frac{r}{2} \| u - v \|_2^2\nonumber
\end{align}
The associated Euler-Lagrange equation is found as:
\begin{equation}
2 ( A v - B \Delta v ) + rv = ru + \lambda
\end{equation}
which easily translates into Fourier domain:
\begin{equation}
v^{t+1} = \mathcal{F}^{-1}\left\{ \frac{\mathcal{F}\{ru + \lambda\}}{\mathcal{F}\{( 2 A+r)\mathcal{I}-2 B \Delta\}} \right\}
\end{equation}
where $\mathcal{I}$ denotes the identity operator.

Here, we rewrite the Laplacian as a linear combination of the identity and shift operators:
\begin{equation}
\Delta = \sum\limits_d \mathcal{S}_d^- + \mathcal{S}_d^+ - 2 \mathcal{I}
\end{equation}
where $\mathcal{S}_{d}^{\pm}$ stands for the forward ($+$) and backward ($-$) shift 
operator along coordinates axis $d$, of which the Fourier transform is found easily as
\begin{equation}
\mathcal{F}\{\mathcal{S}_{d}^{\pm}\} = e^{\pm i\omega_{d}},
\end{equation}
where $\omega_{d}$ is the normalized pulsation along $d$-direction. Accordingly, the 
Fourier transform of the discrete Laplacian is found as
\begin{equation}
\mathcal{F}\{\Delta\} = \sum\limits_d e^{-i\omega_d } + e^{i\omega_d } - 2 = \sum\limits_d \left( 2\cos(\omega_d) - 2 \right)
\end{equation}

The remaining transforms are trivial or can be computed using the \gls{fft} 
as in \citep{estellers_efficient_2011}.


\subsubsection{Lagrangian multiplier update}
At each iteration, the Lagrangian multipliers are updated as noted before:
\begin{equation}
\lambda^{t+1} = \lambda^t + \rho(u^{t+1}-v^{t+1})
\end{equation}

\subsubsection{Region descriptor reestimation}
In regular intervals, after certain number of iterations,
the parameters $\Theta_l$ of the regions can be reestimated 
based on the shifted volumetric samples 
$\mathbf{x}' = \mathbf{x_0} + u(\mathbf{x})$.

\subsubsection{Convergence}
In order to ``fixate'' the evolution when close to convergence,
it is advised to slightly increase the penalty weight $r$ at each 
iteration. Note that as can be seen in the above equations, $r$ 
governs the step-size or leash-length at each iteration, i.e. the 
amount by which the new estimate $u$ may move away from the preceding 
$v$ and vice-versa. 

\subsubsection{Efficient field densification}
As long as the dense deformation field is iteratively interpolated
from the same set of control points that define the $L-1$ contours,
it is possible to pre-cache all the $\psi_k(c_i)$ weights into a
sparse matrix for fast densification. As well, we achieve a 
diffeomorphic transform by ... 
\todo[inline]{this is actually false (IMHO), so far :(. We will need
to add some more conditions to the interopolating kernel so that
the original values of the transform in $c_i$ can be recovered from 
interpolated values of $u_k$}
