\section{Methods}
\label{sec:methods}
%
\subsection{Related work}
\label{sec:methods_background}

\Gls{acwe} and level-set based formulations have been widely applied in image
processing, as reviewed in \citep{suri_shape_2002}. We suggest clustering the 
current methodologies of template-based segmentation methods into three groups. 
The first group typically adds a shape prior term to the energy functional of 
an evolving active contour \citep{bresson_variational_2006,
chan_level_2005,chen_using_2002,cremers_kernel_2006,gastaud_combining_2004}.
These methods generally have an explicit description of the expected relative boundary 
locations of the object to be delineated, and some even model the statistical deviations
from this average shape. By including a coordinates mapping, it is possible to perform
active contours based registration between timesteps in a time-series or between different
images \citep{bertalmio_morphing_2000,wyatt_map_2003,paragios_level_2003,vemuri_joint_2003,
yezzi_variational_2003}.
A summary of these second set of techniques is performed in \citep{droske_mumfordshah_2009},
proposing two different approaches to applying the Mumford-Shah \citep{mumford_optimal_1989}
functional in joint registration and segmentation. Finally, a derivation of the 
latter groups is composed by atlas-based segmentation methods 
\citep{gorthi_segmentation_2009,gorthi_active_2011,pohl_unifying_2005,
pohl_bayesian_2006,wang_joint_2006}, where the prior 
imposes consistent voxel-based classification of contiguous regions.
A comprehensive summary of the \gls{acwe}-derived methodologies with special attention
to joint registration-segmentation methods is found in \citep{gorthi_active_2011}.

Our joint registration and segmentation method is mainly inspired in the active
deformation fields presented by \citep{gorthi_active_2011}, that is a generalization
of the methodologies under second and third groups before. As it is described in
their work, the first joint ``morphing''-segmentation approach (it was not a proper
registration method) was proposed in \citep{bertalmio_morphing_2000}. The first
registration framework is presented by \citep{yezzi_variational_2001} where the
energy functional is defined simultaneously in the moving and target images with
an affine transformation supporting the coordinates mapping (``2 PDEs approach'').
\citep{vemuri_joint_2003} proposed the first atlas-based registration based on the 
level set framework and using only one PDE. \citep{unal_coupled_2005} extended the 
idea of \citep{bertalmio_morphing_2000,yezzi_variational_2001} for non-linear 
registration with a dense deformation field as mapping function. \citep{droske_mumfordshah_2009}
combining ideas from both branches and proposed the propagation of the 
deformation field from the contours to the whole image definition. Finally,
\citep{gorthi_active_2011} present a comprehensive generalization of the methodologies.

All the surveyed frameworks are deduced from the general level set evolution equation
introduced by \citep{osher_fronts_1988}:
\begin{equation}
\frac{\partial \Phi_D(\mathbf{x},t)}{\partial t} = \nu ( \Phi_D(\mathbf{x},t)) \, \left| \nabla \Phi_D(\mathbf{x},t) \right|
\end{equation}
where $\nu$ is the velocity of the flow or speed function that contains the local
segmentation and contour regularization constraints, and $\Phi_D: \Omega \to \mathbb{R}$
is the signed distance function often used to represent implicitly the active contour
by its zero level. An early extension to image registration was proposed by 
\citep{vemuri_joint_2003} but replacing $\Phi_D$ by the intensity function of the
image to register, $\Phi_I: \Omega \to \mathbb{R}$ (the moving image). 
\citep{bertalmio_morphing_2000,vemuri_joint_2003} then define the speed function as
$\nu(\Phi_I(\mathbf{x},t)) = \Phi_I(\mathbf{x},t) - \Phi_T(\mathbf{x},t)$, where $\Phi_T(\mathbf{x})$ is the intensity
function of the target image. Finally, \citep{gorthi_active_2011} propose a different
multiphase label function as level set function ($\Phi_L$) that improves registration
results.

Naming $\Phi_G$ a generic objective function, a dense deformation field of vectors 
$u: \mathbb{R}^n \to \mathbb{R}^n$ (tipically $n = \{ 2, 3 \}$) is introduced. 
Then, the conservation of the morphological description is assumed, such as
$\Phi_G(\mathbf{x},t) = \Phi_G( \mathbf{x} + du, t + dt ) \implies d\Phi_G(\mathbf{x},t) = 0$, where $d\Phi_G$ is
the total derivative of $\Phi_G$. Using the chain rule, it can be rewritten as the 
evolution equation of a vector flow:
\begin{equation}
\frac{\partial u(\mathbf{x},t)}{\partial t}= - \frac{\Phi_G}{\left| \nabla \Phi_G \right|} N_{\Phi_G},
\end{equation}
where $N_{\Phi_G}$ is the normal of the level set. Finally, they introduce the general
level set evolution equation into the vector flow to perform the registration task:
\begin{equation}
\frac{\partial u(\mathbf{x},t)}{\partial t} = - \nu( \Phi_G(\mathbf{x},t) ) N_{\Phi_G}.
\end{equation}
Therefore, the position of the level set function $\Phi_G$ at time $t$ is given by the
deformation field $u(\mathbf{x},t)$ and the initial level set function $\Phi_G(\mathbf{x},0)$, yielding:
\begin{equation}
\frac{\partial u(\mathbf{x},t)}{\partial t} = - \nu( \Phi_G(\mathbf{x} + u(\mathbf{x},t), 0) ) N_{\Phi_G}.
\end{equation}

Here, we propose an active deformation field framework with some differences in
the formulation of the problem. The main diverging points with respect to
\citep{gorthi_active_2011} are: 1) there is no need for an explicit level set function
$\Phi_G$, as we replace the level set gradient computation $N_{\Phi_G}$ with shape
gradients \citep{jehan-besson_dream2s:_2003,herbulot_segmentation_2006}; 2) 
\todo[inline]{regularization is also based on linear diffusion smoothing REF-thirion,
but they use a Gaussian filtering)}; 3) optimization is applied in the spectral
domain, observing anisotropic and inhomogeneous mappings along each direction.


\subsection{Bridging segmentation frameworks}
\label{sec:methods_map}
%
A widely used approach to image segmentation is derived from the
Bayes' rule \eqref{eq:bayes_rule}. This framework seeks for a 
partitioning of a certain observation of features $F: \mathbb{R}^n \to %
\mathbb{R}^C$ (an $n$-dimensional \emph{image} comprising $C$ different 
channels) in piecewise smooth regions $\Omega = \lbrace \Omega_k , 
k\in\left[ 1 \ldots K \right] \rbrace$,  that maximize the \emph{a posteriori}
probability $p(\Omega \mid F)$:
\begin{equation}
p(\Omega \mid F) = \frac{p(F \mid \Omega)\, p(\Omega)}{p(F)},
\label{eq:bayes_rule}
\end{equation}
where $p(F \mid \Omega)$ is the \emph{likelihood} of the realization 
of $F$ given a certain region set $\Omega$. The second term, $p(\Omega)$,
is the a-priori probability of the partitioning. Finally, $p(F)$ is the 
probability of a certain image realization, and thus, it will remain 
constant when computing the \gls{map}. Consequently, the Bayes' rule
can be simplified for the optimization as follows:
\begin{equation}
p(\Omega \mid F) \propto p(F \mid \Omega)\, p(\Omega).
\label{eq:bayes_rule_simplified}
\end{equation}

In the simplest Bayesian segmentation framework, the unknown is the
parameter set $\Theta_k$ of region-wise descriptors that minimize 
\eqref{eq:bayes_rule}. Conversely, we can assume these descriptors as
fixed (fulfilling the concept of conservation of morphological description
mentioned before) and introduce a realization of a dense deformation field $U: %
\mathbb{R}^n \to \mathbb{R}^n$ that maps the initial partition $\Omega_0$
to the estimated one:
\begin{equation}
\Omega' = \Omega_0 + \hat{U}.
\label{eq:omega_tf}
\end{equation}

Thus, \eqref{eq:bayes_rule_simplified} is rewritten as 
$p(U \mid F) \propto p(F \mid U)\, p(U)$ that yields the
following \gls{map} solution:

\begin{equation}
\hat{U} = \underset{U}{\argmax} \left\{ p(U \mid F) \right\} = 
\underset{U}{\argmax} \left\{ p(F \mid U)\, p(U) \right\}.
\label{eq:map_u}
\end{equation}


An extended assumption is that the feature vector realization $F$ is
\emph{i.i.d.}, and thus, it is possible to write the a-posteriori
probability $p(F \mid U)$ as a continuous product with $d\mathbf{x}$ the
infinitesimal voxel size:
\begin{equation}
p(F \mid U) \, p(U) = \underset{k}{\prod} \underset{\mathbf{x}\in \Omega'_k}{\prod}
p_k\left( F(\mathbf{x} + U(\mathbf{x})) \mid U(\mathbf{x}) \right)^{d\mathbf{x}}.
\label{eq:bayes_aposteriori}
\end{equation}
\todo[inline]{I'm not convinced about turning $p(\Omega)$ into $p(U)$.}

A second widely-accepted assumption is the multivariate normal 
distribution of the different tissues in \gls{mri} data. Therefore,
the posterior probability of an infinitesimal voxel can be written as:
\begin{equation}
p_k( F(\mathbf{x}) \mid U(\mathbf{x}) ) = \frac{1}{ \sqrt{(2\pi)^{C}\,\left|\boldsymbol{\Sigma}_{k}\right|}}\,{e^{\left(-\frac{1}{2}  \Delta^2_k (\mathbf{f}') \right)}}.
\label{eq:bayes_mpdf}
\end{equation}
where we can identify the factor in the exponential as the squared \emph{Mahalanobis 
distance} of the feature observed in the displaced position, $\mathbf{f}' = 
F(\mathbf{x} + U(\mathbf{x}))$ with region descriptors
$\Theta_k = \lbrace \boldsymbol{\mu}_k, \boldsymbol{\Sigma}_k \rbrace$:
\begin{equation}
\Delta^2_k (\mathbf{f} \mid \Theta_k ) = (\mathbf{f} - \boldsymbol{\mu}_k)^T \, \boldsymbol{\Sigma}^{-1}_k \, (\mathbf{f} - \boldsymbol{\mu}_k).
\label{eq:bayes_mahalanobis}
\end{equation}

Finally, we can turn the \gls{map} problem into a variational one
applying the following log-transform:
\begin{multline}
E(F \mid U)= -\log \left[ p(F \mid U) \, p(U) \right] = \\
= -\log \left[ \underset{k}{\prod} \underset{\mathbf{x}\in \Omega'_k}{\prod}
p_k( F(\mathbf{x}) \mid U(\mathbf{x}) )^{d\mathbf{x}} \right] = \\
= \sum\limits_k \int_{\Omega'_k} -\log \left[ p_k(F(\mathbf{x}) \mid U(\mathbf{x} ) ) \right] \, d\mathbf{x},
\label{eq:energy_1}
\end{multline}
and introducing the posterior probability term \eqref{eq:bayes_mpdf}, 
we can express the functional in terms of the deformation field $U$ and
the region descriptors $\Theta_k$:
\begin{multline}
E(U) = \\
= \sum\limits_k \int_{\Omega'_k} -\log \left[ \frac{1}{ \sqrt{(2\pi)^{C}\,\left|\boldsymbol{\Sigma}_{k}\right|}}\,{e^{\left(-\frac{1}{2}  \Delta^2_k (\mathbf{f}) \right)}} \right] \, d\mathbf{x} = \\
= \sum\limits_k \int_{\Omega'_k} \left[ \frac{1}{2} \log{ \left( (2\pi)^{C}\,\left|\boldsymbol{\Sigma}_{k}\right| \right)} + \frac{1}{2}  \Delta^2_k (\mathbf{f}) \right] \,d\mathbf{x}.
\end{multline}
Finally, after removing scaling factors and independent constants,
we obtain:
\begin{align}
E(U) = \sum\limits_k \left[ \left|\Omega'_k\right|\,\log \left|\mathbf{\Sigma}_k \right| + \int_{\Omega'_k} \Delta^2_k (\mathbf{f}) \,d\mathbf{x} \right],
\label{eq:map_energy}
\end{align}
where we have a constant term scaled by the total volume of the partition $\Omega_k$ and 
the determinant of the  covariance matrix of the partition $\left|\boldsymbol{\Sigma}_{k}\right|$,
plus an energy term based on the squared \emph{Mahalanobis distance}.
As long as this shape-based registration framework is designed to allow small changes
between boundaries (and therefore, their volumes), and the shape descriptors should not
change significantly due to the conservation principle, the first term can be optionally
dismissed with respect the second one, what yields:
\begin{align}
E(U) = \sum\limits_k \int_{\Omega'_k} \Delta^2_k (\mathbf{f}) \,d\mathbf{x}.
\label{eq:final_map_energy}
\end{align}

This latter expression resembles the Mumford-Shah functional 
\citep{mumford_optimal_1989} which is at the background of all \gls{acwe}-based
methods. In this case, \eqref{eq:final_map_energy} includes covariance as
region descriptor, what modifies the original functional in a way that it 
can deal with more general distributions. This is necessary to avoid the 
assumption that regions $\Omega_k$ have a fixed covariance matrix on their 
complete domain. One immediate advantage of this functional from the original 
one is the possibility to distinguish regions with the same mean vector but 
different covariance matrix \citep{brox_local_2009}. In this work, the 
Mumford-Shah functional is derived for this extension as follows:
\begin{multline}
E(\Theta_k,\Omega_k) = \sum\limits_k \int_{\Omega_k} \left[ \log \left|\mathbf{\Sigma}_k\right| + \Delta^2_k (\mathbf{f}) \right] \,d\mathbf{x} \\
+ \lambda \int_{\Omega_k - K}  ( \left| \nabla \mathbf{\mu} \right| ^2 + \left| \nabla \mathbf{\Sigma}_k \right| ^2 ) \, d\mathbf{x} 
+ \nu |K|,
\end{multline}
that is easily identifiable with \eqref{eq:map_energy} when we apply 
the so-called \emph{cartoon limit}, 
for $\lambda \to \infty$:
\begin{equation}
E(\Theta_k,K) = \sum\limits_k \int_{\Omega_k} \left[ \log \left|\mathbf{\Sigma}_k\right| + \Delta^2_k (\mathbf{f}) \right] \,d\mathbf{x}
+ \nu |K|.
\end{equation}

As long as we do not penalize the edge set $K$ length, $\nu = 0$ and
the result is dual to \eqref{eq:map_energy}:
\begin{equation}
E(\Theta_k,K) = \sum\limits_k \int_{\Omega_k} \left[ \log \left|\mathbf{\Sigma}_k\right| + \Delta^2_k (\mathbf{f}) \right] \,d\mathbf{x}.
\end{equation}


\subsection{Introducing priors}
\label{sec:priors}
%
By introducing $U$ in \eqref{eq:bayes_rule_simplified}, we transformed 
the segmentation problem into a registration one provided with $\Omega_0$ 
(the initial partition) is derived from the shape-priors given in reference 
space. Therefore, the priors can be expressed in terms of the deformation
field:
\begin{equation}
P(U) = \underset{\mathbf{x}}{\prod}\,p(U(\mathbf{x}))\,^{d\mathbf{x}}.
\label{eq:prior_u}
\end{equation}
We define $P(U(\mathbf{x})) = p(\mathbf{u}) = \underset{d}\prod \, p_d(\mathbf{u})$ 
with $d$ the order of the derivative. We also assume that an initial
affine registration let us consider that the deformation field ($d=0$)
and its derivatives up to some order ($d>0$) are all zero mean, and
have some variance $\Sigma_d$. We restrict ourselves to $d=\lbrace 0,1 \rbrace$
(the field and its gradient):
\begin{subequations}
\begin{equation}
p_0(\mathbf{u}) = \mathcal{N}( \mathbf{u} \mid 0, \alpha^{-1}),
\end{equation}
\begin{equation}
p_1(\mathbf{u}) = \mathcal{N}( \nabla \mathbf{u} \mid 0, {\Sigma_\beta}^{-1}), 
\end{equation}
\end{subequations}
And therefore:
\begin{equation}
P(U) = \underset{\mathbf{x}}\prod \left[ p_0(\mathbf{u}) \, p_1(\mathbf{u}) \right]^{d\mathbf{x}}.
\end{equation}




\subsection{Numerical Implementation}
%
Let us denote $\{c_i\}_{i=1 \ldots N_c}$ the nodes of one or several shape-prior
surface(s). In our application, precise tissue interfaces of interest 
extracted from a high-resolution, anatomically correct reference volume. 
On the other hand, we have a number of \gls{dwi}-derived features at each
voxel of the volume. Let us denote by $\mathbf{x}$ the voxel and $f(\mathbf{x}) = 
[ f_1, f_2, \ldots, f_N]^T(\mathbf{x})$ its associated feature vector.

\subsubsection{Deformation field}
\label{sec:deformation_field}
The transformation from reference into \gls{dwi} coordinate space is 
achieved through a dense deformation field $u(\mathbf{x})$, such that:
\begin{equation}
c_i' = T\{c_i\} = c_i + u(c_i),
\end{equation}
what is equivalent to \eqref{eq:omega_tf}. Since the nodes of the anatomical 
surfaces might lay off-grid, it is required to derive $u(\mathbf{x})$ from a discrete 
set of parameters $\{u_k\}_{k=1 \ldots K}$. Densification is achieved through 
a set of associated basis functions $\psi_k$ (e.g. rbf, interpolation splines):
%
\begin{equation}
u(\mathbf{x}) = \sum_k \psi_k(\mathbf{x}) u_k
\end{equation}
%
In our implementation, $\psi_k$ is chosen to be a tensor-product B-Spline kernel.
Then, the transformation writes
%
\begin{equation}
\label{eq:transformation}
c_i' = T\{c_i\} = c_i + u(c_i) = c_i + \sum_k \psi_k(c_i)u_k
\end{equation} 

Based on the current estimate of the distortion $u$, we can compute 
``expected samples'' within the shape prior projected into the \gls{dwi}.
Thus, we now estimate region descriptors of the \gls{dwi} features 
$f(\mathbf{x})$ of the regions defined by the priors in \gls{dwi} space.
%
Using the region descriptors derived in \autoref{sec:methods_map}, we propose
an \gls{acwe}-like, piece-wise constant, variational image segmentation
model (where the unknown is the deformation field)
\cite{chan_active_2001} with the energy functional obtained in 
\eqref{eq:final_map_energy}. This inverse problem is ill-posed
\cite{bertero_ill-posed_1988,hadamard_sur_1902}.
In order to account for deformation field regularity and to render the 
problem well-posed, we include limiting and regularization terms into 
the energy functional \cite{morozov_linear_1975,tichonov_solution_1963}:
%
\begin{multline}
E(u) = \sum\limits_k \int_{\Omega'_k} \Delta^2_k (\mathbf{f}) \,d\mathbf{x} \\
+ \int u^T \, A \, u \, d\mathbf{x} + \int \tr\{(\nabla u^T)^T B (\nabla u^T)\} d\mathbf{x}
\label{eq:complete_energy}
\end{multline}
%
These regularity terms ensure that the segmenting contours in 
\gls{dwi} space are still close to their native shape. The model
easily allows to incorporate inhomogeneous and anisotropic 
regularization \cite{nagel_investigation_1986} to better regularize
the \gls{epi} distortion. \\
%

\subsubsection{Operator splitting}
In order to make the energy minimization computationally more tractable, 
we propose the following operator splitting: Let us optimize the data terms 
and the regularity terms on separate copies of the deformation field, 
now called $u$ and $v$, constrained to be equal:
\begin{multline}
E(u,v) = \sum\limits_k \int_{\Omega'_k(u)} \Delta^2_k (\mathbf{f}) \,d\mathbf{x} \\
+ \int v^T \, A \, v \, d\mathbf{x} + \int \tr\{(\nabla v^T)^T B (\nabla v^T)\} d\mathbf{x}
\label{eq:operator_splitting}
\end{multline}
and now
\begin{equation}
\min_{u,v} \{ E \} \quad s.t. \quad u = v.
\end{equation}

In order to take the equality constraint into account, we may make use of 
augmented Lagrangians (a combination of Lagrangian multipliers and penalty 
terms on the constraint) \cite{bertsekas_multiplier_1976,
glowinski_augmented_1989,nocedal_numerical_2006}:
\begin{equation}
AL(u,v,\lambda,r) = E(u,v) + \langle \lambda, u-v \rangle + \frac{r}{2} \| u - v \|_2^2
\end{equation}

To solve the constraint minimization problem, we may now optimize the 
Augmented Lagrangian in an iterative way:
\begin{equation}
\left\lbrace 
\begin{array}{rcl}
u^{t+1} &=& \argmin_{u} AL(u,v^t,\lambda^t,r)\\
v^{t+1} &=& \argmin_{v} AL(u^{t+1},v,\lambda^t,r)\\
\lambda^{t+1} &=& \lambda^t + \rho(u^{t+1}-v^{t+1})
\end{array}\right.
\end{equation}
where typically $0 < \rho < r$. The two sub-minimization problems will 
now be much easier to handle than the original complete problem.

\subsubsection{Shape gradients}
To compute the gradient-descent of the data-term domain integrals with 
respect to the underlying deformation field, we use shape gradients 
\cite{jehan-besson_dream2s:_2003,herbulot_segmentation_2006}. A little 
bit of theory is therefore in order.

Let $\Omega$ be an image domain and $d\Omega$ its boundary. Further, $r(\mathbf{x})$ 
is an ``arbitrary'' function over the image domain. We now derive the domain 
integral w.r.t. the contour evolution parameter $t$ (time):
\begin{equation}
\frac{\partial}{\partial t} \int_\Omega r(\mathbf{x}) d\mathbf{x} = \int_\Omega \frac{\partial r}{\partial t}(\mathbf{x}) d\mathbf{x} - \int_{d\Omega} r(\mathbf{x}) \left\langle \frac{\partial{d\Omega}}{\partial t}, N_{d\Omega}\right\rangle d\mathbf{x}
\label{eq:shape_gradients}
\end{equation}
where $\left\langle\frac{\partial{d\Omega}}{\partial t}, N_{d\Omega}\right\rangle$ is 
the projection of the boundary movement on the unit inward normal. We recall
here that equation \eqref{eq:shape_gradients} stands the bridge between the 
explicit level set formulations surveyed in \autoref{sec:methods_background} 
with the presented approach in setting up the velocity function.


\subsubsection{Minimization w.r.t. $u$}
The first minimization problem optimizes the data-term. The problem is equivalent 
to minimizing the following energy:
\begin{multline}
E(u,v) = \sum\limits_k \int_{\Omega'_k(u)} \Delta^2_k (\mathbf{f}) \,d\mathbf{x}
+ \langle \lambda, u-v \rangle + \frac{r}{2} \| u - v \|_2^2
\end{multline}
where we identify one instance of domain integrals of the form $\int_\Omega r(\mathbf{x}) 
d\mathbf{x}$ for each region $k$. Optimality requires the derivative of this energy 
with respect to the parameters $u$ to be null. At this point, we decide to ignore 
the influence of the boundary shift on the statistics of the regions (i.e. moving the 
boundary does not significantly impact the $\mu$ and $\Sigma$ descriptors). This means 
that we can drop the derivative of $r(\mathbf{x})$ w.r.t. contour evolution. 
What remains are the surface integrals at the subdomain interfaces, plus the Lagrangian 
and penalty terms:

\begin{multline}
\frac{\partial E}{\partial u_k} = \sum\limits_k \int_{d\Omega'_k} \left[ \Delta^2_{O(k)} (\mathbf{f}) - \Delta^2_k (\mathbf{f}) \right]
\left\langle\frac{\partial{d\Omega'_k}}{\partial t}, N_{d\Omega'_k}\right\rangle \,d\mathbf{x} \\
+ \lambda_k + r(u_k - v_k) = 0
\label{eq:energy_gradient}
\end{multline}
where $O(k)$ is a function that returns the neighboring region at $\mathbf{x}$ (i.e.
the region \emph{outside} the current location in the contour). For the shake of
simple notation, we will refer as $c'$ to the positions of points belonging to
any of the existing contours $d\Omega_k$.
Given the deformation field interpolation stated in \autoref{sec:deformation_field}, 
the boundary moves according to
\begin{equation}
\frac{\partial c'}{\partial u_k} = \psi_k(c')\,e_a
\end{equation}
where $e_a$ is the unit vector along direction $a \in \mathbb{R}^n$ and $n$ the dimension
of the image. Thus
\begin{equation}
\left\langle\frac{\partial c'}{\partial u_k}, N_{c'}\right\rangle = \psi_k(c')\,\hat{\mathbf{n}}_{c'}
\end{equation}
The discrete implementation of \eqref{eq:energy_gradient} is straightforward as
we have a triangularized mesh representation of the interfacing surfaces:
\begin{multline}
\frac{\partial E}{\partial u_k} = \sum\limits_k \sum\limits_{c'} \left[ \Delta^2_{O(k)} (\mathbf{f}) - \Delta^2_k (\mathbf{f}) \right]
\, \psi_k(c')\,\hat{\mathbf{n}}_{c'} \\
+ \lambda_k + r(u_k - v_k) = 0
\label{eq:discrete_energy_gradient}
\end{multline}
The optimal distortion $u_k$ is found at each iteration as:
\begin{multline}
u_k^{t+1} = v_k^t - \frac{1}{r}\lambda_k^{t} \\
- \frac{1}{r} \sum\limits_k \sum\limits_{c'} \left[ \Delta^2_{O(k)} (\mathbf{f}) - \Delta^2_k (\mathbf{f}) \right]
\, \psi_k(c')\,\hat{\mathbf{n}}_{c'}
\label{eq:update_u}
\end{multline}


\subsubsection{Minimization w.r.t. $v$}
For the optimization w.r.t. $v$, the relevant energy writes
\begin{align}
E(v) &= \int v^T A v d\mathbf{x} + \int \tr\{(\nabla v^T)^T B (\nabla v^T)\} d\mathbf{x}\\
&\quad + \langle \lambda, u-v \rangle + \frac{r}{2} \| u - v \|_2^2\nonumber
\end{align}
\todo[inline]{Do not assume this}
Let's assume the simplest, homogeneous isotropic case, $A = \alpha/2$ 
and $B = \beta/2$. The associated Euler-Lagrange equation is found as:
\begin{equation}
\alpha v - \beta\Delta v + rv = ru + \lambda
\end{equation}
which easily translates into Fourier domain:
\begin{equation}
v^{t+1} = \mathcal{F}^{-1}\left\{ \frac{\mathcal{F}\{ru + \lambda\}}{\mathcal{F}\{(\alpha+r)\mathcal{I}-\beta\Delta\}} \right\}
\end{equation}
where $\mathcal{I}$ denotes the identity operator.

Here, we rewrite the Laplacian as a linear combination of the identity and shift operators:
\begin{equation}
\Delta = \sum\limits_d \mathcal{S}_d^- + \mathcal{S}_d^+ - 2 \mathcal{I}
\end{equation}
where $\mathcal{S}_{d}^{\pm}$ stands for the forward ($+$) and backward ($-$) shift 
operator along coordinates axis $d$, of which the Fourier transform is found easily as
\begin{equation}
\mathcal{F}\{\mathcal{S}_{d}^{\pm}\} = e^{\pm i\omega_{d}},
\end{equation}
where $\omega_{d}$ is the normalized pulsation along $d$-direction. \todo{stop!} Accordingly, the 
Fourier transform of the discrete Laplacian is found as
\begin{align}
\mathcal{F}\{\Delta\} &= e^{-i\omega_x } + e^{i\omega_x } + e^{-i\omega_y } + e^{i\omega_y } - 4\nonumber\\
&= 2\left( \cos(\omega_x) + \cos(\omega_y) - 2 \right)
\end{align}

The remaining transforms are trivial or can be computed using FFT (as in \citep{estellers_efficient_2011}).


\subsubsection{Lagrangian multiplier update}
At each iteration, the Lagrangian multipliers are updated as noted before:
\begin{equation}
\lambda^{t+1} = \lambda^t + \rho(u^{t+1}-v^{t+1})
\end{equation}

\subsubsection{Region descriptor reestimation}
In regular intervals, i.e. after $n$ iterations, the parameters $\mu$ and $\Sigma$ of the involved regions need to be reestimated based on the shifted volumetric samples $w_j'$, $g_j'$ and $o_j'$.

\subsubsection{Convergence}
In order to ``fixate'' the evolution when close to convergence, it is advised to slightly increase the penalty weight $r$ at each iteration. Note that as can be seen in the above equations, $r$ governs the step-size or  leash-length at each iteration, i.e. the amount by which the new estimate $u$ may move away from the preceding $v$ and vice-versa. 
