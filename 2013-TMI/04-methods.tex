1\section{Methods}
\label{sec:methods}
%
\subsection{Related work}
\label{sec:methods_background}

We suggest clustering the 
current methodologies of template-based segmentation methods into three groups. 
The first group typically adds a shape prior term to the energy functional of 
an evolving active contour \citep{bresson_variational_2006,
chan_level_2005,chen_using_2002,cremers_kernel_2006,gastaud_combining_2004}.
These methods generally have an explicit description of the expected relative boundary 
locations of the object to be delineated, and some even model the statistical deviations
from this average shape. By including a coordinates mapping, it is possible to perform
active contours based registration between timesteps in a time-series or between different
images \citep{bertalmio_morphing_2000,wyatt_map_2003,paragios_level_2003,vemuri_joint_2003,
yezzi_variational_2003}.
A summary of these second set of techniques is performed in \citep{droske_mumfordshah_2009},
proposing two different approaches to applying the Mumford-Shah \citep{mumford_optimal_1989}
functional in joint registration and segmentation. Finally, a derivation of the 
latter groups is composed by atlas-based segmentation methods 
\citep{gorthi_segmentation_2009,gorthi_active_2011,pohl_unifying_2005,
pohl_bayesian_2006,wang_joint_2006}, where the prior 
imposes consistent voxel-based classification of contiguous regions.
A comprehensive summary of the \gls{acwe}-derived methodologies with special attention
to joint registration-segmentation methods is found in \citep{gorthi_active_2011}.
\Gls{acwe} and level-set based formulations have been widely applied in image
processing, as reviewed in \citep{suri_shape_2002}.

The first joint ``morphing''-segmentation approach (it was not a proper
registration method) was proposed in \citep{bertalmio_morphing_2000}. The first
registration framework is presented by \citep{yezzi_variational_2001} where the
energy functional is defined simultaneously in the moving and target images with
an affine transformation supporting the coordinates mapping (``2 \glspl*{pde} approach'').
\citep{vemuri_joint_2003} proposed the first atlas-based registration based on the 
level set framework and using only one \gls*{pde}. \citep{unal_coupled_2005} extended the 
idea of \citep{bertalmio_morphing_2000,yezzi_variational_2001} for non-linear 
registration with a dense deformation field as mapping function. \citep{droske_mumfordshah_2009}
combining ideas from both branches and proposed the propagation of the 
deformation field from the contours to the whole image definition. Finally,
\citep{gorthi_active_2011} present a comprehensive generalization of the methodologies.

Even though the historical evolution
of joint segmentation and registration procedures is strongly linked
to the level sets approximation, our work is based in derivations of
the \gls{acwe} framework \citep{chan_active_2001}, and related to
\citep{le_guyader_combined_2011}.
{\color{red} {The main diverging points with respect to
\citep{gorthi_active_2011} are: 1) there is no need for an explicit level set function
$\Phi_G$, as we replace the level set gradient computation $N_{\Phi_G}$ with shape
gradients \citep{jehan-besson_dream2s:_2003,herbulot_segmentation_2006};
2) regularization is also based on linear diffusion smoothing \citep{thirion_image_1998},
but we replace the Gaussian filtering by other constraints
studied in \citep{nagel_investigation_1986} to better the problem;
3) optimization is applied in the spectral
domain, observing anisotropic and inhomogeneous mappings along each direction.
With respect \citep{le_guyader_combined_2011}, the main differences are
the distance function, and the spectral solution to the optimization updates,
as we shall cover in \autoref{sec:numerical_implementation}.}}

\todo[inline]{The previous paragraphs in red color are marked by Dominique
as ``to rework'', and I totally agree}


\subsection{Proposed model}
\label{sec:methods_map}
In this paper we formulate the joint registration-segmentation
problem as follows. We look for a spatial mapping (with, typically,
$n=3$ for the 3-dimensional case):
\begin{align}
U\colon T \subset \mathbb{R}^n &\to D \subset \mathbb{R}^n \notag\\
\vec{x} &\mapsto \vec{x}' =\vec{x}+u(\vec{x}),
\label{eq:deformation_field}
\end{align}
such that the known contours in anatomical space $T$ optimally segment
the diffusion space $D$.

In a Bayesian framework, different such mappings $U$ between ``T1-''
and ``\gls*{dwi}-'' spaces are evaluated based on their posterior
probability given the observed data $F$. Using the Bayes' rule, this
posterior \emph{likelihood} can be computed as:
\begin{equation}
P(U \mid F,\Gamma_{l,m}) = \frac{P(F \mid U,\Gamma_{l,m})\, P(U)}{P(F)},
\label{eq:bayes_rule}
\end{equation}
where $P(F \mid U,\Gamma_{l,m})$ is the data-likelihood, and
$\Gamma_{l,m}$ are a set of contours in T1-space that correspond to
two interfacing regions $\Omega$, such that 
$\Gamma_{l,m} = \partial \Omega_l \cap \partial \Omega_m$.

We are interested in a mapping $\hat{U}$ which maximizes this
posterior probability (\gls*{map} estimate, \citep{bishop_pattern_2006}):
\begin{align}
\hat{U} &= \underset{U}{\argmax} \left\{ P(U \mid F,\Gamma_{l,m}) \right\} \notag\\
 &= \underset{U}{\argmax} \left\{ P(F \mid U,\Gamma_{l,m})\, P(U) \right\}.
\label{eq:map_u}
\end{align}

First, we assume independence between pixels, and thus break down the
global data likelihood into a product of pixel-wise conditional probabilities:

\begin{equation}
P(F \mid U,\Gamma_{l,m}) = 
\underset{\vec{x}\in \Omega}{\prod} P\left( F(\vec{x}') \mid U(\Gamma_{l,m}) \right).
\label{eq:bayes_aposteriori}
\end{equation}

Further, within each region $\Omega_l$ defined by the contours $\Gamma$
trough duality, we assume the features to be i.i.d.:
\begin{equation}
P\left( F(\vec{x}') \mid U,\Gamma_{l,m} \right) = p_l( F(\vec{x}')) \quad
\forall \vec{x} \in \Omega_l.
\label{eq:likelihood}
\end{equation}

For convenience, and because this has been shown to be an appropriate
approximation \citep{bishop_pattern_2006}, each such region is modeled
by a multivariate normal distribution with region descriptors
$\Theta_l = \lbrace \boldsymbol{\mu}_l, \boldsymbol{\Sigma}_{l} \rbrace$:

\begin{equation}
p_l( F(\vec{x}') ) = \mathcal{N} \left( F(\vec{x}') \mid \Theta_l \right).
\label{eq:multivariate_normal}
\end{equation}

\todo{I modified this slightly}
Thus, defining the feature observed as $\vec{f}=F(\vec{x})$ and 
the squared \emph{Mahalanobis distance} of $\vec{f}$ with respect
of the descriptors of region $l$ as
\begin{equation}
\mdist{f}{l} = (\vec{f} - \boldsymbol{\mu}_l)^T \, {\boldsymbol{\Sigma}_l}^{-1} \, (\vec{f} - \boldsymbol{\mu}_l),
\end{equation}
we write:
\begin{align}
P( F \mid U, \Omega ) &= \underset{l}{\prod} \underset{\vec{x} \in \Omega_l}{\prod}
\mathcal{N} ( \vec{f} \mid \boldsymbol{\mu}_l, \boldsymbol{\Sigma}_{l} ) \notag\\
&= \frac{1}{ \sqrt{(2\pi)^{C}\,\left|\boldsymbol{\Sigma}_{l}\right|}}\,{e^{\left(-\frac{1}{2}
\mdist{f'}{l} \right)}}.
\label{eq:pdf}
\end{align}

We choose a Thikonov regularization prior as follows:
\begin{equation*}
P(U) = \underset{\vec{x}}{\prod} p(u(\vec{x}))
\end{equation*}
with further:
\begin{align*}
p(u(\vec{x})) &= p_0(u(\vec{x})) \, p_1(u(\vec{x})) \notag\\
p_0(u(\vec{x})) &= \mathcal{N}( u(\vec{x}) \mid 0, A^{-1}) \notag\\
p_1(u(\vec{x})) &= \mathcal{N}( Du(\vec{x}) \mid 0, B^{-1}),
\end{align*}
that consider that the distortion and its gradient have zero
mean and variance governed by $A$ and $B$. Since the anisotropy
is technically aligned with the imaging axes, these can
be simplified:
\begin{align}
p_0(u(\vec{x})) &= \underset{\vec{x}}{\sum} \mathcal{N}( u(\vec{x}) \mid 0, (\boldsymbol{\alpha}^{\circ\frac12}\,\vec{I}_n)^{-1}) \notag\\
p_1(u(\vec{x})) &= \underset{\vec{x}}{\sum} \mathcal{N}( \nabla \cdot u(\vec{x}) \mid 0, (\boldsymbol{\beta}^{\circ\frac12}\,\vec{I}_n)^{-1})
\label{eq:priors}
\end{align}

Finally, we can turn the \gls{map} problem into 
a variational one
applying the following log-transform:
\begin{align}
E(F \mid U) &= -\log \underset{l}{\prod}
\underset{\vec{x} \in \Omega_l}{\prod} 
p_l( \vec{f}')\,p_0( u(\vec{x}))\,p_1( u(\vec{x})) \notag\\
&= C + \underset{l}{\sum}
\underset{\vec{x} \in \Omega_l}{\sum}
\mdist{f'}{l} \notag\\
&+ \underset{\vec{x} \in \Omega}{\sum} \boldsymbol{\alpha} \cdot u(\vec{x})^{\circ2}
+ \underset{\vec{x} \in \Omega}{\sum} \boldsymbol{\beta} \cdot (\nabla \cdot u(\vec{x}))^{\circ2},
\label{eq:energy}
\end{align}
that is the dual expression to the energy functional corresponding
to a discrete \gls*{acwe} framework \citep{chan_active_2001}
with anisotropic regularization as studied in
\citep{nagel_investigation_1986}.

{\color{red} {Using the region descriptors derived in \autoref{sec:methods_map}, we propose
an \gls{adf}-like, \gls{acwe}-based, piece-wise constant, image segmentation
model (where the unknown is the deformation field)
\cite{chan_active_2001} with the energy functional obtained in 
\eqref{eq:final_map_energy}. This inverse problem is ill-posed
\cite{bertero_ill-posed_1988,hadamard_sur_1902}.
In order to account for deformation field regularity and to render the 
problem well-posed, we include limiting and regularization terms into 
the energy functional \cite{morozov_linear_1975,tichonov_solution_1963}:
%
\begin{multline}
E(u) = \sum\limits_l \int_{\Omega'_l} \mdist{f}{l} \,d\vec{x} \\
+ \int \vec{u}^T \, A \, \vec{u} \, d\vec{x} + \int \tr\{(\nabla \vec{u}^T)^T B (\nabla \vec{u}^T)\} d\vec{x}
\label{eq:complete_energy}
\end{multline}
%
These regularity terms ensure that the segmenting contours in 
\gls{dwi} space are still close to their native shape. The model
easily allows to incorporate inhomogeneous and anisotropic 
regularization \cite{nagel_investigation_1986} to better regularize
the \gls{epi} distortion.}}

\todo[inline]{this last paragraphs in red color were located a bit
further, but I think we can keep some of the refs and still reproduce
the continuous version of the energy functional}

\subsection{Numerical Implementation}
\label{sec:numerical_implementation}
%
Let us denote $\{\vec{c}_i\}_{i=1 \ldots N_c}$ the nodes of one or several shape-prior
surface(s). In our application, precise tissue interfaces of interest 
extracted from a high-resolution, anatomically correct reference volume. 
On the other hand, we have a number of \gls{dwi}-derived features at each
voxel of the volume. Let us denote by $\vec{x}$ the voxel and $F(\vec{x}) = 
[ f_1, f_2, \ldots, f_N]^T(\vec{x})$ its associated feature vector.

\subsubsection{Deformation field}
\label{sec:deformation_field}
The transformation from ``T1-'' into ``\gls{dwi}-'' coordinate space is 
achieved through a dense deformation field $u(\vec{x})$, such that:
\begin{equation}
\vec{c}_i' = U\{\vec{c}_i\} = \vec{c}_i + u(\vec{c}_i),
\label{eq:nodes_tfm}
\end{equation}
where $U$ is defined in \eqref{eq:deformation_field}. Since the nodes of the anatomical 
surfaces might lay off-grid, it is required to derive $u(\vec{x})$ from a discrete 
set of parameters $\{\vec{u}_k\}_{k=1 \ldots K}$. Densification is achieved through 
a set of associated basis functions $\psi_k$:
%
\begin{equation}
u(\vec{x}) = \sum_k \psi_k(\vec{x}) \vec{u}_k
\label{eq:intp_kernel}
\end{equation}
%
In our implementation, $\psi_k$ is chosen to be a tensor-product B-Spline kernel
of degree 3 ($B_3$).
Then, introducing \eqref{eq:intp_kernel} into \eqref{eq:nodes_tfm} and replacing
$\psi$ by the actual kernel function, the transformation writes
%
\begin{equation}
\vec{c}_i' = \vec{c}_i + \sum_k \left[ \vec{u}_k \, \underset{d}{\prod} B_3( (\vec{c}_i - \vec{x}_k) \cdot \hat{\mathbf{e}}_d ) \right].
\label{eq:transformation}
\end{equation} 

Based on the current estimate of the distortion $\vec{u}$, we can compute 
``expected samples'' within the shape prior projected into the \gls{dwi}.
Thus, we now estimate region descriptors of the \gls{dwi} features 
$\vec{f}'$ of the regions defined by the priors in \gls{dwi} space.

\subsubsection{Gradient descent}
To find the minimum of the energy functional \eqref{eq:energy},
we propose a gradient-descent approach with respect to the underlying 
deformation field through the following \gls*{pde}:
\begin{equation}
\frac{\partial u(\vec{x},t)}{\partial t} \propto - \frac{\partial E(\vec{u})}{\partial \vec{u}},
\label{eq:general_gradient_descent}
\end{equation}
with $t$ being an artificial time parameter of the contour
evolution, included to express the iterative character of the approach.
Now, we can introduce \eqref{eq:energy} in
\eqref{eq:general_gradient_descent}:
\begin{align}
\frac{\partial E(\vec{u})}{\partial \vec{u}_k} &=
\frac{ \partial }{\partial \vec{u}_k} \Big\{
C + \underset{l}{\sum}
\underset{\vec{x} \in \Omega_l}{\sum} \mdist{f'}{l} \notag\\
&+ \underset{\vec{x} \in \Omega}{\sum} \boldsymbol{\alpha} \cdot u(\vec{x})^{\circ2}
+ \underset{\vec{x} \in \Omega}{\sum} \boldsymbol{\beta} \cdot (\nabla \cdot u(\vec{x}))^{\circ2}
\Big\}.
\label{eq:gradient_descent}
\end{align}

Whereas related \glspl*{adf} introduced in \autoref{sec:methods_background}
make use of explicit level-set formulations to solve \eqref{eq:gradient_descent},
we alternatively use \emph{shape-gradients}
\cite{jehan-besson_dream2s:_2003,herbulot_segmentation_2006}.
Let $r(\vec{x})$ be an ``arbitrary'' function over the image domain,
and $\Omega$ a certain image region with $\Gamma_{l,m}$ its corresponding
outer boundary as defined in \autoref{sec:methods_map}.
We now derive the domain integral w.r.t. $t$:
\begin{equation*}
\frac{\partial}{\partial t} \int_\Omega r(\vec{x}') d\vec{x} = \int_\Omega \frac{\partial}{\partial t}r(\vec{x}') d\vec{x} - \int_{\Gamma_{l,m}} r(\vec{x}') \left\langle \frac{\partial \Gamma_{l,m} }{\partial t}, N_{\Gamma_{l,m}}\right\rangle d\vec{x}
\end{equation*}
where $\left\langle\frac{\partial\Gamma_{l,m}}{\partial t}, N_{\Gamma_{l,m}}\right\rangle$ is 
the projection of the boundary movement on the unit inward normal $N_{\Gamma_{l,m}}$. Assuming
that the region descriptors $\Theta_l$ vary slowly enough, we can consider
that $\frac{\partial}{\partial t} r(\vec{x}') = 0$ and thus:
\begin{equation}
\frac{\partial}{\partial t} \int_\Omega r(\vec{x}') d\vec{x} = 
- \int_{\Gamma_{l,m}} r(\vec{x}') \left\langle \frac{\partial \Gamma_{l,m} }{\partial t}, N_{\Gamma_{l,m}}\right\rangle d\vec{x}
\label{eq:shape_gradients}
\end{equation}

Therefore, we can apply a discretized interpretation of \eqref{eq:shape_gradients}
to compute the data term in \eqref{eq:gradient_descent} as follows:
\begin{align}
\frac{\partial E_{data}(\vec{u})}{\partial \vec{u}_k} &=
\underset{l}{\sum} \frac{ \partial }{\partial \vec{u}_k} \left\{
 \underset{\vec{x} \in \Omega_l}{\sum} \mdist{f'}{l} \right\} \notag\\
&= \underset{l,m}{\sum} \underset{i}{\sum}
\left[ \mdist{f_i'}{l} - \mdist{f_i'}{m} \right]
\left\langle \frac{\partial \vec{c}_i'}{\partial \vec{u}_k}, \vec{n_i}'\right\rangle,
\label{eq:gradient_wshape}
\end{align}
in this case, the formulation has been adapted to the non-binary case, $l,m$
being any pair of neighboring regions, and $\Gamma_{l,m}$ the contour separating
them such that $\vec{x}' = \vec{c}' \in\Gamma_{l,m} \iff \vec{x}\in \partial\Omega_i \cap \partial\Omega_j$
and $\vec{n_i}'$ is the unit inward normal to the contour at $c_i'$.

Finally, we can compute:
\begin{align}
\frac{\partial \vec{c}_i'}{\partial \vec{u}_k} &= \frac{\partial}{\partial \vec{u}_k} 
\left\{ \vec{c}_i + \sum_k \psi_k(\vec{c}_i) \vec{u}_k \right\} 
= \psi_k(\vec{c_i})\, \hat{\vec{e}}
\end{align}
where $\hat{\vec{e}}$ is the coordinates system's unit vector.
Projecting this gradient onto the surface normal,
$\left\langle \frac{\partial}{\partial \vec{u}_k}{\vec{c}_i}', \vec{n_i}'\right\rangle
= \psi_k(\vec{c_i})\, \hat{\vec{n}}_i$, then the
full gradient evolution equation \eqref{eq:gradient_descent} yields:
\begin{align}
\frac{\partial E(\vec{u}_k)}{\partial \vec{u}_k} =
&- \underset{l,m}{\sum} \underset{i}{\sum}
\left[ \mdist{f_i'}{l} - \mdist{f_i'}{m} \right]
\psi_k(\vec{c}_i)\, \hat{\vec{n}}_i \notag\\
&+2\, \boldsymbol{\alpha} \vec{u}_k
-2\, \boldsymbol{\beta} \Delta \vec{u}_k,
\label{eq:gradient_final}
\end{align}

\subsubsection{Semi-implicit Euler-forward optimization}
It is necessary to discretize $t$ in order to obtain a numerical
implementation of the equation \eqref{eq:gradient_final}:
\begin{align}
\frac{\vec{u}_k^{t+1}-\vec{u}_k^t}{\tau} =
&- \underset{l,m}{\sum} \underset{i}{\sum}
\left[ \mdist{f_i'}{l} - \mdist{f_i'}{m}\right]
\psi_k(\vec{c}_i)\, \hat{\vec{n}}_i \notag\\
&+2\, \boldsymbol{\alpha} \vec{u}_k^{t+1}
-2\, \boldsymbol{\beta} \Delta \vec{u}_k^{t+1}.
\end{align}

The associated Euler-Lagrange equation is found as:
\begin{equation}
(\tau^{-1} +2\, \boldsymbol{\alpha} - 2\, \boldsymbol{\beta} \Delta )\, \vec{u}_k^{t+1} = 
\tau^{-1} \vec{u}_k^t - \frac{\partial}{\partial \vec{u}_k} E_{data}(\vec{u}_k^t),
\end{equation}
that is a linear system that we translate into Fourier domain,
to obtain the next deformation field $\vec{u}_k^{t+1}$:
\begin{multline}
\vec{u}_k^{t+1} = \\
 \mathcal{F}^{-1} \left\lbrace
\frac{\mathcal{F}\lbrace \tau^{-1}\vec{u}_{k}^{t} - \underset{l,m}{\sum} \underset{i}{\sum}
\left[ \mdist{f_i'}{l} - \mdist{f_i'}{m}\right] \rbrace}
     {\mathcal{F}\lbrace (\tau^{-1} +2\,\boldsymbol{\alpha})\mathcal{I} - 2\,\boldsymbol{\beta} \Delta ) \rbrace}
     \right\rbrace
\end{multline}

Here, we rewrite the Laplacian as a linear combination of the identity and shift operators:
\begin{equation}
\Delta = \sum\limits_d \mathcal{S}_d^- + \mathcal{S}_d^+ - 2 \mathcal{I}
\end{equation}
where $\mathcal{S}_{d}^{\pm}$ stands for the forward ($+$) and backward ($-$) shift 
operator along coordinates axis $d$, of which the Fourier transform is found easily as
\begin{equation}
\mathcal{F}\{\mathcal{S}_{d}^{\pm}\} = e^{\pm \vec{i}\omega_{d}},
\end{equation}
where $\omega_{d}$ is the normalized pulsation along $d$-direction. Accordingly, the 
Fourier transform of the discrete Laplacian is found as
\begin{equation}
\mathcal{F}\{\Delta\} = \sum\limits_d e^{-\vec{i}\omega_d } + e^{\vec{i}\omega_d } - 2 = \sum\limits_d \left( 2\cos(\omega_d) - 2 \right)
\end{equation}

The remaining transforms are trivial or can be computed using the \gls{fft} 
as in \citep{estellers_efficient_2011}.

\subsubsection{Region descriptor reestimation}
In regular intervals, after certain number of iterations,
the parameters $\Theta_l$ of the regions can be reestimated 
based on the shifted volumetric samples 
$\vec{x}' = \vec{x_0} + u(\vec{x})$.

\subsubsection{Convergence}
\todo[inline]{discuss choice of $\tau$, Courant-Friedrichs-Lewy (CFL) condition, etc.}

\subsubsection{Efficient field densification}
As long as the dense deformation field is iteratively interpolated
from the same set of control points that define the $L-1$ contours,
it is possible to pre-cache all the $\psi_k(c_i)$ weights into a
sparse matrix for fast densification. As well, we achieve a 
diffeomorphic transform by
